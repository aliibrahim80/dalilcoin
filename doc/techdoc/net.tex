The intention of the {\module{net}} module ({\file{net.ml}} and {\file{net.mli}}) is to create and handle connections
to remote nodes.
Not everything has been implemented, but there is enough implemented
to form connections between peers and pass messages containing inventory,
transactions, block headers and block deltas.

\section{Network Messages}

There are several different message types enumerated by the type {\type{msgtype}}.
These largely follow the message types of Bitcoin's messages.
\begin{itemize}
\item {\constr{Version}}: used (only) in the handshake protocol, which is similar to Bitcoin's handshake.
\item {\constr{Verack}}: used (only) in the handshake protocol, which is similar to Bitcoin's handshake.
\item {\constr{Addr}}: intended to be used as in Bitcoin, but is currently unused.
\item {\constr{Inv}}: inventory message with a list of triples $(m,h,k)$ where $m$ is a message type (as a byte), $h$ is a block height (which is often irrelevant) and $k$ is a hash value (identifying some corresponding data).
\item {\constr{GetData}}: currently unused
\item {\constr{MNotFound}}: currently unused
\item {\constr{GetSTx}}: request a transaction with its signatures (see Chapter~\ref{chap:assetstx}).
\item {\constr{GetTx}}: request a transaction (see Chapter~\ref{chap:assetstx}).
\item {\constr{GetTxSignatures}}: request signatures for a transaction (see Chapter~\ref{chap:assetstx}).
\item {\constr{GetHeader}}: request a block header (see Chapter~\ref{chap:block}).
\item {\constr{GetHeaders}}: request several block headers (see Chapter~\ref{chap:block}).
\item {\constr{GetBlock}}: request a block (currently unused, as headers and deltas are requested independently).
\item {\constr{GetBlockdelta}}: request a block delta (see Chapter~\ref{chap:block}).
\item {\constr{GetBlockdeltah}} (deprecated)
\item {\constr{STx}}: a transaction with its signatures (see Chapter~\ref{chap:assetstx}).
\item {\constr{Tx}}: a transaction (see Chapter~\ref{chap:assetstx}).
\item {\constr{TxSignatures}}: signatures for a transaction (see Chapter~\ref{chap:assetstx}).
\item {\constr{Block}}: a block (currently unused, as headers and deltas are sent independently).
\item {\constr{Headers}}: a block header (see Chapter~\ref{chap:block}).
\item {\constr{Blockdelta}}: a block delta (see Chapter~\ref{chap:block}).
\item {\constr{Blockdeltah}} (deprecated)
\item {\constr{GetAddr}}: currently unused
\item {\constr{Mempool}}: currently unused
\item {\constr{Alert}}: currently unused
\item {\constr{Ping}}: intended to be used to ensure a connection is alive, but currently unused.
\item {\constr{Pong}}: intended to be used to ensure a connection is alive, but currently unused.
\item {\constr{Reject}}: currently unused
\item {\constr{GetCTreeElement}}: request an compact tree element (see Chapter~\ref{chap:ctre}).
\item {\constr{GetHConsElement}}: request an hcons element (see Chapter~\ref{chap:ctre}).
\item {\constr{GetAsset}}: request an asset (see Chapter~\ref{chap:assetstx}).
\item {\constr{CTreeElement}}: an compact tree element (see Chapter~\ref{chap:ctre}).
\item {\constr{HConsElement}}: an hcons element (see Chapter~\ref{chap:ctre}).
\item {\constr{Asset}}: an asset (see Chapter~\ref{chap:assetstx}).
\item {\constr{Checkpoint}}: a checkpoint (see Chapter~\ref{chap:blocktree}).
\item {\constr{AntiCheckpoint}}: currently unused
\item {\constr{NewHeader}}: sending a newly staked header (deprecated)
\item {\constr{GetCheckpoint}}: request a checkpoint (see Chapter~\ref{chap:blocktree}).
\end{itemize}
Message types can be converted to integers and strings by {\func{int\_of\_msgtype}}
and {\func{string\_of\_msgtype}}, respectively.
{\func{msgtype\_of\_int}} computes the message type of an integer (raising {\exc{Not\_found}}
if the integer is negative or greater than 36).

The function {\func{inv\_of\_msgtype}} computes {\func{int\_of\_msgtype}} for a converted
message type. For example, {\constr{GetTx}} is converted to {\constr{Tx}}.
In general the message type is converted from a request message type to a response message type
in order to keep up with what inventory a remote peer has advertised
in relation to what inventory has been requested.

Messages have the following format:
\begin{itemize}
\item Four byte magic number: 0x51656454 for testnet, 0x5165644d for mainnet.
\item Either the byte $0$ or the byte $1$ followed by a 20 byte hash value $h$ indicating that the message is a reply to a specific request (identified by $h$).
\item A byte indicating the message type (\type{msgtype}).
\item Four bytes giving the length $l$ of the payload.
\item 20 bytes giving the hash of the payload.
\item The payload in $l$ bytes, which must hash (via {\func{hash160}}) to give the correct hash value.
\end{itemize}

Three exceptions are exported from {\module{net}}:
\begin{itemize}
\item {\exc{GettingRemoteData}}: indicates some data is being requested from peers and this data is required
in order to complete the current computation. This exception is not used in {\module{net}} itself, but is raised
by other code when a request is sent to a peer and the computation will not wait for a reply.
\item {\exc{RequestRejected}}: indicates that a socks4 connection failed.
\item {\exc{IllformedMsg}}: indicates a message had a bad format.
\end{itemize}

The internal function {\func{rec\_msg}} reads a message from a channel, given a block height.
The block height is required to know the maximum allowed length of a message payload
(computed from {\func{maxblockdeltasize}}).
Since the current block height changes, there is an exposed reference {\var{netblkh}}
which is used by connection threads to determine the current block height.
The value {\var{netblkh}} is updated by code that tracks the current best block (see Chapters~\ref{chap:blocktree} and~\ref{chap:stk}).
The exception {\exc{IllformedMsg}} is raised
by {\func{rec\_msg}} unless the message has the correct format.

The internal function {\func{send\_msg}} handles formatting and sending a message.
In particular, it is given an output channel, the hash of the payload,
an optional hash indicating what the message is a reply to,
the message type
and the message payload as a string (to be interpreted as a sequence of bytes).

\section{Network Connections}

Network connections have a state summarized in the record type {\type{connstate}} consisting of the following fields:
\begin{itemize}
\item {\field{conntime}}: the time the connection was created.
\item {\field{realaddr}}: the address of the connecting socket.
\item {\field{connmutex}}: a ``mutex'' to prevent race conditions when communicating with the peer.
\item {\field{sendqueue}}: messages on the queue to be sent by the connection sender thread to the peer.
\item {\field{sendqueuenonempty}}: a condition to wake up the connection sender thread to send a message on the queue.
\item {\field{handshakestep}}: an integer indicating the current handshake step, which progresses to 5 at which point the handshake protocol has been completed.
\item {\field{peertimeskew}}: the number of seconds of skew between the peers (computed from the advertised time in the handshake).
\item {\field{protvers}}: the protocol version advertised by the peer in the handshake.
\item {\field{useragent}}: the user agent given by the peer in the handshake.
\item {\field{addrfrom}}: the string giving the address the peer gave in the handshake.
\item {\field{banned}}: a boolean which becomes true if the connection is banned due to misbehavior.
\item {\field{lastmsgtm}}: the time of the last message exchanged.
\item {\field{pending}}: a list of message hashes where the peer is expected to give a response, as well as information about
how long the node has waited and a function for handling the reply if and when it is received.
\item {\field{sentinv}}: a list of inventory already sent to the peer
\item {\field{rinv}}: a list of remote inventory
\item {\field{invreq}}: a list of requested inventory
\item {\field{first\_header\_height}}: intended to be the block height at which the peer has headers (currently unused)
\item {\field{first\_full\_height}}: intended to be the block height at which the peer has block headers and deltas (currently unused)
\item {\field{last\_height}}: indended to be the latest block height of the peer (currently unused)
\end{itemize}

All current connections are stored in the ref list {\var{netconns}}.
Each entry of {\var{netconns}}
has two threads (the ``connection listener'' thread and the ``connection sender'' thread), a file descriptor and its corresponding input and output channels,
and a reference to an optional connection state ({\type{connstate}}).
If the reference becomes None, then the connection is considered dead.

\section{Network Listener}

If an ip address to listen for connections is given, the function {\func{openlistener}}
creates a listener socket waiting for connections.
This listener socket is sent to the function {\func{netlistener}} in its own thread
upon startup (see {\file{qeditas.ml}}).
The variable {\var{netlistenerth}} contains the thread in case it is required later.

The function {\var{netlistener}} is an infinite loop accepting connections
and
removing dead connections (those with no connection state) using {\func{remove\_dead\_conns}}.
A connection is accepted if it is not a self-connect (as judged by the value of {\var{this\_nodes\_nonce}})
and if there are fewer than {\var{maxconns}} connections on {\var{netconns}}.
An initial connection state ({\type{connstate}}) is created with {\field{handshakestep}} set to $1$.
A connection listener thread is created calling {\func{connlistener}}
and a connection sender thread is created calling {\func{connsender}}.
The triple of the two threads and (a reference to the optional) connection state is added to {\var{netconns}}.

{\func{connsender}} is a loop which 
sends the messages on {\field{sendqueue}} to the peer
and then waits for the condition {\field{sendqueuenonempty}}.
If at any time the connection state of the connection is set to {\val{None}},
the thread ends.
Likewise, certain exceptions ({\exc{Unix\_error}}, {\exc{End\_of\_file}}, {\exc{ProtocolViolation}}, {\exc{SelfConnection}})
result in closing the socket, setting the connection state to {\val{None}} and ending the thread.
All other exceptions are logged but otherwise ignored.

{\func{connlistener}} is a loop which waits for messages (using {\func{rec\_msg}})
and calls {\func{handle\_msg}} to handle the message.
If after handling the message {\field{banned}} is {\val{true}}, then the
exception {\exc{ProtocolViolation}} is raised.
If at any time the connection state of the connection is set to {\val{None}},
then the connection is considered dead and {\exc{End\_of\_file}} is raised.
Certain exceptions ({\exc{Unix\_error}}, {\exc{End\_of\_file}}, {\exc{ProtocolViolation}}, {\exc{SelfConnection}})
result in closing the socket, setting the connection state to {\val{None}} and ending the thread.
All other exceptions are logged but otherwise ignored.

The function {\func{handle\_msg}} behaves differently depending on whether the message is a reply.
If the message is a reply to a message with hash value $h$, then
the associated value is looked up on the {\field{pending}} field of the connection state
(and removed from {\field{pending}}).
This gives a function $f$ which is called with the message type and message payload.
If no associated value is on {\field{pending}}, the message is ignored.

If the message is not a reply, then a distinction is made based on whether or not the
connection is still in the handshake phase.
If the connection is in the handshake phase, then the message must be either a
{\constr{Version}} or {\constr{Verack}} message
and is handled appropriately, updating the connection state.
Once the handshake protocol has been completed, {\field{handshakestep}} will be $5$
and
the function given by {\var{send\_inv\_fn}} is called to send the initial
inventory message to the peer.
(Initially, {\var{send\_inv\_fn}} is set to a trivial function that does nothing,
but the value is reset to the {\func{send\_inv}} function in {\module{blocktree}}.
{\func{send\_inv}} in {\module{blocktree}} gathers recent block headers, block deltas and transactions
to send as inventory.)

If the connection is not in the handshake phase, then
a function $f$ to handle the message type is looked up in the hash table {\var{msgtype\_handler}}
and this $f$ is called on the channels, the connection state and the message payload.
If there is no handler function is found in {\var{msgtype\_handler}},
then {\exc{Failure}} is raised.

The particular functions associated with message types in {\var{msgtype\_handler}}
are given in later modules (\module{blocktree}, \module{ctre} and \module{assets}). (The reason is that only in later modules are the types for
the deserialized data in the message available.)
In many cases, no handler is given (due to the incomplete nature of the Qeditas implementation)
meaning the messages will lead to a {\exc{Failure}} exception being raised.

\section{Network Seeker}

The function {\func{netseeker}} is used to start a thread
which tries to use known peers (from past connections stored in a {\file{peers}} file)
or fallback nodes (see {\var{testnetfallbacknodes}} and {\var{fallbacknodes}}
which are currently empty) to initiate
new connections.
The main code is in {\func{netseeker\_loop}}.
The thread is stored in {\var{netseekerth}}
in case it is needed later.

\section{Other Exported Network Functions and Data}

There are a few important functions called by other modules in order to
communicate with peers.

\begin{itemize}
\item {\func{peeraddr}}: return the address of an optional given {\type{connstate}} as a string.
\item {\func{tryconnectpeer}}: given an address of a peer, try to connect to the peer, adding it to {\var{netconns}} if successful.
Upon success, return the triple added to {\var{netconns}}.
Upon failure, return {\val{None}}.
\item {\func{addknownpeer}}: update the hash table {\var{knownpeers}} to either add the new peer or to update the latest connection time of an already known peer.
\item {\func{removeknownpeer}}: remove a given peer from the {\var{knownpeers}} hash table.
\item {\func{getknownpeers}}: iterated over the {\var{knownpeers}} hash table and collect the first 1000 peers which have been active in the past week.
\item {\func{loadknownpeers}}: load the {\file{peers}} file (if it exists) and add the peers which have been active in the past week to the {\var{knownpeers}} hash table.
\item {\func{saveknownpeers}}: create a new {\file{peers}} file consisting of the contents of the {\var{knownpeers}} hash table (as a text file with the peer's address followed by the last time the connection was active).
\item {\exc{BannedPeer}} is an exception which is raised when trying to connect to a banned peer.
\item {\var{bannedpeers}} is a hash table remembering the addresses of banned peers to prevent reconnection.
\item {\func{banpeer}} adds an address to {\var{bannedpeers}}.
\item {\func{clearbanned}} removes all addresses from {\var{bannedpeers}}.
\item {\func{network\_time}}: returns the current local time modified by the median skew time of all connected nodes.
\item {\func{queue\_msg}}: puts a non-reply message onto the {\field{sendqueue}} of a connection.
\item {\func{queue\_reply}}: puts a reply message onto the {\field{sendqueue}} of a connection.
\item {\func{find\_and\_send\_requestdata}}: searches through the current peers to find
one that has some desired data in its inventory (but has not already had the data requested).
If one is found, the data is requested.
Otherwise, {\exc{Not\_found}} is raised.
\item {\func{broadcast\_requestdata}}: request some data from all peers that have
the data in the inventory and from whom the data has not already been requested.
\item {\func{broadcast\_inv}}: send an {\constr{Inv}} (inventory) message to all peers.
\end{itemize}

