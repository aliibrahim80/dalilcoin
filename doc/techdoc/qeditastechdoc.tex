\documentclass{book}

\usepackage{makeidx}
\usepackage{amsmath}
\usepackage{amssymb}
\usepackage{mathpartir}

\def\tforall{\hat{\forall}}
\def\cA{\mathcal{A}}
\def\cC{\mathcal{C}}
\def\cD{\mathcal{D}}
\def\cE{\mathcal{E}}
\def\cF{\mathcal{F}}
\def\cG{\mathcal{G}}
\def\cH{\mathcal{H}}
\def\cK{\mathcal{K}}
\def\cO{\mathcal{O}}
\def\cP{\mathcal{P}}
\def\cR{\mathcal{R}}
\def\cT{\mathcal{T}}
\def\stp{~\mbox{stp}}
\def\ptp{~\mbox{ptp}}
\def\prop{~\mbox{prop}}
\def\pprop{~\mbox{polyprop}}
\newcommand\benorm[1]{[#1]^{\downarrow}}
\newcommand\dnorm[2]{[#2]^{#1}}
\newcommand\bednorm[2]{[#2]^{#1\downarrow}}
\newcommand\tmh[1]{\mathbf{\sharp}_{#1}}
\newcommand\tmhr[1]{\mathbf{TR}_{\sharp}(#1)}
\newcommand\gpa[1]{\mathbf{\hat\sharp}_{#1}}
\newcommand\hyp[1]{\mathbf{H}_{#1}}
\newcommand\known[1]{\Box_{#1}}
\newcommand\defin[1]{{\emph{#1}}\index{#1}}
\newcommand\branch[1]{{\texttt{#1}}\index{#1@\texttt{#1}}\index{branch!#1@\texttt{#1}}}
\newcommand\file[1]{{\texttt{#1}}\index{#1@\texttt{#1}}\index{file!#1@\texttt{#1}}}
\newcommand\dir[1]{{\texttt{#1}}\index{#1@\texttt{#1}}\index{directory!#1@\texttt{#1}}}
\newcommand\module[1]{{\textsf{#1}}\index{#1@\textsf{#1}}\index{module!#1@\textsf{#1}}}
\newcommand\coqmod[1]{{\textsf{#1}}\index{#1@\textsf{#1}}\index{Coq module!#1@\textsf{#1}}}
\newcommand\exec[1]{{\texttt{#1}}\index{#1@\texttt{#1}}\index{executable!#1@\texttt{#1}}}
\newcommand\commlinearg[1]{{\texttt{#1}}\index{#1@\texttt{#1}}\index{command line argument!#1@\texttt{#1}}}
\newcommand\var[1]{{\textsf{#1}}\index{#1@\textsf{#1}}\index{variable!#1@\textsf{#1}}}
\newcommand\val[1]{{\textsf{#1}}\index{#1@\textsf{#1}}\index{value!#1@\textsf{#1}}}
\newcommand\constr[1]{{\textsf{#1}}\index{#1@\textsf{#1}}\index{constructor!#1@\textsf{#1}}}
\newcommand\field[1]{{\textsf{#1}}\index{#1@\textsf{#1}}\index{field!#1@\textsf{#1}}}
\newcommand\exc[1]{{\textsf{#1}}\index{#1@\textsf{#1}}\index{exception!#1@\textsf{#1}}}
\newcommand\func[1]{{\textsf{#1}}\index{#1@\textsf{#1}}\index{function!#1@\textsf{#1}}}
\newcommand\serfunc[1]{{\textsf{#1}}\index{#1@\textsf{#1}}\index{function!serialization!#1@\textsf{#1}}\index{serialization function!#1@\textsf{#1}}}
\newcommand\type[1]{{\textsf{#1}}\index{#1@\textsf{#1}}\index{type!#1@\textsf{#1}}}
\newcommand\coqtype[1]{{\textsf{#1}}\index{#1@\textsf{#1}}\index{Coq type!#1@\textsf{#1}}}
\newcommand\coqfunc[1]{{\textsf{#1}}\index{#1@\textsf{#1}}\index{Coq function!#1@\textsf{#1}}}
\newcommand\dbtpsh[3]{{#3}\Uparrow_{#1}^{#2}}
\newcommand\dbsh[3]{{#3}\uparrow_{#1}^{#2}}
\newcommand\dbsub[3]{{#3}[#1 := #2]}

\def\dev{\branch{dev}}
\def\master{\branch{master}}
\def\testing{\branch{testing}}
\def\initdistr{\branch{initdistr}}

\title{Qeditas Technical Documentation}
\author{The Qeditas Developers}

\makeindex

\begin{document}
\maketitle

\tableofcontents

\chapter{Introduction}

This document is intended as a reference
for those wanting to understand or modify the
code supporting Qeditas.


\chapter{Configuration Related Code}\label{chap:config}

The modules \module{config} and \module{setconfig}
are for customizing the configuration of Qeditas.
The \file{configure} script creates an OCaml file \file{config.ml}
setting default values for the variables exposed in the interface \file{config.mli}:
\begin{itemize}
\item \var{datadir} : the location of the main directory containing the local Qeditas configuration file, wallet file, and other data (usually \dir{.qeditas} in the user's home directory)
\item \var{testnet} : set to true if Qeditas is running on the testnet instead of the mainnet
\item \var{staking} : set to true if Qeditas should stake
\item \var{ip} : optionally the IP address to listen for incoming connections
\item \var{ipv6} : optionally the IPv6 address to listen for incoming connections
\item \var{port} : the port to listen for incoming connections
\item \var{socks} : None if connections are not routed through SOCKS; Some($v$) if connections are routed through SOCKS protocol $v$ where $v$ is 4 or 5\footnote{At the moment, 5 is not yet supported.}
\item \var{maxconns} : the maximum number of connections
\item \var{seed} : the initial seed which is used to initialized the current stake modifier and future stake modifier.
\item \var{lastcheckpoint} : the last checkpoint (currently unused)
\item \var{randomseed} : an optional string used to seed the OCaml Random module. If {\var{randomseed}} is given it should be cryptographically strong and new each time Qeditas is started. If {\var{randomseed}} is not given, then Random is initialized using data from {\file{/dev/random}}. If {\file{/dev/random}} does not exist (e.g., under MS Windows), {\var{randomseed}} must be given and it is the user's responsibility to ensure {\var{randomseed}} is cryptographically strong and fresh.
\item \var{checkpointskey} : the private key for signing checkpoints (in Qeditas testnet WIF format). Signed checkpoints are only intended for the testnet, and only until the testnet is sufficiently stable. The corresponding public key is $(x,y)$ where
$$
\begin{array}{r}
x = 6371720373269100296662749352347839551092563796413818519910\\ 1530429614494608215 \\
y = 1455153899310935243255864964656407400101005941691152503242\\ 8212764782058901234
\end{array}
$$
These values can be found in {\file{blocktree.ml}} as the
settings for {\var{checkpointspubkeyx}}
and {\var{checkpointspubkeyy}}.
If a future version of Qeditas should use a different signing key for checkpoints, simply update {\var{checkpointspubkeyx}}
and {\var{checkpointspubkeyy}} in {\file{blocktree.ml}} to the new public key values.
\end{itemize}

The functions exposed in the interface \file{setconfig.mli}
override the default compiled settings by reading a configuration file
and checking the command line arguments of \exec{qeditasd} or \exec{qeditascli}.
This is done by calling \func{datadir\_from\_command\_line}
to set \var{datadir} from the command line if the argument \commlinearg{-datadir} was given,
then calling \func{process\_config\_file} to read the \file{qeditas.conf} file in \var{datadir},
and finally calling \func{process\_config\_args} to set the remaining configuration variables
by processing other command line arguments than \commlinearg{-datadir}.



\chapter{Serialization}\label{chap:ser}

The module \module{ser} (\file{ser.ml} and \file{ser.mli}) contains the basic code for serializing data.
Throughout the code there are functions with names of the form {\sf{seo\_}}$\tau$ (output)
and {\sf{sei\_}}$\tau$ (input).
In each case {\sf{seo\_}}$\tau$ is a function for creating a serialized output for an
element of type $\tau$
and {\sf{sei\_}}$\tau$ is a function for creating an element of type $\tau$ given its serialization.

There are two representations for the serializations: strings and channels
and the input and output functions are polymorphic so they can support both representations.
The types \type{seosbt} and \type{seist} correspond to the string representation
and there are corresponding atomic
functions \func{seosb} (output bits to a string buffer), \func{seosbf} (flush output to string buffer) and \func{seis} (input bits from a string).
The types \type{seosct} and \type{seict} correspond to the channel representation
and there are corresponding atomic
functions \func{seoc} (output bits to a channel), \func{seocf} (flush output channel) and \func{seic} (input bits from a channel).

The functions to output its take three arguments: the number of bits $n$, the bits to output as an integer $m$ with $0\leq m < 2^n$ and the serialization output object.
The functions to flush the output takes the serialization output object
and ensures any remaining its are output, assuming the remaining its are zero.
The functions to input its from a channel take two arguments: the number of bits $n$ to input and the serialization input object. It returns both an integer $m$ where $0\leq m < 2^n$ and the serialization input object.

The remainder of the functions defined in \module{ser} are for
serialization basic data: booleans, bytes, 32-bit integers, 64-bit integers, big integers (the OCaml type \type{big\_int}) assumed to be 256-bit integers and strings.
There are also other serialization functions for integers: \serfunc{seo\_varint} and \serfunc{sei\_varint} uses the varint representation used in Bitcoin, while \serfunc{seo\_varintb} and \serfunc{sei\_varintb} uses a different compact representation to represent numbers less than $65,536$.

In addition there are functions used to construct serialization functions
for list types, option types and product types with up to $6$ components.
For example, there are functions \serfunc{seo\_list} and \serfunc{sei\_list}.
When serialization functions are needed for lists of bytes, we simply
use \serfunc{seo\_list} applied to \serfunc{seo\_int8}
and \serfunc{sei\_list} applied to \serfunc{sei\_int8}.

Note that the serialization code is inherently higher-order (functions are first-class values).
Firstly, the atomic functions are passed as arguments to the serialization functions
for each type so the same serialization code can be used for both representations.
Secondly, the serialization functions themselves are passed to functions like \serfunc{seo\_list}
and \serfunc{sei\_list}.

There is one minor issue with the serialization code which may be confusing
and hopefully will not be a nightmare to maintain.
The bits are used to construct a byte from least significant to most significant.
As a consequence, different ways to output the same sequence of bits can be confusing.
Let $o$ be an atomic output function (either $\func{seosb}$ or $\func{seoc}$).
For example, suppose we wish to output the bit $0$ followed by the bit $1$.
We can do this in one of two ways:
\begin{itemize}
\item Call $o$ twice: first as $o\,1\,0$ (to output the one bit $0$) and then as $o\,1\,1$ (to output the one bit $1$).
\item Call $o$ once as $o\,2\,2$ (to output the two bits $10$, the binary representation of $2$).
\end{itemize}
In some places this can make serialization code difficult to correctly interpret.

{\bf{Note:}} The unit tests for the {\module{ser}} module are in {\file{basicunittests.ml}}
in the {\dir{src/unittests}}
directory in the {\branch{testing}} branch.
These unit tests give a number of examples demonstrating how the functions described above should behave.


\chapter{Cryptographic Hashing}\label{chap:hash}

The modules {\module{hashaux}}, {\module{sha256}}, {\module{ripemd160}}, {\module{hash}}
and {\module{htree}} contain code for cryptographic hashing functions.
The two hashing functions supported are {\tt{SHA256}}~\cite{sha256} and {\tt{RIPEMD-160}}~\cite{ripemd160}.
The {\tt{RIPEMD-160}} code only supports hashing a 256 bit input and is
assumed to be called on the result of applying {\tt{SHA256}}.

Profiling suggests that the hashing functions are the biggest computational bottleneck
in Qeditas. Improvements to the code described here could make Qeditas run significantly faster.

{\bf{Note:}} The unit tests for these modules are in {\file{basicunittests.ml}}
in the {\dir{src/unittests}}
directory in the {\branch{testing}} branch.
These unit tests give a number of examples demonstrating how the functions described below should behave.

\section{Auxiliary Functions}

The module {\module{hashaux}} implements a few helper functions needed by both hashing functions.
\begin{itemize}
\item {\func{hexsubstring\_int32}} takes a string of hexadecimal digits and a position. The 8 characters starting at the position are interpreted as a 32-bit integer (big endian).
\item {\func{int32\_hexstring}} takes a string buffer and a 32-bit integer and adds 8 hexadecimal digits to the buffer representing the integer (big endian).
\item {\func{big\_int\_sub\_int32}} takes a big integer $x$ and an integer $i$ and returns the 32-bit integer resulting from shifting away $i$ bits of $x$ (i.e., dividing by $2^i$) and then taking the 32 least significant bits (i.e., modulo $2^{32}$).
\item {\func{int32\_big\_int\_bits}} takes a 32-bit integer $x$ and an integer $i$ and returns the big integer resulting from shifting $x$ forward by $i$ bits (i.e., multiplying by $2^i$).
\item {\func{int32\_rev}} takes a 32-bit integer of the form $b_3 2^{24} + b_2 2^{16} + b_1 2^{8} + b_0$
and returns the reversed 32-bit integer $b_0 2^{24} + b_1 2^{16} + b_2 2^{8} + b_3$.
\end{itemize}

\section{Sha256}

The module {\module{sha256}} defines a type {\type{md256}} (message digest of 256 bits)
as a product of 8 32-bit integers.
(The type {\type{md256}} is also sometimes used to represent other 256-bit numbers, such as the
$x$ or $y$ component of a public key.)
There is also an array {\var{currblock}} of 16 32-bit integers.
Various other arrays are used internally and not exposed by the interface.

The following functions are defined:

\begin{itemize}
\item {\func{sha256init}} initializes the state to begin performing the hashing operation.
\item {\func{sha256round}} performs one round of the hashing operation.
\item {\func{getcurrmd256}} returns the current {\type{md256}} (extracted from the internal array {\var{currhashval}}).
\item {\func{sha256str}} returns the result of hashing a given string with {\tt{SHA256}}.
\item {\func{sha256str}} returns the result of double hashing a given string with {\tt{SHA256}}.
\item {\func{md256\_hexstring}} converts a 256-bit message digest to the corresponding hexadecimal string.
\item {\func{hexstring\_md256}} converts a hexadecimal string to the 256-bit message digest to the corresponding hexadecimal string.
\item {\func{md256\_big\_int}} converts a 256-bit message digest to the corresponding big integer.
\item {\func{big\_int\_md256}} converts a big integer (assuming it is less than $2^{256}$) to the 256-bit message digest to the corresponding hexadecimal string.
\item {\serfunc{seo\_md256}} serializes a 256-bit message digest.
\item {\serfunc{sei\_md256}} deserializes a 256-bit message digest.
\end{itemize}

\section{Ripemd160}

The module {\module{ripemd160}} implements {\tt{RIPEMD-160}} restrict to 256-bit message digests as inputs.
The module defines a type {\type{md}} (message digest of 160 bits)
as a product of 5 32-bit integers.

The following functions are defined:

\begin{itemize}
\item {\func{ripemd160\_md256}} returns the result of hashing a given 256-bit message digest with {\tt{RIPEMD-160}}.
\item {\func{md\_hexstring}} converts a 160-bit message digest to the corresponding hexadecimal string.
\item {\func{hexstring\_md}} converts a hexadecimal string to the 160-bit message digest to the corresponding hexadecimal string.
\end{itemize}

\section{Hash}

The module {\module{hash}} is important. It defines a type {\type{hashval}} as 5 32-bit integers (representing a 160-bit hash value).

A function {\func{hash160}} takes an arbitrary string to the result of hashing first by {\tt{SHA256}} and then by {\tt{RIPEMD-160}}.
The type {\type{hashval}} is implemented the same way as the type {\type{md}} in the module {\module{ripemd160}}. If they were defined differently, the function {\func{hash160}} would be ill-typed.

{\bf{Note:}} The Coq formalization contains Coq module a {\coqmod{CryptoHashes}} which corresponds
to some of what is in the {\module{hash}} module.
In particular, a type of {\coqtype{hashval}} is defined along with functions to hash natural numbers,
addresses (which are defined to be 160 bit sequences in the Coq module {\coqmod{Addr}})
and pairs of hash values. These functions are injective and give disjoint hash values.
From these, a number of other hashing functions are defined in ways that continue
to ensure injectivity and disjointness. The Coq representation is idealized.
Hash values is infinite and the hashing functions are not cryptographic hashing functions.
For more information, see~\cite{White2015a}.

There are a variety of functions for creating, using and combining hash values.
The following functions 
\begin{itemize}
\item {\func{hashval\_bitseq}} converts a hash value to a list of 160 booleans.
\item {\func{hashval\_hexstring}} converts a hash value to a string of 40 hexadecimal digits.
\item {\func{hexstring\_hashval}} converts a string of 40 hexadecimal digits to a hash value.
\item {\func{printhashval}} prints a hash value as 40 hexadecimal digits.
\item {\func{hashval\_rev}} performs a bytewise reversal of the hash value.\footnote{This seems to be unused.}
\item {\func{hashval\_big\_int}} converts a hash value to a big integer.
\item {\func{big\_int\_hashval}} converts a big integer to a hash value.
\item {\serfunc{seo\_hashval}} serializes hash values.
\item {\serfunc{sei\_hashval}} deserializes hash values.
\end{itemize}
The following functions create (effectively) unique hash values from given input.
Internally in each case the value being hashed is prefixed with a distinct 32-bit
integer so that the hash value given by different functions will be unique.
For example, {\func{hashint32}} prefixes the 32-bit integer with the 32-bit integer $1$
while {\func{hashint64}} prefixes the 64-bit integer with the 32-bit integer $2$.
\begin{itemize}
\item {\func{hashint32}} hashes a 32-bit integer.
\item {\func{hashint64}} hashes a 64-bit integer.
\item {\func{hashpair}} hashes a pair of hashes.
\item {\func{hashpubkey}} hashes a public key, given as two {\type{md256}} values.
\item {\func{hashpubkeyc}} hashes a compressed public key, given by a boolean (indicating if $y$ is even or odd) and one {\type{md256}} values (giving $x$).
\item {\func{hashtag}} combines a hash value with a 32-bit integer to create a different hash value. This
is used when we wish to ensure later data structures create unique hash values.
\item {\func{hashlist}} hashes a list of hash values. This could be implemented by a simple recursion using {\func{hashpair}}, but this would be inefficient. Instead the list is iterated over with {\func{sha256round}} being called when appropriate.
\item {\func{hashfold}} is given a function $f$ which returns a hashval for a given input and a list of appropriate inputs for $f$ and iteratively calls $f$ on the components of the list while performing {\func{sha256round}} to compute a hash value for the list of hash values computed by $f$ over the list.
\item {\func{hashbitseq}} takes a list of booleans and creates a hash values.
The naive way of doing this using {\func{hashlist}} would be too inefficient.
Instead the booleans are treated as 32-bit integers by considering them in groups of 32.
\end{itemize}
Sometimes optional hash values are used. This is important, for example, when we want to have an ``empty'' hash value $\bot$ corresponding to the hash of some ``empty'' data.
\begin{itemize}
\item {\func{ohashlist}} takes a list of hash values and computes an optional hash value.
The optional hash value is $\bot$ if and only if the input is the empty list.
\item {\func{hashopair}} takes two optional hash values and returns an optional hash value.
The output is $\bot$ if and only if both inputs were $\bot$.
\item {\func{hashopair1}} takes a hash value $x$ and an optional hash value $y$
and returns a hash value (known to not be $\bot$).
{\func{hashopair1}} is essentially the special case of {\func{hashopair}} where the first value is known not to be $\bot$.
\item {\func{hashopair2}} takes an optional hash value $x$ and a hash value $y$
and returns a hash value (known to not be $\bot$).
{\func{hashopair2}} is essentially the special case of {\func{hashopair}} where the second value is known not to be $\bot$.
\end{itemize}

In addition, various types of addresses are defined.
Fundamentally there are four kinds of addresses:
{\type{p2pkhaddr}} (pay to public key hash addresses, a.k.a., {\defin{p2pkh addresses}}),
{\type{p2shaddr}} (pay to script hash addresses, a.k.a., {\defin{p2sh addresses}}),
{\type{termaddr}} ({\defin{term addresses}}) and {\type{pubaddr}} ({\defin{publication addresses}}).
Each of these types is defined the same way as hash values (as 5 32-bit integers)
and so an object of one of these types can be used as an object of another.
\begin{itemize}
\item {\type{p2pkhaddr}} A pay to public key hash addresses is the hash value
obtained by hashing a public key. The intention is that the holder of the corresponding private key
can sign transactions related to the address.
The code for checking such signatures is in the module {\module{signat}}.
\item {\type{p2shaddr}} A pay to script hash address is the hash value
obtained by hashing a script.\footnote{Qeditas uses essentially the same scripting language as Bitcoin as of early 2015. Two Bitcoin operations are not supported: OP\_SHA1 and OP\_RIPEMD160.}
Such a script can act as a generalized signature in the following sense:
the script is executed and if the result is $1$ then the generalized signature is accepted.
This is a ``generalized signature'' since some of the script operations check a signature.
The code for executing scripts and checking generalized signatures is in the module {\module{script}}.
\item {\type{termaddr}}
Term addresses are hash values obtained in one of three ways:
\begin{enumerate}
\item A term address may be the hash root of a closed simply typed term $t$.
This is the global (theory independent) term address of $t$.
\item Given a theory $T$ and a closed term $t$ which has type $\tau$ in the theory $T$,
the combined hash of $T$, hash root of $t$ and the hash of $\tau$
gives a term address.\footnote{The combined hash is again hashed with a tag with $32$ to avoid the possibility that the combined hash value is the same as a different kind of term address.}
This is the address of the term $t$ in the theory $T$.
\item Given a theory $T$ and a closed proposition $t$,
the combined hash of $T$ and hashroot of $t$
gives a term address.\footnote{The combined hash is again hashed with a tag with $33$ to avoid the possibility that the combined hash value is the same as a different kind of term address.}
\end{enumerate}
Ownership information about a term or proposition (either globally or as part of a theory)
is stored at corresponding term address.
The author of the first document published which defines a term or proves a proposition can and must
also supply ownership information. This ownership information determines the conditions under
which the term or proposition can be imported into future documents.
Term addresses corresponding to terms or propositions within a theory are also used
to ensure terms have the correct type (without needing to repeat the definition) in the theory
and to ensure propositions are already known (without needing to repeat a proof).
\item {\type{pubaddr}}
A publication address corresponds to the hash root of a published document, theory specification or signature specification.
\end{itemize}

In addition to the four basic kinds of addresses, there are two other types of addresses:
\begin{itemize}
\item {\type{payaddr}}
The type {\type{payaddr}} (of {\defin{pay addresses}}) is the disjoint sum of the types {\type{p2pkhaddr}} and {\type{p2shaddr}}.
This is implemented by taking a boolean along with 8 32-bit integers.
The 8 32-bit integers is a hash value representing either a {\type{p2pkhaddr}}
or a {\type{p2pkhaddr}}.
The boolean is {\val{false}} if the hash value represents a {\type{p2pkhaddr}},
and is {\val{true}} if the hash value represents a {\type{p2shaddr}}.
\item {\type{addr}}
The type {\type{addr}} (of addresses) is the disjoint sum of the four types
{\type{p2pkhaddr}}, {\type{p2shaddr}},
{\type{termaddr}} and {\type{pubaddr}}.
This is implemented by taking an integer $i\in\{0,1,2,3\}$ along with 8 32-bit integers.
The 8 32-bit integers is a hash value representing either a {\type{p2pkhaddr}}, a {\type{p2shaddr}},
a {\type{termaddr}} or a {\type{pubaddr}}.
If $i=0$, the hash value represents a {\type{p2pkhaddr}}.
If $i=1$, the hash value represents a {\type{p2shaddr}}.
If $i=2$, the hash value represents a {\type{termaddr}}.
If $i=3$, the hash value represents a {\type{pubaddr}}.
\end{itemize}

The following functions operate on addresses.
\begin{itemize}
\item {\func{hashval\_p2pkh\_payaddr}} gives the pay address corresponding to a hash value interpreted as a pay to public key hash addresses.
\item {\func{hashval\_p2sh\_payaddr}} gives the pay address corresponding to a hash value interpreted as a pay to script hash address.
\item {\func{hashval\_p2pkh\_addr}} gives the address corresponding to a hash value interpreted as a pay to public key hash addresses.
\item {\func{hashval\_p2sh\_addr}} gives the address corresponding to a hash value interpreted as a pay to script hash address.
\item {\func{hashval\_term\_addr}} gives the address corresponding to a hash value interpreted as a term address.
\item {\func{hashval\_pub\_addr}} gives the address corresponding to a hash value interpreted as a publication address.
\item {\func{addr\_bitseq}} returns a list of 162 booleans corresponding to an address, where the first two booleans determine which kind of address it is and the remaining 160 are the hash value.
\item {\func{bitseq\_addr}} returns an address given a list of 162 booleans.
\item {\func{p2pkhaddr\_payaddr}} converts a pay to public key hash address (a hash value) to a pay address by indicating it is a pay to public key hash address.
\item {\func{p2shaddr\_payaddr}} converts a pay to public key hash address (a hash value) to a pay address by indicating it is a pay to script hash address.
\item {\func{p2pkhaddr\_addr}} converts a pay to public key hash address (a hash value) to an address by indicating it is a pay to public key hash address.
\item {\func{p2shaddr\_addr}} converts a pay to public key hash address (a hash value) to an address by indicating it is a pay to script hash address.
\item {\func{payaddr\_addr}} converts a pay address to an address. In practice this simply means converting the first component from a boolean to an integer ({\val{false}} to $0$ and {\val{true}} to $1$).
\item {\func{termaddr\_addr}} converts a term address (a hash value) to an address by indicating it is a term address.
\item {\func{pubaddr\_addr}} converts a publication address (a hash value) to an address by indicating it is a publication address.
\item {\func{payaddr\_p}} checks if an address is a pay address.
\item {\func{p2pkhaddr\_p}} checks if an address is a pay to public key hash address.
\item {\func{p2shaddr\_p}} checks if an address is a pay to script hash address.
\item {\func{termaddr\_p}} checks if an address is a term address.
\item {\func{pubaddr\_p}} checks if an address is a publication address.
\item {\func{hashaddr}} hashes the address creating a unique hash value. (This is different from the underlying hash value of the address since the prefix is included before rehashing.)
\item {\func{hashpayaddr}} performs the same operation of {\func{hashaddr}} but only on pay addresses.
\item {\func{hashtermaddr}} performs the same operation of {\func{hashaddr}} on term addresses.
\item {\func{hashpubaddr}} performs the same operation of {\func{hashaddr}} on publication addresses.
\item {\serfunc{seo\_addr}} serializes an address.
\item {\serfunc{sei\_addr}} deserializes an address.
\item {\serfunc{seo\_payaddr}} serializes a pay address.
\item {\serfunc{sei\_payaddr}} deserializes a pay address.
\item {\serfunc{seo\_termaddr}} serializes a term address.
\item {\serfunc{sei\_termaddr}} deserializes a term address.
\item {\serfunc{seo\_pubaddr}} serializes a publication address.
\item {\serfunc{sei\_pubaddr}} deserializes a publication address.
\end{itemize}

\section{Hash Trees}

The module {\module{htree}} defines a polymorphic type {\type{htree}}
which stores data in a tree indexed by a list of booleans.
In practice the list of booleans comes from a hash value.
The following functions are exported:
\begin{itemize}
\item {\func{htree\_lookup}} is given a boolean list and an {\type{htree}}
and returns the data if found and {\val{None}} if it is not found.
\item {\func{htree\_create}} is given a boolean list $\overline{b}$ and data $x$
and returns a new {\type{htree}} with only this single entry ($x$ at position $\overline{b}$).
\item {\func{htree\_insert}} is given an {\type{htree}}, a boolean list and data $x$
and returns the result of inserting the data into the {\type{htree}} ($x$ at position $\overline{b}$).
This will shadow data already at position $\overline{b}$ if there were any. (In practice this should never
happen since $\overline{b}$ should be obtained from a hash value determined by $x$.)
\item {\func{ohtree\_hashroot}} computes an optional hash value
given a function $f$ (which computes optional hash values for members of the underlying type),
the current depth $n$
and an optional {\type{htree}}.
This is essentially the Merkle root of the {\type{htree}}.
\end{itemize}

In practice {\type{htree}} is used in two ways: one is to story theories
and the other is to store theory-specific signatures.\footnote{Here {\em{signature}} is used
to mean a collection of constants, definitions and known propositions and should not
be confused with cryptographic signatures.}
These specific uses will be discussed in Chapter~\ref{chap:math}
where the module {\module{mathdata}} is considered.


\chapter{Cryptocurrency Operations}\label{chap:cryptocurr}

The modules \module{secp256k1}, \module{cryptocurr}, \module{signat} and \module{script}
contain code for cryptocurrency related operations.
In particular, \module{secp256k1} implements the elliptic curve {\tt{secp256k1}}~\cite{sec2final},
\module{cryptocurr} supports base 58 formats for private keys and addresses,
\module{signat} supports cryptographic signatures
and \module{script} supports the Bitcoin scripting language (mostly).

{\bf{Note:}} The unit tests for these modules are in {\file{basicunittests.ml}}
in the {\dir{src/unittests}}
directory in the {\branch{testing}} branch.
These unit tests give a number of examples demonstrating how the functions described below should behave.

\section{Elliptic Curve Code}

The module \module{secp256k1} contains the operations for the corresponding elliptic curve~\cite{sec2final}.
Most of the code in this module was taken from the code for Egal~\cite{Brown2014}.
256-bit integers are represented using big integers (\type{big\_int}) from OCaml's {\tt{nums}} library.
A function {\func{evenp}} is defined and exposed since it is used elsewhere.

The 256-bit prime $p$ used in the elliptic curve is
exposed as the big integer {\val{\_p}}.
The following functions implement operations modulo $p$:
\begin{itemize}
\item {\func{add}} implements addition modulo $p$.
\item {\func{mul}} implements multiplication modulo $p$.
\item {\func{pow}} implements $x^i$ modulo $p$ where $x$ is a big integer and $i$ is an integer.
\item {\func{eea}} implements the Extended Euclidean Algorithm which is used to compute
multiplicative inverses modulo $p$.
\end{itemize}

Points on the elliptic curve are represented by element of type {\type{pt}}
which is defined to be an optional pair $(x,y)$ of big integers.
{\val{None}} represents the zero point (point at infinity).
The following functions are defined and exposed:
\begin{itemize}
\item {\func{addp}} implements addition of two points.
\item {\func{smulp}} implements scalar multiplication of a big integer to a point.
\end{itemize}

The base point $g$ on the curve (which generates the group) is exposed as {\val{\_g}}.
The order $n$ of $g$ is exposed as the big integer {\val{\_n}}.
The function {\func{curve\_y}} takes a boolean $e$ and a big integer $x$
and returns the big integer $y$ such that $(x,y)$ is a point on the curve
where $y$ is even if $e$ is true
and $y$ is odd if $e$ is false.

As usual, there are serialization functions {\serfunc{seo\_pt}} and {\serfunc{sei\_pt}} for
points. The serialization functions assume the components $x$ and $y$ of the point
are positive integers less than $2^{256}$.

\section{Cryptocurrency Operations}

The module {\module{cryptocurr}} implements
functions which convert private keys and addresses 
to and from base 58 representations.
\begin{itemize}
\item {\func{base58}} converts a big integer into a base 58 string.
\item {\func{frombase58}} converts a base 58 string to a big integer.
\item {\func{qedwif}} converts a big integer (private key) and a boolean (indicating if it is for a compressed address) to a base 58 string.
The Qeditas WIF format uses a two byte prefix of $5,8$ for compressed addresses
and $2,30$ for uncompressed addresses.
The result is that compressed WIFs start with the character {\tt{k}} and uncompressed WIFs start
with the character {\tt{K}}.
\item {\func{privkey\_from\_wif}} takes a Qeditas WIF string and returns the corresponding
private key (as a big integer) along with a boolean indicating if it is for the compressed address.
\item {\func{privkey\_from\_btcwif}} takes a Bitcoin WIF string and returns the corresponding
private key (as a big integer) along with a boolean indicating if it is for the compressed address.
This function is included to make it easy for people to import Bitcoin private keys corresponding to the
initial distribution.
\item {\func{pubkey\_hashval}} takes a non-zero public key (a pair $(x,y)$) and a boolean (indicating if the compressed address should be used)
and returns the $20$ bytes which result from hashing
either the compressed public key ($2$ with $x$ if $y$ is even; $3$ with $x$ if $y$ is odd)
or the uncompressed public key ($4$ with $x$ and $y$).
This hash value is also the corresponding pay to public key hash address.
\item {\func{hashval\_from\_addrstr}} takes a string with a Qeditas address and returns the underlying hash value.
\item {\func{hashval\_btcaddrstr}} takes a hash value and returns the corresponding Bitcoin address.
\item {\func{addr\_qedaddrstr}} takes a Qeditas address and returns a base 58 string representation of the address.
The prefix byte used is different for the four different kinds of addresses:
\begin{itemize}
\item Pay to public key hash addresses use a prefix byte of $58$
and so these Qeditas addresses begin with the character {\tt{Q}}.
\item Pay to script hash addresses use a prefix byte of $120$
and so these Qeditas addresses begin with the character {\tt{q}}.
\item Term addresses use a prefix byte of $66$
and so these Qeditas addresses begin with the character {\tt{T}}.
\item Publication addresses use a prefix byte of $56$
and so these Qeditas addresses begin with the character {\tt{P}}.
\end{itemize}
\item {\func{qedaddrstr\_addr}} takes a string with a base 58 Qeditas address and returns the Qeditas address.
\item {\func{btcaddrstr\_addr}} takes a string with a base 58 Bitcoin address (either p2pkh or p2sh) and returns the Qeditas address.
\end{itemize}
Some of the code in this module was taken from the code for Egal~\cite{Brown2014}.
Egal included {\tt{BIP 32}} code that isn't needed in Qeditas.
Egal relied on openssl to compute {\tt{SHA256}} and {\tt{ripemd160}} hashes, but Qeditas does this itself.

\section{Cryptographic Signature Checking}

The module {\module{signat}} implements functions for creating and verifying
cryptographic signatures over the elliptic curve.
A cryptographic signature (represented by the type {\type{signat}})
is a pair $(r,s)$ of big integers.
As usual, the functions ${\serfunc{seo\_signat}}$ and ${\serfunc{sei\_signat}}$
serialize and deserialize elements of type {\type{signat}}.
Let $n$ be the order of the group for {\tt{secp256k1}}.
\begin{itemize}
\item {\func{decode\_signature}} takes a base 64 string and returns $(i,c,(r,s))$
where $i\in\{0,1,2,3\}$ (``recid'') is a tag to help recover the public key from the signature
and what was signed,
$c$ (``fcomp'') is a boolean indicating if the signature is for a compressed public key
and $(r,s)$ is the cryptographic signature.
\item {\func{signat\_big\_int}} takes a big integer $e<n$ (in practice $e<2^{160}$),
a big integer private key $k<n$ and a random big integer $R<n$
and returns a signature $(r,s)$.
The signature $(r,s)$ signs $e$ with the private key $k$.
\item {\func{signat\_hashval}} is the same as {\func{signat\_big\_int}} except
it is given a hash value $h$ to sign instead of a big integer.
The implementation simply converts $h$ to a (160-bit) big integer using {\func{hashval\_big\_int}} and calls
{\func{signat\_big\_int}}.
The result is a signature $(r,s)$ signing $h$ with the given private key.
\item {\func{verify\_signed\_big\_int}} takes a big integer $e$, a point (public key) $(x,y)$
and a signature $(r,s)$
and returns a boolean indicating if $(r,s)$ is a valid
signature of $e$ by the private key corresponding to $(x,y)$.
\item {\func{verify\_p2pkhaddr\_signat}}
takes a big integer $e$, a {\type{p2pkhaddr}} $\alpha$ (equivalently, a hash value),
a signature $(r,s)$,
an integer $i\in\{0,1,2,3\}$
and a boolean $c$.
It uses $e$, $(r,s)$ and $i$ to recover a point on the curve using
{\func{recover\_key}}.
If {\func{recover\_key}} returns the zero point, then the signature is not valid and the boolean false is returned.
Otherwise, {\func{recover\_key}} returns a public key $(x,y)$.
The {\type{p2pkhaddr}} corresponding to $(x,y)$ (compressed if $c$ is true, uncompressed otherwise)
is computed using {\func{pubkey\_hashval}} and compared with $\alpha$.
If they are the same, then the signature is valid and the boolean true is returned.
Otherwise, the signature is not valid and the boolean false is returned.
\item {\func{verifybitcoinmessage}} is used to verify a bitcoin signed message returning a boolean (true if valid, false otherwise).
The inputs are a {\type{p2pkhaddr}} $\alpha$, $i\in\{0,1,2,3\}$, a boolean $c$, a cryptographic signature $(r,s)$
and a string $m$ (the message).
If Qeditas is running in the testnet, then the message is prefixed with {\tt{testnet:}}.
This allows people to sign, for example, endorsements which are valid on the testnet, but not on the mainnet.
The message is then modified the same way as the Bitcoin core client (essentially
including {\tt{Bitcoin Signed Message:}} and some characters indicating the length of this prefix and the length of the message).
The remainder of the work is performed by the internal {\func{verifymessage}} function:
The message is double {\tt{SHA256}} hashed and converted to a big integer $e$.
The public key is attempted to be recovered
using $e$, $(r,s)$ and $i$ using {\func{recover\_key}} (with false returned upon failure).
Assuming the public key $(x,y)$ is recovered, the final check verifies that the hash of the public key
(compressed if $c$ is true, uncompressed otherwise) is $\alpha$.
\item {\func{verifybitcoinmessage\_recover}} is used to verify a bitcoin signed message returning an optional public key $(x,y)$ (the corresponding public key if the signature is valid, the {\tt{None}} if not).
It behaves equivalently to {\func{verifybitcoinmessage}}
except upon success it returns the public key as ${\tt{Some}}(x,y)$
and returns {\tt{None}} upon failure.
In this case, internal {\func{verifymessage\_recover}} function is used.
\end{itemize}

There is also an internal function {\func{recover\_key}} which
computes a public key $(x,y)$
from a big integer $e$ (from the hash value of what was signed), a signature $(r,s)$ and a ``recid'' $i\in\{0,1,2,3\}$.
This should be the public key corresponding to the private key which was used to construct $(r,s)$
from $e$.

{\bf{Note:}} The Coq module {\coqmod{CryptoSignatures}} is intended to correspond to
the {\module{signat}} module. It defines a Coq type {\coqtype{signat}} and functions
to simulate signing with a private key and checking a signature.
The actual implementation is trivial, but only the required properties are exported.

\section{Scripts and Generalized Signatures}

The module {\module{script}} implements the Bitcoin scripting language
(with the exceptions of OP\_SHA1 and OP\_RIPEMD160).
The main reason the Bitcoin scripting language is included is so p2sh
addresses in the Bitcoin snapshot can be redeemed.
Scripts are represented by lists of integers which should be bytes.
The following functions operate on scripts:
\begin{itemize}
\item {\func{hash160\_bytelist}} takes a script and computes its hash by
taking the SHA256 and then RIPEMD160.
The hash value returned should be interpreted as a {\type{p2shaddr}}.
The procedure is the same way Bitcoin computes p2sh addresses.
\item {\func{eval\_script}}
evaluates a script in context.
The inputs are a big integer $e$ (which corresponds to what is meant to be ``signed''),
the script $\overline{s}$ and two stacks.
The function returns the two stacks which result from evaluating the script.
\item {\func{verify\_p2sh}}
compares a script's hash against the given {\type{p2shaddr}}
and verifies that the script evaluates to ``true.''
A boolean is returned.
The inputs are a big integer $e$ (which corresponds to what is meant to be ``signed''),
a {\type{p2shaddr}} $\beta$ and a script $\overline{s}$.
The only way {\val{true}} will be returned is if the following occurs:
\begin{enumerate}
\item The script is evaluated and the top of the main stack is a script $\overline{s_1}$ which hashes to give $\beta$.
\item The script $\overline{s_1}$ is popped off the main stack and evaluated leaving something nonzero at the top of the main stack.
\end{enumerate}
\end{itemize}
The process of evaluating the script is more complicated than it is in Bitcoin.
The reason is that {\tt{OP\_CHECKSIG}} and {\tt{OP\_CHECKMULTISIG}} may be signatures
by endorsement. An endorsement may be, for example, an endorsement of a p2sh address
to a p2pkh address. In order to check the endorsement, another script must be checked
to be a valid ``signature'' of a different value (the hash of the endorsement message).
For this reason, the main functions that do the work: {\func{eval\_script}},
{\func{eval\_script\_if}},
{\func{checksig}},
{\func{checkmultisig}} and 
{\func{check\_p2sh}}
are mutually recursive.

In order to account for endorsements in a uniform way,
the type {\type{gensignat}} of ``generalized signatures'' is defined.
There are six constructors of type {\type{gensignat}} corresponding to 
six ways of making a signature:
\begin{itemize}
\item {\val{P2pkhSignat}} is an ordinary cryptographic signature $(r,s)$ corresponding
to a given public key $(x,y)$ which may or may not be compressed. (Note that the public key is explicitly given here and need not be recovered during signature checking.)
The public key corresponds to a {\tt{p2pkhaddr}} (again, one compressed and one uncompressed).
\item {\val{P2shSignat}} is a script and should be checked by calling {\func{verify\_p2sh}} above.
The script corresponds to a certain {\tt{p2shaddr}}. (Note that the correspondence is not
direct. The script itself is not hashed to obtain the {\tt{p2shaddr}}.
Instead the script should evaluate to yield a script which hashes
to the {\tt{p2shaddr}} at the top of the main stack.)
\item {\val{EndP2pkhToP2pkhSignat}} This is a p2pkh signature via a p2pkh endorsement.
This means that two public keys (and two booleans indicating compressed or uncompressed) are given
along with two signatures.
One of the signatures is of a Bitcoin message with the appropriate base 58 Qeditas pay to public key hash address, signed with the private key for the other public key.
The other signature is the signature of what should be signed.
\item {\val{EndP2pkhToP2shSignat}}
This is a p2sh signature via an endorsement a p2pkh endorsement.
That is, a signature of a Bitcoin message with the appropriate base 58 Qeditas pay to script hash
address is given, corresponding to the public key of the p2pkh address.
Also, a script corresponding to the p2sh address is given which ``signs'' what should be signed.
\item {\val{EndP2shToP2pkhSignat}}
This is a p2pkh signature via an endorsement a p2sh endorsement.
Here a script is given which ``signs'' the Bitcoin message with an endorsement to a base 58 Qeditas pay to public key hash address.
An ordinary cryptographic signature (corresponding to the p2pkh address)
signing what is to be ``signed'' is given.
\item {\val{EndP2shToP2shSignat}}
This is a p2sh signature via an endorsement a p2sh endorsement.
Here a script is given which ``signs'' the Bitcoin message with an endorsement to a base 58 Qeditas pay to script hash address.
A separate script corresponding to the other p2sh address is given which ``signs'' what should be signed.
\end{itemize}
As usual, there are two functions {\serfunc{seo\_gensignat}} and
{\serfunc{sei\_gensignat}} for serializing generalized signatures.

There is one important function for working with generalized signatures:
\begin{itemize}
\item {\func{verify\_gensignat}} takes a big integer $e$ (corresponding to the hash value
of what should be signed), a generalized signature
and an address, and returns a boolean indicating if the generalized signature
verifies that the owner of the address ``signed'' $e$.
The address should be either a pay to public key hash address or pay to script hash address.
(If a term address or publication address is given, {\val{false}} is returned.)
Each of the six cases is considered separately in the code, but the idea is clear:
check that the signature corresponds to the given address and
check that the signature ``signs'' $e$.
\end{itemize}


\chapter{Database}\label{chap:db}

For several data types we will need to manipulate persistent storage of values 
indexed by a hash value. (We will call this a ``database'' although 
it is only a key-value mapping.)
One way to do this would be to use a standard library built for this purpose,
such as leveldb.
However, integrating leveldb with the OCaml code has proven challenging.
Instead (at least for the moment) the database has been implemented
by simply using files in directories.
The particular implementation has been abstracted using a module type
so that the implementation of the module can be easily replaced.\footnote{Trent Russell was responsible for the initial/current implementation of the database code.}

The module type {\moduletype{dbtype}} is actually a functor type.
It depends on a signature with
\begin{itemize}
\item a type {\tt{t}} (the type of the values to be stored),
\item a string {\tt{basedir}} (indicating the top level directory where these key-value pairs will be stored) and
\item functions {\tt{seival}} and {\tt{seoval}} for deserializating and serializing the data from and to channels.
\end{itemize}
An implementation of {\moduletype{dbtype}} must implement the following:
\begin{itemize}
\item {\func{dbget}} taking a hash value (as the key) to value of type {\tt{t}} (or raising {\exc{Not\_found}}).
\item {\func{dbexists}} takes a hash value (as the key) and returns {\val{true}} if there is an entry with this key and returns {\val{false}} otherwise. (One could use {\func{dbget}} for this purpose, but {\func{dbget}} must take the time to deserialize corresponding the value.)
\item {\func{dbput}} takes a hash value (as the key) and a value of type {\tt{t}} and stores the key-value pair.
\item {\func{dbdelete}} takes a hash value (as the key) and deletes the entry with this key, if one exists. If there is no entry, {\func{dbdelete}} does nothing.
\end{itemize}

There is a functor {\module{Dbbasic}} which returns
a module implementing {\moduletype{dbtype}}, given an implementation of {\tt{t}}, {\tt{basedir}}, {\tt{seival}} and {\tt{seoval}}.
The implementation of {\module{Dbbasic}}
uses subdirectories of {\tt{basedir}}
with three files: {\file{index}},
{\file{data}}
and {\file{deleted}}.
The file {\file{data}} contains serializations of the values stored in this directory
and the file {\file{index}} contains the keys (hash values) along with integers giving
the position of the corresponding data in {\file{data}}.
The file {\file{deleted}} is a list of heys (hash values)
that have been marked as deleted (but the keys are still in {\file{index}}
and the value is still in {\file{data}}).

The keys in {\file{index}} are ordered. The actual ordering is not important, as long
as it is a total ordering. Since hash values are internally implemented as
a 5-tuple signed 32-bit integers, the OCaml function {\func{compare}}
can be used. This means $h = (x_0,x_1,x_2,x_3,x_4)$ is less than $k=(y_0,y_1,y_2,y_3,y_4)$
if 
$x_0 < y_0$ (as a signed 32-bit integer)
or $x_0 = y_0$ and $x_1 < y_1$
or $x_0 = y_0$, $x_1 = y_1$ and $x_2 < y_2$
or $x_0 = y_0$, $x_1 = y_1$, $x_2 = y_2$ and $x_3 < y_3$
or $x_0 = y_0$, $x_1 = y_1$, $x_2 = y_2$, $x_3 = y_3$ and $x_4 < y_4$.

The maximum number of entries in the files in a directory is 65536,
but new entries are also not allowed after the {\file{data}} file
exceeds 100 MB.
After no more entries are permitted in a directory,
a subdirectory named using the next byte (in hex) of the key is
created (if necessary) and this subdirectory is used, unless it is also full.

Some auxiliary functions are used:
\begin{itemize}
\item {\func{find\_in\_deleted}} checks if a key is in the {\file{deleted}} file of a directory.
\item {\func{load\_deleted}} loads all the hash values (keys) in the {\file{deleted}} file of a directory.
\item {\func{undelete}} removes a key from the {\file{deleted}} file of a directory
by loading all the deleted keys and then recreating the {\file{deleted}} file without given the key.
\item {\func{count\_deleted}} gives the number of entries in the {\file{deleted}} file of a directory.
\item {\func{find\_in\_index}}
searches for a key
by loading the {\file{index}} file and doing a binary search.
If it is found, then the position of the value in the {\file{data}}
file is returned. Otherwise, {\exc{Not\_found}} is raised.
\item {\func{load\_index}} loads the index file as a list of pairs of hash values and positions.
This list is reverse sorted by the hash values.
\item {\func{count\_index}} gives the number of entries in the {\file{index}} file of a directory.
\item {\func{dbfind}} and {\func{dbfind\_a}}
are used to search for a subdirectory where the {\func{index}} file
contains a given key, returning the directory and value position
if one is found. If none is found, {\exc{Not\_found}}
is raised.
\item {\func{find\_next\_space}}, {\func{find\_next\_space\_a}} and {\func{find\_next\_space\_b}}
are used to find the next appropriate subdirectory and position where a key
can be included.
\item {\func{defrag}} cleans up by actually deleting key-value pairs which have been deleted.
\end{itemize}

The implementation of {\module{Dbbasic}} works as follows:
\begin{itemize}
\item There are two hash table {\val{cache1}} and {\val{cache2}}
which store (roughly) the last 128 to 256 entries looked up
so that these can be returned again quickly.
(When one cache has 128 entries, the other is cleared and begins
to be filled.)
The internal functions {\func{add\_to\_cache}} and {\func{del\_from\_cache}}
handle adding and removing key-value pairs from the cache.
\item {\func{dbget}} tries to find the key in the cache.
If it is not in the cache, {\func{dbfind}} is called to 
try to find a subdirectory and value position.
If one is found and the key has not been deleted, then
the value is deserialized from the {\file{data}} file starting at the position
and returned.
Otherwise, {\exc{Not\_found}} is raised.
\item {\func{dbexists}} is analogous to {\func{dbget}} except it
does not deserialize the value if it is found.
Instead it returns {\val{true}} if a subdirectory and value position were found
(and the key is not marked as deleted),
and returns {\val{false}} otherwise.
\item {\func{dbput}} takes a key $k$ and value $v$.
The function {\func{dbfind}} is called to find an entry for $k$ if one exists.
If one is found and it has been marked as deleted, then undelete it.
If one is found and it has not been deleted, then simply return -- as the
key value pair already exists.
Otherwise, call {\func{dbfind\_next\_space}} to find the next subdirectory
and position where the new value can be stored (which is at the end of the {\file{data}} file,
or $0$ if no {\file{data}} file yet exists).
The function {\func{load\_index}} at this subdirectory is used
to get a reverse sorted list of the current keys and positions.
The list is reversed and the new key and position are merged into the list.
This new list is stored in {\file{index}} replacing the previous contents.
The value is deserialized and appended to the end of the {\file{data}} file.
\item {\func{dbdelete}} takes a key and uses {\func{dbfind}} to find a subdirectory
where the corresponding {\file{index}} file contains the key.
If none is found, then do nothing.
Assume a subdirectory is found.
If the key is already in the {\file{deleted}} file of this subdirectory, then do nothing.
Otherwise, append the key to the {\file{deleted}} file.
If the number of deleted entries in this subdirectory exceeds 1024, then {\func{defrag}} is called.
\end{itemize}

One simple database module {\module{DbBlacklist}}
is defined by giving {\module{Dbbasic}} the type {\type{bool}} and 
base directory {\dir{blacklist}}.
This is intended to save keys corresponding to some blacklisted data that
should not be requested from peers.

Other instantions of {\module{Dbbasic}}
occur where the corresponding data types are defined.
For assets this is in {\module{assets}}
and
for (signed) transactions this is in {\module{tx}}
(see Chapter~\ref{chap:assetstx}).
For hcons elements and ctree elements,
giving approximations of parts of ledger trees,
this is in {\module{ctre}}
(see Chapter~\ref{chap:ctre}).
For block headers and blocks,
this is in {\module{block}}
(see Chapter~\ref{chap:block}).


\chapter{Formalized Mathematics}\label{chap:math}

The \module{mathdata} contains the code for representing types,
terms and proofs
and the code for type checking and proof checking.

Much of the code in this module was taken from the code for Egal~\cite{Brown2014} system.
The main difference in the syntax is Qeditas provides explicit support for type variables.


\chapter{Assets and Transactions}\label{chap:assetstx}

The module \module{assets} defines a type {\type{asset}}.
It also contains code to support the inputs and outputs of transactions.
The module \module{tx} defines a type {\type{tx}} of transactions
and a type {\type{stx}} of signed transactions,
as well as code for checking the validity of transactions
and their signatures.
``Validity'' of a transaction is a weak form of correctness
that a transaction must satisfy before asking if it is supported by the current ledger.

\section{Assets}

An asset consists of four pieces of information:
a hash value (the {\defin{asset id}}),
a 64-bit integer giving the block in which the asset was published (the {\defin{birthday}}),
an {\type{obligation}} (indicating who controls the asset)
and a {\type{preasset}} (determining the kind of asset).
In the case of an asset in the initial distribution,
the asset id is the 160-bit hash value corresponding to the p2pkh or p2sh address.
(Since no p2pkh or p2sh addresses in the snapshot had the same 160-bit address,
these asset ids are unique.)
In the case of an asset created as the output of a transaction,
the asset id is formed from hashing the transaction id paired with the index of the output
creating the identifier.
Assets in the initial distribution are given birthday $0$
and the first block will be considered to have block height $1$.
Obligations and preassets are described below.

{\bf{Note:}} Unit tests for the {\module{assets}} module are in {\file{assetsunittests.ml}}
in the {\dir{src/unittests}}
directory in the {\branch{testing}} branch.
These unit tests give a number of examples demonstrating how the functions described below should behave.
The {\branch{testing}} branch is, however, out of date with the code in the {\branch{dev}} and {\branch{master}} branches.

{\bf{Note:}} The Coq module {\coqmod{Assets}} is intended to correspond to {\module{assets}}.
There are Coq types {\coqtype{preasset}}, {\coqtype{obligation}} and {\coqtype{asset}}
corresponding to the types with the same names defined in OCaml.
One difference is that in the OCaml code an obligation also keeps a boolean indicating if the
asset is a reward since this was needed to implement forfeiture of rewards in case a staker
double signs.
Readers can examine the formal properties proven in {\coqmod{Assets}} to have a better
idea of what properties corresponding OCaml functions should satisfy.
For more information, see~\cite{White2015b}, although preassets in the version described there are restricted to currency units.

\subsection{Obligations}

We first consider the type {\type{obligation}}:
\begin{verbatim}
type obligation = (payaddr * int64 * bool) option
\end{verbatim}
Note that an
obligation may be empty, which usually means the address that {\defin{holds}}
the asset can spend it (here {\defin{holds}} refers to the address in the ledger
tree where the asset is stored).
In case an obligation is not empty, it consists of a triple $(\alpha,n,r)$.
Here $\alpha$ is a pay address (a p2pkh address or p2sh address) which must
sign in order to spend the asset.
(The holder of the asset is the one who can use the asset to stake.
Hence obligations can be used to ``loan'' an asset to a staker
without giving the staker the ability to ``spend'' the asset.)
The integer $n$ is the earliest block height at which the asset can be spent.
(The intention here is to ``lock'' an asset for a period of time. Such ``locked''
assets are given preference when staking.)
The boolean $r$ indicates if the asset is a reward for staking a block.\footnote{Philosophically, this should not be part of the ``obligation,'' but the reward indicator was added late and this was a simple way to include it.}
Rewards are considered special in the sense that they can be forfeited in the first 6 blocks
after the reward is issued, if the issuer provably double signs within the next 6 blocks.

\subsection{Categories of Preassets and Assets}

There are 11 kinds of {\defin{assets}}, as determined
by the corresponding {\defin{preasset}}: currency units, bounties, object ownership,
proposition ownership, negated proposition ownership,
object rights, proposition rights, markers,
theory publications, signature publications and document publications.
The type {\type{preasset}} consists of the following 11 corresponding constructors:
\begin{itemize}
\item ${\constr{Currency}}(n)$ represents $n$ cants of currency units, where $n$ is a 64-bit integer.
A {\defin{cant}} is the smallest currency unit considered in Qeditas.\footnote{The word ``cants'' is pronounced with a hard c as it is derived from the name Cantor.}
Currency units can be transfered by fulfilling the appropriate obligation (which usually simply
means signing the transaction spending the asset with an appropriate private key).
% Currency units are always held at pay addresses. % I no longer remember if this is true.
\item ${\constr{Bounty}}(n)$ represents $n$ cants as a bounty on a proposition.
Bounties are held at term addresses, specifically at the term address of a proposition in a theory.
A bounty can be spent (and transformed into currency) by the proposition owner 
or negated proposition owner.
Typically neither the proposition nor its negation have been proven in the theory
(and so it has neither a proposition owner nor a negated proposition owner)
when the bounty is placed.\footnote{The ``proposition owner'' is determined by the (nonempty) obligation at the proposition ownership asset held at the term address, if there is such an asset. Likewise, the ``negated proposition owner'' is determined by the obligation at the negated proposition ownership asset held at the term address, if there is such an asset.}
If someone publishes a document in which the proposition is proven,
the publisher declares the proposition owner.
Likewise, if someone publishes a document in which the negation of the proposition is proven,
the publisher declares the negated proposition owner.
In either case, the new owner (presumably the publisher) can then collect the bounty.
\item ${\constr{OwnsObj}}(\alpha,p)$ corresponds to a declaration of object ownership of a term (either
a pure term or a term in a theory).
The $\alpha$ is a pay address and the $p$ is an optional 64 bit integer.
The actual object owner is determined by the obligation of the corresponding asset (and so may
or may not be $\alpha$).
The address $\alpha$ is intended as an address others can pay in order to purchase rights
to use the object (as an imported parameter) in future documents.
The optional value $p$ gives the price (in cants) to purchase one right.
If $p$ is $0$, then the object can be freely used (without a need to purchase rights).
If $p$ is {\val{None}}, then the object cannot be used
in this way at all (and rights cannot be purchased).
(The object can always be used in a new document by repeating the definition.)
\item ${\constr{OwnsProp}}(\alpha,p)$ corresponds to a declaration of proposition ownership of a term (either
a pure term or a term in a theory).
The $\alpha$ is a pay address and the $p$ is an optional 64 bit integer.
The actual proposition owner is determined by the obligation of the corresponding asset (and so may
or may not be $\alpha$).
The address $\alpha$ is intended as an address others can pay in order to purchase rights
to use the proposition (as an imported known) in future documents.
The optional value $p$ gives the price (in cants) to purchase one right.
If $p$ is $0$, then the proposition can be freely used (without a need to purchase rights).
If $p$ is {\val{None}}, then the proposition cannot be used at all (and rights cannot be purchased).
(The proposition can always be used in a new document by reproving it.)
\item ${\constr{OwnsNegProp}}$ corresponds to a declaration of a negated proposition ownership of a term
in a theory.
Again, the ``owner'' is determined by the corresponding obligation.
This kind of asset is only to facilitate the collection of a bounty by disproving a conjecture
with a bounty.
\item ${\constr{RightsObj}}(\alpha,n)$ corresponds to the right to use
the object with term address $n$ times.
Some or all of these rights will be consumed when publishing a document which imports the object
as a parameter (omitting the definition).
Note that to use objects within a theory, rights may be required for the
pure object (independent of the theory) and for the object within the theory.
These are two different term addresses.
\item ${\constr{RightsProp}}(\alpha,n)$ corresponds to the right to use the proposition
with term address $\alpha$ $n$ times.
Some or all of these rights will be consumed when publishing a document which imports the proposition
as a known (without proof).
Note that to use propositions within a theory, rights may be required for the
pure proposition (independent of the theory) and for the proposition within the theory.
These are two different term addresses.
\item ${\constr{Marker}}$ is for part of the protocol for publishing a document.
A publication address is determined by the (privately known) publication with a (privately known) nonce.
A marker must be at the publication address (as an {\defin{intention to publish}})
for 144 blocks (see {\var{intention\_minage}}) before the actual publication can be published.
The idea is that the true author of the document publishes the marker roughly a day before
revealing the publication itself. The publication is revealed in the transaction publishing it.
At that point, a plagiarist could take the publication, compute a new nonce, publish a new marker
and then try to publish their copy. However, they would need to wait at least 144 blocks before their
copied version could be published. By that time, the original publication should already be published.
The order of publication is important since this may determine ownership of
newly defined objects and newly proven propositions.
\item ${\constr{TheoryPublication}}(\alpha,\nu,\tau)$ is a preasset
for publishing a theory specification ({\type{theoryspec}}) $\tau$.
The pay address $\alpha$ identifies the author (possibly ``publisher'' is more accurate)
and the corresponding transaction
creating such an asset must be signed by $\alpha$.
The hash value $\nu$ is a nonce to determine the publication address for the marker
which must be published 144 blocks before the publication can be published.
\item ${\constr{SignaPublication}}(\alpha,\nu,h,\Sigma)$ is a preasset
for publishing a signature specification ({\type{signaspec}}) $\Sigma$.
The pay address $\alpha$ identifies the author and the corresponding transaction
creating such an asset must be signed by $\alpha$.
The hash value $\nu$ is a nonce to determine the publication address for the marker
which must be published 144 blocks before the publication can be published.
The optional hash value $h$ identifies the theory in which the signature belongs.
An object or proposition can only be included in a signature if no rights are required
to use the object or proposition.
(The empty theory is identified by giving {\val{None}} for $h$.)
\item ${\constr{DocPublication}}(\alpha,\nu,h,\Delta)$ is a preasset
for publishing a document ({\type{doc}}) $\Delta$.
The pay address $\alpha$ identifies the author and the corresponding transaction
creating such an asset must be signed by $\alpha$.
The hash value $\nu$ is a nonce to determine the publication address for the marker
which must be published 144 blocks before the publication can be published.
The optional hash value $h$ identifies the theory in which the signature belongs.
(The empty theory is identified by giving {\val{None}} for $h$.)
\end{itemize}

The type {\type{asset}} of assets is now simply defined as a product.
\begin{verbatim}
type asset = hashval * int64 * obligation * preasset
\end{verbatim}
The functions {\func{assetid}}, {\func{assetbday}},
{\func{assetobl}} and {\func{assetpre}}
extract the components from the asset.

\subsection{Types for Transaction Inputs and Outputs}

The inputs of transactions will be pairs of addresses and asset identifiers (hash values)
of assets held at these addresses. The type {\type{addr\_assetid}}
plays the role of a transaction input and is defined as follows:
\begin{verbatim}
type addr_assetid = addr * hashval
\end{verbatim}
The outputs of transactions are triples of addresses, obligations and preassets.
(The asset identifier is determined by the transaction itself and the birthday
is determined by the block height in which the transaction is included.)
The type {\type{addr\_preasset}} plays the role of a transaction output and is defined as follows:
\begin{verbatim}
type addr_preasset = addr * (obligation * preasset)
\end{verbatim}
The inputs and outputs of a transaction can be elaborated into a pair of an address with an asset
in certain situations.
While checking a transaction is supported the input assets are looked up from the ledger tree
using the asset identifier. 
A transaction output gives the obligation and preasset.
When a transaction is being included in a block at a given height,
we know the birthday and can use this (along with the asset identifier which 
is derived from the transaction) to form the asset.
The type {\type{addr\_asset}} is included to represent such an elaborated input or output.
\begin{verbatim}
type addr_asset = addr * asset
\end{verbatim}

\subsection{Functions}

The functions {\func{hashobligation}} hashes an obligation (returning {\val{None}}
for the {\val{None}} obligation).
The functions {\func{hashpreasset}},
{\func{hashasset}},
{\func{hash\_addr\_assetid}},
{\func{hash\_addr\_preasset}} and
{\func{hash\_addr\_asset}}
hash the corresponding types.

As usual, there are functions for serializing and deserializing elements of these types:
{\serfunc{seo\_obligation}},
{\serfunc{sei\_obligation}},
{\serfunc{seo\_preasset}},
{\serfunc{sei\_preasset}},
{\serfunc{seo\_asset}},
{\serfunc{sei\_asset}},
{\serfunc{seo\_addr\_assetid}},
{\serfunc{sei\_addr\_assetid}},
{\serfunc{seo\_addr\_preasset}},
{\serfunc{sei\_addr\_preasset}},
{\serfunc{seo\_addr\_asset}}
and
{\serfunc{sei\_addr\_asset}}.

The purpose of the remaining functions exported by the {\module{assets}} module are as follows:
\begin{itemize}
\item {\func{new\_assets}} takes a birthday $b$, an address $\alpha$,
an {\type{addr\_preasset}} list (transaction outputs), a hash value (which should
be the hash of the transaction)
and an output index (which should be $0$ in the initial call)
and returns a list of assets which would be put into address $\alpha$
if the transaction is published at block height $b$.
The transaction hash and output index are used to compute the asset ids.
\item {\func{remove\_assets}} takes an asset list and a list of asset identifiers (the ``spent list'') and
returns the asset list after removing the assets with ids in the spent list.
\item {\func{get\_spent}} takes an address $\alpha$ and an {\type{addr\_assetid}} list (transaction inputs)
and returns a list of asset ids being spent from the given address.
\item {\func{add\_vout}} is similar to {\func{new\_assets}} except it is not specific to an address.\footnote{It might make sense to delete one of these functions in favor of the other.}
It takes a birthday $b$, a hash value (which should be the hash of the transaction),
an {\type{addr\_preasset}} list (transaction outputs)
and an output index (which should be $0$ in the initial call),
and returns an {\type{addr\_asset}} list consisting of the fully elaborated output assets
(assuming the transaction is published at block height $b$).
\item {\func{preasset\_value}} takes a preasset, a birthday (which should
be the birthday of the corresponding asset) and a block height
and returns the optional number of cants that a preasset is worth.
Only currency units and bounties are worth cants. For other preassets, {\val{None}} is returned.
If the preasset is a bounty preasset with $v$ cants, then the value is $v$ cants.
If the birthday is not $0$ (so the preasset was not part of the initial distribution)
and the preasset is a currency preasset with $v$ cants, then the value is $v$ cants.
Currency assets from the initial distribution (with birthday $0$) are treated in a special manner.
In particular, their value will halve along with the block rewards.
Suppose the birthday is $0$ and the preasset is a currency asset with $v$ cants.
Until block height $280,000$, the preasset is worth $v$ cants.
Block height $280,000$ is when the second reward halving occurs and should
occur roughly 5 years after the network begins running.
For the next $210,000$ blocks after block height $280,000$ the value
is divided in half (rounding down).
The value continues to be divided in half each $210,000$ blocks from that point on
until block height $11,410,000$ at which point all currency preassets with birthday $0$
have value $0$ cants.
Block height $11,410,000$ should occur after roughly 200 years.\footnote{The code for halving the value of the unclaimed initial distribution was added by Trent Russell in early 2016, at his suggestion.}
\item {\func{asset\_value}} returns the value of the underlying preasset.
\item {\func{asset\_value\_sum}} returns the sum of the value of a list of assets (where {\val{None}} is counted as $0$).
\item {\func{output\_signaspec\_uses\_objs}} takes
an {\type{addr\_preasset}} list (transaction outputs)
and returns a list of pairs of term addresses.
For each object imported as a parameter by a signature specification being published as one of the outputs,
$(\alpha,\beta)$ will be on the output list
where $\alpha$ is the term address given by the hash root $h$\footnote{Recall that term addresses are actually hash values, so $\alpha = h$.}
of the term which was used to define the object
and $\beta$ is the term address given by hashing $h$ with the type of the object and with the identifier of the current theory
(and then tagging this with $32$ to avoid accidental collision).
If an object is imported by multiple
different signatures being published, then the pair will be on the list multiple times.
The information is obtained by calling {\func{signaspec\_uses\_objs}} on appropriate preassets.
\item {\func{output\_signaspec\_uses\_props}} takes
an {\type{addr\_preasset}} list (transaction outputs)
and returns a list of pairs of term addresses.
For each proposition imported as a known by a signature specification being published as one of the outputs,
$(\alpha,\beta)$ will be on the output list
where $\alpha$ is the term address given by the hash root $h$
of the proposition
and $\beta$ is the term address given by hashing $h$ with the identifier of the current theory
(and then tagging this with $33$ to avoid accidental collision).
If a proposition is imported by multiple
different signatures being published, then the pair will be on the list multiple times.
The information is obtained by calling {\func{signaspec\_uses\_props}} on appropriate preassets.
\item {\func{output\_doc\_uses\_objs}} takes
an {\type{addr\_preasset}} list (transaction outputs)
and returns a list of pairs of term addresses.
For each object imported as a parameter by a document being published as one of the outputs,
$(\alpha,\beta)$ will be on the output list
where $\alpha$ is the term address given by the hash root $h$
of the term which was used to define the object
and $\beta$ is the term address given by hashing $h$ with the type of the object and with the identifier of the current theory
(and then tagging this with $32$ to avoid accidental collision).
If an object is imported by multiple
different documents being published, then the pair will be on the list multiple times.
The information is obtained by calling {\func{doc\_uses\_objs}} on appropriate preassets.
\item {\func{output\_doc\_uses\_props}} takes
an {\type{addr\_preasset}} list (transaction outputs)
and returns a list of pairs of term addresses.
For each proposition imported as a known by a document being published as one of the outputs,
$(\alpha,\beta)$ will be on the output list
where $\alpha$ is the term address given by the hash root $h$
of the proposition
and $\beta$ is the term address given by hashing $h$ with the identifier of the current theory
(and then tagging this with $33$ to avoid accidental collision).
If a proposition is imported by multiple
different document being published, then the pair will be on the list multiple times.
The information is obtained by calling {\func{doc\_uses\_props}} on appropriate preassets.
\item {\func{output\_creates\_objs}} takes
an {\type{addr\_preasset}} list (transaction outputs)
and returns a list of triples $(t,h,k)$
identifying objects defined in a document being published as one of the outputs.
Here $t$ is the (optional) hash value identifier of the theory in which the document lives,
$h$ is hash root of the term defining the object
and $k$ is the hash of the type of the object.
(The term address of the pure object will be $h$
and the term address of the object in the theory will be the hash of
$h$ with $t$, $k$ and the tag $32$.)
If an object is created by multiple
different documents being published, then the triple will be on the list multiple times.
The information is obtained by calling {\func{doc\_creates\_objs}} on appropriate preassets.
If the pure term address for a created object is unowned,
then it is new and must be given an owner (both as a pure object
and as an object within the theory) with the same transaction
publishing the document.
If the pure term address for a created object is owned,
but the term address within the theory is unowned, then
the object is new for the theory and the term address within the theory
must be given an owner (as an object) with the same transaction
publishing the document.
\item {\func{output\_creates\_props}} takes
an {\type{addr\_preasset}} list (transaction outputs)
and returns a list of pairs $(t,h)$
identifying propositions which are known as the result of a publication in the outputs.
Usually, this will mean $t$ is the (optional) hash value identifier of the theory in which the document lives
and $h$ is the hash root of a proposition proven in a document being published.
Alternatively, a pair $(t,h)$ can be included due to an axiom being assumed in a newly published theory
specification. In this case, $t$ is the hash value identifier of the theory derived from the
new theory specification and $h$ is the hash root of one of the propositions given as an axiom of the theory.
Again, multiple publications may result in $(t,h)$ being included multiple times.
The function uses {\func{doc\_creates\_props}}.
If the pure term address for a created proposition is unowned,
then it is new and must be given an owner (both as a pure proposition
and as a proposition within the theory) with the same transaction
publishing the document.
If the pure term address for a created proposition is owned,
but the term address within the theory is unowned, then
the proposition is new for the theory and the term address within the theory
must be given an owner (as a proposition) with the same transaction
publishing the document.
\item {\func{output\_creates\_neg\_props}} takes
an {\type{addr\_preasset}} list (transaction outputs)
and returns a list of pairs $(t,h)$
identifying propositions whose negations are proven in a document published in the outputs.
Here this means there is a document being published in the theory identified by the
(optional) hash value $t$
and a proposition $\neg s$\footnote{Here $\neg s$ actually means literally negation ($\lambda_o x_0\to\bot$) applied to $s$ or $s\to\bot$ where $\bot$ is $\forall_o x_0$.}
is proven in the document and $h$ is the hash root of $s$.
The information is obtained by calling {\func{doc\_creates\_neg\_props}} on appropriate preassets.
There is no requirement to declare an owner for a created negated proposition.
Negated propositions cannot be ``used'' in the sense that an object or proposition can
be used. The only purpose for declaring ownership of a negated proposition is to collect a bounty.
\item {\func{rights\_out\_obj}} takes an {\type{addr\_preasset}} list (transaction outputs)
and a term address $\alpha$ and sums the number of rights to use $\alpha$ as an object
created by the outputs.
\item {\func{rights\_out\_prop}} takes an {\type{addr\_preasset}} list (transaction outputs)
and a term address $\alpha$ and sums the number of rights to use $\alpha$ as a proposition
created by the outputs.
\item {\func{count\_obj\_rights}} takes a list of assets and a term address $\alpha$
and sums the number of rights to use $\alpha$ as an object contained in the asset list.
\item {\func{count\_prop\_rights}} takes a list of assets and a term address $\alpha$
and sums the number of rights to use $\alpha$ as a proposition contained in the asset list.
\item {\func{count\_rights\_used}} takes list of pairs of term addresses and a term address $\alpha$
and counts the number of times $\alpha$ occurs as one of the pairs. (Technically it counts
how many times $\alpha$ is at least one of the pairs, but in practice $(\alpha,\alpha)$ will
never occur.)
The purpose is to determine how many times $\alpha$ is ``used'' by publications
given in outputs of a transaction.
\item {\func{obj\_rights\_mentioned}} takes an {\type{addr\_preasset}} list (transaction outputs)
and returns a list of term addresses.
A term address is included in the output if
object rights for it are being explicitly output
(as {\constr{RightsObj}} preassets)
or if the object is being used in a document being published.
(We do not count uses in signature specification publications since
a signature specifications will only be allowed if rights are not required.)
\item {\func{prop\_rights\_mentioned}} takes an {\type{addr\_preasset}} list (transaction outputs)
and returns a list of term addresses.
A term address is included in the output if
proposition rights for it are being explicitly output
(as {\constr{RightsProp}} preassets)
or if the proposition is being used in a document being published.
(We do not count uses in signature specification publications since
a signature specifications will only be allowed if rights are not required.)
\item {\func{rights\_mentioned}} combines the results of {\func{obj\_rights\_mentioned}}
and {\func{prop\_rights\_mentioned}} to give a list of all term addresses
where rights are either being output or may be consumed to publish a document.
\item {\func{units\_sent\_to\_addr}} takes an address $\beta$
and an {\type{addr\_preasset}} list (transaction outputs)
and sums the value of the currency units being sent to $\beta$.
The purpose of this is to facilitate the purchase of rights by paying $\beta$.
\item {\func{out\_cost}} takes an {\type{addr\_preasset}} list (transaction outputs)
and sums the total cost of publishing the transaction.
This includes the value of all currency and bounty assets created by the output
as well as the burn cost required to publish theory specifications
and signature specifications.
\end{itemize}

\subsection{Asset Database}

The module {\module{DbAsset}} implements a database for storing assets
(see Chapter~\ref{chap:db}).
At the moment, this uses the basic file storage implementation {\module{Dbbasic2}}.
The function {\func{get\_asset}} takes a hash value (an asset id) and
uses {\func{dbget}} to try to look up the asset from the database.
If it is not found in the database,
then the asset is requested from network peers\footnote{This is not currently implemented. Some earlier code to do this is commented out.}
and the exception {\exc{GettingRemoteData}} is raised.
The idea is that the next time ${\constr{get\_asset}}$ is
called the asset
may have been put into the database after it was received from a peer.

\subsection{Creation of Objects and Propositions}\label{sec:outputcreates}

Let $h$ be a theory identifier (an optional hash value),
$\Delta$ be a document
and $\alpha$ be a term address.
\begin{itemize}
\item We say $(h,\Delta)$ {\defin{creates the object}} at $\alpha$ if
      there is a definition ${\constr{DocDef}}(\delta,s)$ in $\Delta$
      where $\alpha$ is either the term address of the pure object $s$ or
      of the object $s$ of type $\delta$ in the theory with theory identifier $h$.
\item We say $(h,\Delta)$ {\defin{creates the proposition}} at $\alpha$ if
      there is a proof ${\constr{DocPfOf}}(s,\cD)$ in $\Delta$
      where $\alpha$ is either the term address of the pure proposition $s$ or
      of the proposition $s$ in the theory with theory identifier $h$.
\item We say $(h,\Delta)$ {\defin{creates the negated proposition}} at $\alpha$ if
      there is a proof ${\constr{DocPfOf}}(s,\cD)$ in $\Delta$
      and a proposition $t$ where $s$ is either $\neg t$ or $t\to\bot$\footnote{Here $\bot$ is $\forall_o x_0$ and $\neg$ is $\lambda_o x_0\to\bot$.}
      and
      $\alpha$ is the term address 
      of the proposition $t$ in the theory with theory identifier $h$.\footnote{We do not consider negated pure propositions. Negated propositions only need to be considered for collection of bounties by disproving a conjecture. Conjectures only make sense for propositions in a theory. (Note that every proposition is provable in an inconsistent theory, and so for every pure proposition there is some theory in which the proposition is provable.)}
\end{itemize}

We extend these definitions from single documents to transaction outputs (which
may publish several documents).

Let $o$ be an {\type{addr\_preasset}} list (transaction outputs)
and $\alpha$ be a term address.
\begin{itemize}
\item We say $o$ {\defin{creates the object}} at $\alpha$
if there is some 
$$(\delta,(\omega,{\constr{DocPublication}}(\gamma,\nu,h,\Delta)))\in o$$
where $(h,\Delta)$ creates the object at $\alpha$.
\item We say $o$ {\defin{creates the proposition}} at $\alpha$
if there is some 
$$(\delta,(\omega,{\constr{DocPublication}}(\gamma,\nu,h,\Delta)))\in o$$
where $(h,\Delta)$ creates the proposition at $\alpha$.
\item We say $o$ {\defin{creates the negated proposition}} at $\alpha$
if there is some 
$$(\delta,(\omega,{\constr{DocPublication}}(\gamma,\nu,h,\Delta)))\in o$$
where $(h,\Delta)$ creates the negated proposition at $\alpha$.
\end{itemize}

Ownership of an object, proposition or negated proposition will be originally
justified by the creation of the object, proposition or negated proposition.
We need a notion of {\defin{support}} for this purpose.
Ownership preassets
are those of the form
${\constr{OwnsObj}}(\beta,p)$,
${\constr{OwnsProp}}(\beta,p)$
and
${\constr{OwnsNegProp}}$.
Let $o$ be an {\type{addr\_preasset}} list (transaction outputs)
and $\alpha$ be a term address.
We define when $o$ supports an ownership preasset at $\alpha$
by considering the three kinds of preassets.
\begin{itemize}
\item We say $o$ {\defin{supports}} ${\constr{OwnsObj}}(\beta,p)$ at $\alpha$
      if $o$ creates the object at $\alpha$.
\item We say $o$ {\defin{supports}} ${\constr{OwnsProp}}(\beta,p)$ at $\alpha$
      if $o$ creates the proposition at $\alpha$.
\item We say $o$ {\defin{supports}} ${\constr{OwnsNegProp}}$ at $\alpha$
      if $o$ creates the negated proposition at $\alpha$.
\end{itemize}

These notions will be used when we give the conditions
for a ledger tree to support a transaction.
In terms of the code, there is no single function checking if $o$ support $u$ at $\alpha$.
The functions
{\func{output\_creates\_objs}},
{\func{output\_creates\_props}},
and
{\func{output\_creates\_neg\_props}}
can be used to obtain term addresses which are created as objects, propositions
and negated propositions.
When we need to check for support in {\func{ctree\_supports\_tx\_2}}
in the module {\module{ctre}}
we will already have the values returned by
{\func{output\_creates\_objs}},
{\func{output\_creates\_props}},
and
{\func{output\_creates\_neg\_props}}
and will make use of these values at the time.

\section{Transactions}

The module \module{tx} defines a type {\type{tx}} for transactions,
a type {\type{stx}} for signed transactions
and functions for testing the validity of transactions and signed transactions.
Here {\defin{validity}} of a transaction refers only to properties that
can be checked without reference to the state of the ledger.
For properties that require the ledger state, we will speak of {\defin{support}}
for a transaction (see Chapter~\ref{chap:ctre}).
There is a slight exception, however. We check validity of input signatures 
({\func{check\_tx\_in\_signatures}}) of
a transaction relative to a list of assets being spent.
These assets would need to be looked up in the ledger
since the transaction only mentions the asset identifiers.

{\bf{Note:}} Unit tests for the {\module{tx}} module are in {\file{txunittests.ml}}
in the {\dir{src/unittests}}
directory in the {\branch{testing}} branch.
These unit tests give a number of examples demonstrating how the functions described below should behave.
The {\branch{testing}} branch is, however, out of date with the code in the {\branch{dev}} and {\branch{master}} branches.

{\bf{Note:}} The Coq module {\coqmod{Transactions}} is intended to correspond to {\module{tx}}.
The Coq types {\coqtype{Tx}} and {\coqtype{sTx}} correspond to the types {\type{tx}} and {\type{stx}} in the OCaml version.
Readers can examine the formal properties proven in {\coqmod{Transactions}} to have a better
idea of what properties corresponding OCaml functions should satisfy.
For more information, see~\cite{White2015b}.

The type {\type{tx}} of transactions is simply defined as a pair of
an {\type{addr\_assetid}} list (a list of pairs of addresses and hash value asset ids)
and an {\type{addr\_preasset}} list (a list of addresses associated with an obligation and a preasset).
\begin{verbatim}
type tx = addr_assetid list * addr_preasset list
\end{verbatim}

In order to avoid needing to give multiple signatures corresponding to the same address,
we allow transaction signatures to reference already given signatures.
This is accomplished using the type {\type{gensignat\_or\_ref}} defined as follows:
\begin{verbatim}
type gensignat_or_ref = GenSignatReal of gensignat
                      | GenSignatRef of int
\end{verbatim}
The idea is that we can have a list such at $[\sigma_1,\sigma_2,0,1]$
where $\sigma_1$ and $\sigma_2$ are of type {\type{gensignat}} giving real signatures
and $0$ and $1$ are references to $\sigma_2$ and $\sigma_1$, respectively.
The function {\func{getsig}} is used to determine the signature given a {\type{gensignat\_or\_ref}}
and a list of signatures.
In practice, the list of signatures given to {\func{getsig}} will be a list of the
previous signatures.
In the example $[\sigma_1,\sigma_2,0,1]$ the initial list of previous signatures is empty.
In this case {\func{getsig}} is first called with the signature $\sigma_1$ returns $\sigma_1$ and the list $[\sigma_1]$.
The next call to {\func{getsig}} would be with $\sigma_2$ and the list $[\sigma_1]$
returning $\sigma_2$ and the list $[\sigma_2;\sigma_1]$.
The third call to {\func{getsig}} would be with the reference $0$ and the list $[\sigma_2;\sigma_1]$
and return $\sigma_2$ (as element $0$ on the list) and the unchanged list $[\sigma_2;\sigma_1]$.
The fourth call to {\func{getsig}} would be with the reference $1$ and the list $[\sigma_2;\sigma_1]$
and return $\sigma_1$ (as element $1$ on the list) and the unchanged list $[\sigma_2;\sigma_1]$.

The type {\type{stx}} of signed transactions is a transaction associated
with two lists of generalized signatures ({\type{gensignat}}) or references to other signatures.
\begin{verbatim}
type stx = tx * (gensignat_or_ref list * gensignat_or_ref list)
\end{verbatim}
The first list gives ``input'' signatures and the second list gives the ``output'' signatures.
The ``input'' signatures are required to spend or move the assets in the input.
The ``output'' signatures are for the authors of publications.
(Without these ``output'' signatures, a plagiarist could create his own transaction
with someone elses publications and use his transaction to assign ownership of
new objects and propositions.)

The serialization and deserialization functions are
{\serfunc{seo\_tx}},
{\serfunc{sei\_tx}},
{\serfunc{seo\_txsigs}},
{\serfunc{sei\_txsigs}},
{\serfunc{seo\_stx}}
and
{\serfunc{sei\_stx}}.

We briefly describe the following exported functions:
\begin{itemize}
\item {\func{hashtx}} hashes a transaction, giving the identifier for the transaction (the {\defin{transaction id}}).
Note that this does not depend on signatures, and so transaction malleability is not an issue.
\item {\func{tx\_inputs}} is a projection function giving the input list of a transaction.
\item {\func{tx\_outputs}} is a projection function giving the output list of a transaction.
\item {\func{no\_dups}} is simply a helper function to ensure a list is duplicate free.\footnote{This is simply exported because the same function is required in the {\module{block}} module to ensure that no transaction is listed more than once in a block. As it has nothing to do with transactions, it should be moved to a more generic module imported by both {\module{tx}} and {\module{block}}.}
\item {\func{tx\_inputs\_valid}} takes a transaction input list
and checks that there is at least one input and there are no duplicate inputs.
\item {\func{tx\_outputs\_valid}} takes a transaction output list
and checks that it is valid in that the following conditions hold:
\begin{enumerate}
\item At most one owner (as an object or proposition) is declared for each term address.\footnote{Note that it is legal for one term address to obtain an owner as an object and another owner as a proposition. In fact, this should be common for pure terms. For example, the term $\forall_o x_0\to x_0$ (or $\top$) will have
a hash root $h_\top$. This can be both defined and owned as an object as well as proven and owned as a proposition.}
The function checking this is {\func{tx\_outputs\_valid\_one\_owner}}.
\item Each preasset is sent to an appropriate kind of address.
Ownership preassets are sent to term addresses
and publications and markers are sent to publication addresses.\footnote{This should probably be extended to ensure currency units and rights are only sent to pay addresses and bounties are only sent to term addresses.}
The function checking this is {\func{tx\_outputs\_valid\_addr\_cats}}.
\end{enumerate}
\item {\func{tx\_valid}} checks that both the inputs and outputs are valid in the sense above.
\item {\func{tx\_signatures\_valid}} takes a block height $b$, an asset list
and a signed transaction and checks that the input signatures and output signatures
are valid.
The work is partitioned in {\func{check\_tx\_in\_signatures}}
to check the input signatures
and {\func{check\_tx\_out\_signatures}}
to check the output signatures.
\begin{itemize}
\item {\func{check\_tx\_in\_signatures}} ensures that for each input
(except those spending markers and bounties)
there is an appropriate input signature. Here there are two possibilities.
There could be a signature permitting the spending of the asset ({\func{check\_spend\_obligation}})
or a signature permitting the movement of the asset ({\func{check\_move\_obligation}}).
A signature permitting the spending of the asset is either a signature
by the pay address in the obligation (assuming the appropriate block height has been reached)
or by the address where the asset is held if there is no obligation.\footnote{The obligation {\val{None}} defaults
to being the address where the asset is held with no block height requirement.}
A signature permitting the movement of the asset is by the address where the asset is held (assuming it is
a pay address) and is only allowed if there an output with exactly the same obligation and preasset
as the asset in question.
Essentially this allows the ``movement'' of an asset out of an address to a new address.\footnote{This could mitigate the effect of someone ``spamming'' someone else's address with unwanted and unowned assets.}
\item {\func{check\_tx\_out\_signatures}} ensures the authors of all publications
have signed the transaction. Note that the asset list is not required here.
\end{itemize}
\item {\func{tx\_signatures\_valid\_asof\_blkh}} is like {\func{tx\_signatures\_valid}}
but not given a block height. Instead it finds the minimum block height at which the
signatures are valid and returns this block height (or {\val{None}} if there is no such block height).
\item {\func{txout\_update\_ottree}} takes a transaction output list
and uses it to update the current (possibly empty) {\type{ttree}} by including any new
theories created by publishing theory specifications.
\item {\func{txout\_update\_ostree}} takes a transaction output list
and uses it to update the current (possibly empty) {\type{stree}} by including any new
signatures created by publishing signature specifications.
\end{itemize}

\subsection{Databases for Transactions and Signatures}

The module {\module{DbTx}} implements a database for storing transactions
and the module {\module{DbTxSignatures}} implements a database for storing
transaction signatures
(see Chapter~\ref{chap:db}).
In both cases the key is the transaction id (the hash of the transaction,
not including the signatures).
To recover a value of type {\type{stx}} both of the values are required.
At the moment, both of these use the basic file storage implementation {\module{Dbbasic2}}.



\chapter{Ledger Trees}\label{chap:ctre}

The {\module{ctre}} module
implements compact trees (``ctrees'')
and supporting functions.
Compact trees are used to approximate the state of the ledger
by recording what assets are held at what addresses.
The {\module{ctregraft}} module
implements a way to graft information onto a compact
tree in order to form an approximation with more information.

{\bf{Note:}} Unit tests for the {\module{ctre}} and {\module{ctregraft}} modules are in {\file{ctreunittests.ml}}
in the {\file{src/unittests}}
directory in the {\branch{testing}} branch.
These unit tests give a number of examples demonstrating how the functions described below should behave.

\section{Compact Ledger Trees}

We describe the main content of the module {\module{ctre}}.

{\bf{Note:}} The largest part of the Coq formalization deals with verifying compact trees
indeed represent ledgers as intended.
The Coq module {\coqmod{LedgerStates}} represents ledgers as functions
describing the current state of the ledger (see the Coq type \coqtype{statefun}).
The Coq module {\coqmod{MTrees}} uses a form of Merkle tree~\cite{Merkle1980} to approximate such a
{\coqtype{statefun}} function.
The Coq types {\coqtype{hlist}} and {\coqtype{nehlist}} defined in {\coqmod{MTrees}} correspond to 
the types {\type{hlist}} and {\type{nehlist}} defined in {\module{ctre}} in the OCaml code.
The dependent Coq type {\coqtype{mtree}} $n$ is a Merkle tree with height $n$,
where the default case is $n=162$ (since Qeditas addresses are determined by 162 bits).
The Coq module {\coqmod{CTrees}} uses the compact tree (essentially a Patricia tree)
to represent the Merkle tree and hence approximate the ledger state.
The dependent Coq type {\coqtype{ctree}} $n$ is a compact tree with height $n$.
In the OCaml code the corresponding type is the simple type {\type{ctree}}.
Several functions are defined by recursion on $n$ in both the Coq and OCaml versions,
even though in the OCaml version the requirement that the compact tree has height $n$
is no longer enforced by the type system.
Note that {\coqmod{CTrees}} (mostly) corresponds to {\module{ctre}},
while {\coqmod{LedgerStates}} and {\coqmod{MTrees}} are only needed in the theory
and (for the most part) have no counterpart in the OCaml code.
Note also that frames and compact tree abbreviation nodes are not
represented in the Coq formalization as these were added later to the
OCaml code.
For more information, see~\cite{White2015b}.

\subsection{Coin-age}

We first consider some variables which affect the likelihood of
coins to stake.\footnote{Since this has nothing to do with compact trees, it should be moved to a more appropriate module.}
The current settings have been chosen after
doing some simulations to determine a reasonable mixture
between staking of coins in the initial distribution
vs. staking of new coins issued through block rewards.
(Initial simulations revealed that block rewards could
easily dominate staking in the first few weeks unless their
influence was dampened.)

Typically, unlocked currency assets age quadratically as $(1+\lfloor\frac{a}{512}\rfloor)^2$ where $a$ is the number of blocks
since the asset became {\defin{mature}}. This continues until a maximum age is reached.
Users may commit currency assets to stake by locking them.
Locked non-reward assets mature more quickly and, once mature, have their maximum age
until it is close to the block height at which they will be unlocked, at which point
they will be ineligible for staking.
Rewards are necessarily locked. They age like unlocked currency assets until it is close to
the block height at which they will be unlocked, at which point they will be ineligible for staking.

\begin{itemize}
\item {\var{maximum\_age}} is the number of blocks after which unlocked coins
stop aging. This is currently set to $2^{14}$. With a 10 minute average block time,
this means unlocked coins reach their maximum age after roughly $4$ months.
(Coins in the initial distribution are exceptional: they have birthday $0$ and start with their maximum age.)
\item {\var{maximum\_age\_sqr}} is $(1+\lfloor \frac{a}{512}\rfloor)^2$ where $a$ is {\var{maximum\_age}}.
This is the maximum factor that can be used to determine an asset's coin-age.
Given current settings this is $33^2$, i.e., $1089$.
\item {\var{reward\_maturation}} indicates how old a reward must be before it can begin staking. This is currently set to $512$.
\item {\var{unlocked\_maturation}} indicates how many blocks must pass before
an unlocked asset can be used for staking. It is currently set to $512$.
\item {\var{locked\_maturation}} indicates how many blocks must pass before
a new locked currency asset can be used for staking.
It is currently set to $8$. This implies that $8$ blocks after creating a locked
currency asset, the currency asset can be used for staking with its maximum age,
until it is close to being unlocked.\footnote{The intention here is to encourage
stakers to commit to ``locking'' some of their coins only for use in staking.}
\item {\var{close\_to\_unlocked}} indicates the point at which locked assets can
no longer be used for staking. It is currently set to $32$, meaning that
the locked asset cannot be used for staking if it is $32$ blocks from being spendable
(or even after it is spendable).
\end{itemize}

Rewards must be locked until a certain block height before they can be spent.
In order to prevent rewards from dominating the early staking process,\footnote{If rewards could be unlocked quickly, then they could be
spent to create a locked non-reward asset. This locked non-reward asset would
very quickly mature and begin staking with its maximum age.}
this lock time is very long at first (16384 blocks, roughly 4 months), but reduces to
128 blocks (roughly 1 day) over the course of the first 114688 blocks (roughly 2 years).
The function {\var{reward\_locktime}} takes a given block height and
returns the minimum number of blocks a reward must be locked.
For the first 16834 blocks, the reward locktime is $16384$ (the same as {\var{maximum\_age}}).
Every 16384 blocks, the reward locktime is halved
until it reaches 128, where it remains indefinitely.

The function {\func{coinage}} computes the {\defin{coin-age}} of a
currency asset given the current block height.
As described above the ``age'' ranges from $0$ to $1089$,
with a quadratic increase each $512$ blocks.
(Specifically, the ``age'' progresses to be $n^2$
where $n$ is incremented from $1$ to $33$ each $512$ blocks
and then remains at $33$.)
Assume $v$ is the number of cants in the currency asset.
If the birthday of the asset is $0$, then it is part of the initial distribution
and its coin-age is $1089 v$.
Other than coins in the initial distribution,
there are three cases: unlocked currency assets, locked rewards and locked non-rewards.
Unlocked currency asset and locked rewards mature after $512$ blocks and then age quadratically
as described above. In the case of locked rewards, the coin-age drops to $0$
after block height $l-32$, where $l$ is the lock given for the reward.
Locked non-rewards mature after $8$ blocks and then have coin-age $1089 v$
until the coin-age drops to $0$ after block $l-32$, where $l$ is the lock given.

\subsection{Approximating Asset Lists by Hlists}

An asset list can be approximated by an {\defin{hlist}} $\cH$, a value of type {\type{hlist}}.
The constructors for {\type{hlist}} are as follows:
\begin{itemize}
\item ${\constr{HHash}}(h)$ approximates a nonempty asset list with hash root $h$.\footnote{The hash root of an asset list does not seem to be explicitly defined in the code. However, it could be defined using {\func{ohashlist}} and {\func{hashasset}} and this seems to be the intended hash root.}
\item ${\constr{HNil}}$ approximates the empty asset list.
\item ${\constr{HCons}}(a,\cH)$ where $a$ is an asset and $\cH$ is an hlist.
\end{itemize}
The idea is that an hlist explicit lists a prefix of the assets ending with a hash root
summarizing the rest of the asset list. It is also possible for the hlist to list all the assets.
Elements of this type can be serialized and deserialized using
{\serfunc{seo\_hlist}}
and {\serfunc{sei\_hlist}}.

The type {\type{nehlist}} represents nonempty hlists and has two constructors:
\begin{itemize}
\item ${\constr{NehHash}}(h)$ corresponds to the hlist ${\constr{HHash}}(h)$.
\item ${\constr{NehCons}}(a,\cH)$ corresponds to the hlist ${\constr{HCons}}(a,\cH)$.
\end{itemize}
Elements of this type can be serialized and deserialized using
{\serfunc{seo\_nehlist}}
and
{\serfunc{sei\_nehlist}}.

We briefly relevant functions.
\begin{itemize}
\item {\func{nehlist\_hlist}} converts a nonempty hlist to an hlist.
\item {\func{hlist\_hashroot}} computes an optional hash root for an hlist, with {\val{None}} playing
the role of the hash root for {\constr{HNil}}.
\item {\func{nehlist\_hashroot}} computes a hash root for an nehlist,
the same as the one given by {\func{hlist\_hashroot}}.
\item {\func{in\_hlist}} and {\func{in\_nehlist}} check if an asset is explicitly listed in
the hlist or nehlist. Note that this will return {\val{false}} if the asset is not explicitly listed
but is an asset on the part of the list being summarized by ${\constr{HHash}}$ or ${\constr{NehHash}}$.
\item {\func{print\_hlist}} and {\func{print\_hlist\_to\_buffer}} print hlists and are included for debugging purposes.
\end{itemize}

\subsection{Compact Trees}

The intention of a compact tree is to provide a binary tree approximation of
a function from addresses (162-bit sequences) to hlists, where the hlists
approximate the assets held at the addresses.
In general, functions on compact trees will be defined by recursion and so we
usually need to consider compact trees at level $n$ (corresponding to functions
from $n$-bit sequences to hlists).
A $0$ bit corresponds to the left child while a $1$ bit corresponds to the right child.

The vast majority of the leaves will be empty, and so compact trees
only store the nonempty parts explicitly.
The empty compact tree (with no assets stored at leaves) can be thought of as
represented by {\val{None}}, and the type {\type{ctree option}}
is used when the empty compact tree should be considered.

Compact trees $\cC$ are values of type ${\type{ctree}}$, which is defined by the following constructors.
\begin{itemize}
\item ${\constr{CLeaf}}(\overline{b},\cH)$ is a compact tree with a single nonempty leaf
at the location determined by the bit sequence (list of booleans) $\overline{b}$
and containing the nonempty hlist $\cH$.
\item ${\constr{CHash}}(h)$ is a compact tree with a hash $h$. This approximates every compact tree with hash root $h$.
\item ${\constr{CAbbrev}}(h_r,h_a)$ is an abbreviation for a compact tree with hash root $h_r$
and which hashes to give $h_a$.
The actual compact tree which hashes to $h_a$
should be saved in a file which can be loaded into memory as needed.\footnote{In the corresponding Coq code, there is no notion of an abbreviation node. This was added in the implementation because too much memory is consumed if one attempts to keep entire compact trees in memory.}
\item ${\constr{CLeft}}(\cC)$ is the compact tree with $\cC$ as its left child and the empty tree as its right child.
\item ${\constr{CRight}}(\cC)$ is the compact tree with the empty tree as its left child and $\cC$ as its right child.
\item ${\constr{CBin}}(\cC_0,\cC_1)$ is the compact tree with two nonempty children: $\cC_0$ on the left and $\cC_1$ on the right.
\end{itemize}
Elements of type ${\type{ctree}}$ can be serialized and deserialized using
{\serfunc{seo\_ctree}}
and
{\serfunc{sei\_ctree}}.

Three hashing functions are important:
\begin{itemize}
\item {\func{hashctree}} computes a unique hash value for a compact tree.
This is used to obtain a unique name identifying a compact tree
and is used, in particular, to give names to files storing a binary
representation of the compact tree.
\item {\func{ctree\_hashroot}} computes a hash root for the compact tree.
Many different compact trees will give the same hash roots, however they will
always approximate the same ledger state.
For example, they can differ in where they include ${\constr{CHash}}$
and ${\constr{CAbbrev}}$ nodes, as well has the level of detail included in
hlists at the leaves.
\item {\func{octree\_hashroot}} takes an optional compact tree and returns an optional hash value.
It simply returns {\val{None}} for the empty compact tree {\val{None}}
and returns the result of {\func{ctree\_hashroot}} otherwise.
\item {\func{ctree\_lookup\_asset}} takes an asset id (a hash value), a ctree and a bit sequence.
It traverses to leaf in the ctree following the bit sequence.
It then tries to look up the asset with the asset id in the nonempty hlist at the leaf.
If the leaf is empty or the asset is not found, then {\val{None}} is returned. Otherwise, the asset is returned.
\item {\func{remove\_hashed\_ctree}} deletes a file in which an abbreviated compact tree has been saved.
\item {\func{archive\_unused\_ctrees}} uses the difference between two compact trees to determine which compact tree abbreviations can be ``archived.'' The intention is that once a block height has become a checkpoint,
then the nodes in the compact tree unused beyond that height will no longer be needed.
These nodes are simply listed in a file {\file{archive}} and can be explicitly deleted later.
\item {\func{remove\_unused\_ctrees}} is similar to {\func{archive\_unused\_ctrees}} except that
it actually deletes the files.
\item {\func{ctree\_pre}} takes a bit sequence, a compact tree and an integer
and returns the subtree located where the bit sequence indicates (or {\val{None}} if
the subtree is empty, hence implicit) along with the depth.\footnote{It is unclear why the depth is returned or even computed as it seems to be simply the length of the bit sequence added to the integer. It is possible this is simply a remnant of some debugging information.}
\item {\func{ctree\_addr}} given an address and a compact tree,
returns the leaf (as a compact tree) at that address.
This leaf could either be {\val{None}} indicating no assets are held at the address,
or it could be a nonempty hlist, approximating the assets held at the address.
It also returns the depth, which should always be 162, so it can be ignored.
\item {\func{get\_ctree\_abbrev}} loads and returns a ctree given its hash, or
raises the {\exc{Failure}} exception if the appropriate file cannot be found.
\item {\func{octree\_lub}} takes the ``least upper bound'' of two optional compact trees.
These are assumed to be ``compatible'' in the sense that they must have the same hash root.
This means either both will be empty ({\val{None}})
or both will be compact trees.
The least upper bound of two empty compact trees is the empty compact tree.
For nonempty compact trees, the recursive structure is followed,
abreviations are expanded, and if one of the trees is a {\constr{CHash}}
node then the other tree is taken.
The idea is that if some information is in at least one of the trees,
then the information will be in the resulting tree.
\item {\func{print\_ctree}} and {\func{print\_ctree\_all}} print compact trees and are included for debugging purposes.
\end{itemize}

\subsection{Local Frames and Remote Frames}

A {\defin{frame}} specifies how to represent a compact tree,
including information about which parts of the tree should
be explicit and which should be summarized as a hash root.
In addition, a local frame specifies when to save
part of a compact tree in a file as an abbreviation.
There are two types of frames implemented as the types
{\type{frame}} (for {\defin{local frames}}) and {\type{rframe}} (for
{\defin{remote frames}}).\footnote{In the Coq formalization, there is
no notion of a frame. Local frames were implemented to have a way of
saving parts of a compact tree in local files, to be loaded when
needed. Remote frames were implemented to have a way of communicating
to remote nodes which part of the compact tree are being explicitly
stored and modified by a node. Such information would be useful for a
node that wanted to request a previously unfollowed part of the
compact tree. At the moment, local frames are used to save parts of
compact trees in files in the {\dir{ctrees}} directory, while remote
frames are not used at all.}

Local frames $\cF$ (or simply {\defin{frames}}) are elements of the type {\type{frame}}, which is defined with the
following constructors:
\begin{itemize}
\item ${\constr{FHash}}$ indicates that only the hash root of the corresponding compact tree should be stored.
\item ${\constr{FAbbrev}}(\cF)$ indicates that the corresponding compact tree $\cC$ should be represented
as an abbreviated pair of hash values $h_r$ and $h_a$.
Here $h_r$ is the hash root of the compact tree and $h_a$ is a full hash of the
compact tree $\cC'$ resulting from representing $\cC$ according to the frame $\cF$.
\item ${\constr{FAll}}$ indicates that the full compact tree should be included.
\item ${\constr{FLeaf}}(\overline{b},n)$ indicates that only the indicated leaf should be included, and all other nonempty parts should be approximated by ${\constr{CHash}}$-nodes.\footnote{This is intended to be useful if a node wants to follow specific addresses.}
The optional value $n$ indicates how long of an asset prefix the nonempty hlist at the leaf should explicitly list. If $n$ is {\val{None}}, then all assets are listed.
\item ${\constr{FBin}}(\cF_0,\cF_1)$ indicates that the left child of the compact tree should be represented according to $\cF_0$ and the right child according to $\cF_1$.
\end{itemize}
Elements of this type can be serialized and deserialized using
{\serfunc{seo\_frame}}
and
{\serfunc{sei\_frame}}.

Remote frames $\cR$ are elements of the type {\type{rframe}}
and essentially consists of frames without abbreviation nodes.
Hence the type is defined by four constructors which correspond to the other four
constructors described above:
\begin{itemize}
\item ${\constr{RFHash}}$
\item ${\constr{RFAll}}$
\item ${\constr{RFLeaf}}(\overline{b},n)$
\item ${\constr{RFBin}}(\cF_0,\cF_1)$
\end{itemize}
Elements of this type can be serialized and deserialized using
{\serfunc{seo\_rframe}}
and
{\serfunc{sei\_rframe}}.

The current local frame is stored in the variable
{\var{localframe}} and the hash of the current local
frame is stored in the variable {\var{localframehash}}.
These values are intended to be set upon start-up
by loading a value from a file (see {\func{load\_currentframe}})
and be modifiable by the user.
(There is code in {\file{qeditascli.ml}} to change the current frame
using commands,
calling some of the functions described below.)

A remote frame of the form ${\constr{RFBin}}({\constr{RFAll}},{\constr{RFAll}})$
is a redex with reduct ${\constr{RFAll}}$.
A remote frame is normal if it has no redexes.

We briefly describe some of the functions involving frames
exported by the {\module{ctre}} module.
\begin{itemize}
\item {\func{frame\_filter\_ctree}} takes a local frame $\cF$ and a compact tree $\cC$
and changes the representation of $\cC$ according to $\cF$.
This means nodes in $\cC$ corresponding to ${\constr{FHash}}$ nodes in $\cF$
are replaced by ${\constr{CHash}}$ nodes storing the hash root of the node.
It also means nodes in $\cC$ corresponding to ${\constr{FAbbrev}}$ nodes in $\cF$
are replaced by ${\constr{CAbbrev}}$ nodes storing the hash root and full hash of the node,
saving the subtree in a file if necessary.
It is possible that $\cC$ does not have enough information to form the
representation required by $\cF$ (e.g., if a ${\constr{CHash}}$ node occurs in $\cC$ corresponds
to ${\constr{FBin}}$ in $\cF$),
in which case the exception {\exc{InsufficientInformation}} is raised.
\item {\func{frame\_filter\_octree}} is a wrapper for {\func{frame\_filter\_ctree}} handling
optional ctrees, sending {\val{None}} to {\val{None}}.
\item {\func{rframe\_filter\_ctree}} is similar to {\func{frame\_filter\_ctree}} except using remote frames.
\item {\func{rframe\_filter\_octree}} is similar to {\func{frame\_filter\_octree}} except using remote frames.
\item {\func{normalize\_frame}} takes a local frame and creates a remote frame
by removing abbreviation nodes and by normalizing the result.
\item {\func{rframe\_lub}} combines two normalized remote
 frames to give a normalized remote frame
 describing what at least one of the two frames stores.
 The intention is that this function could be used for a peer to summarize
 what information at least one of its peers has, and then share this
 summary with all peers. In principle this could help locate a peer with specific
 information in the ledger tree.
\item {\func{wrap\_frame}} ensures that a local frame starts with ${\constr{FAbbrev}}$, by adding it if necessary.
\item {\func{hashframe}} hashes a frame to obtain a unique identifier for it.
\item {\func{frame\_add\_leaf}} changes a frame so that a given leaf (specified by an address)
will be explicit, possibly up to a given prefix length.
\item {\func{frame\_set\_hash\_pos}} changes a frame so that a certain position (specific by a list of booleans) will be summarized by a hash root.
\item {\func{frame\_set\_abbrev\_pos}} changes a frame so that a certain position will be abbreviated.
\item {\func{frame\_set\_abbrev\_level}} changes a frame to ensure that every node at a certain
level of a tree will be abbreviated.\footnote{One example of a frame that has been used to compute the initial distribution compact tree has abbreviations at levels $2$, $10$ and $18$. This means that the top two levels of the compact tree will end in abbreviations for the trees representing p2pkh addresses, p2sh addresses, term addresses and publication addresses. These subtrees in turn will have abbreviations corresponding to the first and second bytes of the addresses.}
\item {\func{frame\_set\_all\_pos}} changes a frame so that everything below a certain position will be explicit.
\item {\func{build\_rframe\_to\_req}} takes a local frame and a compact tree
and finds the parts of the compact tree where information is missing.
These missing parts are used to create a remote frame.
The intention is that this remote frame can be used to request the missing
information from peers.
\item {\func{split\_rframe\_for\_reqs}} takes an integer $n$ and a remote frame
and returns a list of remote frames with the given frame as a least upper bound.
This is essentially computed by traversing to depth $n$ and partitioning the
request into $2^n$ parts to request.
The intention is to use this to factor large requests 
into small requests of parts of the compact tree from peers.
\item {\func{load\_root\_abbrevs\_index}} loads the contents of a file {\file{rootabbrevsindex}}
which associates hash roots of compact trees
to hashes of local frames and compact trees.
This index is used to find the abbreviation for a compact tree given a hash root
and hash of a local frame.
The intention is to only store this information for the hash root for the top
of a compact tree (corresponding to addresses, i.e., 162-bit sequences)
and not subtrees (of which there are too many).
\item {\func{lookup\_frame\_ctree\_root\_abbrev}} takes a hash root of a compact tree
and the hash of a local frame
and returns the hash of the compact tree
(or raises the exception {\exc{Not\_found}}).
Note that this hash is needed to load the abbreviation.
\item {\func{lookup\_all\_ctree\_root\_abbrevs}} takes a hash root and returns a list of
pairs of hash values, the first being the hash of a local frame
and the second being the hash of a corresponding compact tree.
\end{itemize}

\subsection{Transactions}

We now describe functions relating compact trees and transactions.
One of the main concepts is that of {\defin{support}}.
In short, we say a compact tree supports a transaction if for each
input there is a corresponding asset held at the given address
and that a number of conditions hold.
To be more precise, there are further dependencies.
For example, some of these conditions depend on the block height
(e.g., to ensure an old enough intention justifies a publication).\footnote{Lock heights are not checked here, but instead are checked with the signatures of transactions. The reason is that lock heights prevent an asset from being spent, but do not prevent assets from being moved. The distinction between being spent and being moved is not relevant for support.}
Also, we must check the correctness of publications
which may require looking up a theory or signature.
These dependencies
are explicit in the function {\func{ctree\_supports\_tx}}
which checks if a compact tree supports a transaction,
given a block height, theory tree ({\type{ttree}}) and signature tree ({\type{stree}}).
The conditions to be checked often make reference to the actual assets
being spent (which are referred to in the transaction by their assetids).
The function {\func{ctree\_lookup\_input\_assets}} elaborates
the inputs by looking up the assets.
The exception {\exc{NotSupported}} is raised if
some condition required for a transaction to be supported fails,
which can happen either because the assets being spent cannot be found
in the compact tree or because of failure of some condition.

Let $\cC$ be a compact tree.
We say an asset $a$ is {\defin{held at $\alpha$ in $\cC$}}
if there is a nonempty hlist $\cH$ at leaf $\alpha$ in $\cC$ and $a$
is explicitly listed in $\cH$ (see {\func{in\_nehlist}}).
Let $\iota$ be an {\type{addr\_assetid}} list (a list of transaction inputs)
and $\iota'$ be a list of pairs $(\alpha,a)$ of addresses and assets.
We say $\iota'$ is an {\emph{elaboration of $\iota$ relative to $\cC$}}
if for each $(\alpha,h) \in\iota$ there is a pair $(\alpha,a) \in\iota'$
where $a$ is held at $\alpha$ in $\cC$ and $a$ has assetid $h$.\footnote{Technically, in the implementations the lists $\iota$ and $\iota'$ will list assetids and assets in the same order.}
The function {\func{ctree\_lookup\_input\_assets}} 
computes an elaboration $\iota'$ given $\iota$ and $\cC$,
raising {\exc{NotSupported}} if 
there is no asset $a$ with assetid $h$ held at $\alpha$ in $\cC$
for some $(\alpha,h)$ in $\iota$.

The function {\func{ctree\_supports\_tx}} simply
calls {\func{ctree\_lookup\_input\_assets}} to obtain $\iota'$
and then calls {\func{ctree\_supports\_tx\_2}} with this extra information.
The function {\func{ctree\_supports\_tx\_2}} checks the conditions for
$\cC$ to support $\tau=(\iota,o)$ with elaborated input $\iota'$
relative to a block height $b$, a theory tree and a signature tree.\footnote{The function {\func{ctree\_supports\_tx\_2}} is also given a list of assets, but this is simply the second components of the pairs in $\iota'$.}
Support requires several conditions.
We describe each condition as a single sentence followed by
a longer description.\footnote{Similar conditions can also be found in the Coq formalization as {\coqfunc{ctree\_supports\_tx}} in {\file{CTrees.v}}.}
\begin{enumerate}
\item {\sc{All output addresses are supported.}}
That is, for each $(\alpha,u)\in o$ there is no ${\constr{CHash}}$ node along the
path to the leaf in $\cC$ with position $\alpha$.
This is needed so that the new assets can be added to the leaves.
For each $(\alpha,u)\in o$ there are two possibilities: either there are no assets currently
held at $\alpha$
or there are assets represented by the nonempty hlist $\cH$ at $\alpha$.
If there are no assets, the new nonempty hlist will only contain the new assets.
If there are currently assets represented by $\cH$, we only need to
push the new assets onto the explicit prefix of $\cH$.
Note that this is possible even if $\cH$ is only the hash root of the hlist.
\item {\sc{If an object or a proposition is used in a signature specification, then it must be royalty-free to use.}}
For the definition of ``used'' see
{\func{output\_signaspec\_uses\_objs}}
and
{\func{output\_signaspec\_uses\_props}}.
Recall that each signature specification is intended for a specific theory (possibly the empty theory).
Each parameter in a signature specification will correspond to two term
addresses: one for the pure object and one for the object in the theory.
Both of these addresses must be owned as objects and the ownership assets
must both give $0$ as the price of a right to use the object.
Likewise each axiom in a signature specification will correspond to two term
addresses: one for the pure proposition and one for the proposition in the theory.
Both of these addresses must be owned as propositions and the ownership assets
also must give $0$ as the price of a right to use the proposition.
The reason for this condition is to prevent someone from paying once
for the right to use an object or a proposition
and then publishing it in a signature which is then free for anyone to use.
Of course, someone can purchase the ownership assets from the owners
and then make the corresponding objects and propositions free to use.
\item {\sc{If rights are consumed in the input, then they must be mentioned in the output.}}

\item {\sc{Rights must be balanced.}}

\item {\sc{Publications are correct and were declared in advance by a sufficiently old intention.}}


\item {\sc{If an ownership asset is spent in the input, then it must be included as an output.}}
That is, once a term address has an owner, it will always have an owner.
This is necessary since ownership is used to determine which objects have certain types and which propositions
have been proven (in both cases relative to a theory).
\item {\sc{Newly claimed ownership must be actually new and must be supported by a document published by the transaction.}}
\item {\sc{New objects and propositions must be given ownership by the transaction publishing the document.}}

\item {\sc{Bounties can only be collected by the owners of propositions or negated propositions.}}

\end{enumerate}
An attentive reader will note that none of these conditions require the currency units consumed
in the input to be at least as great as the currency units created in the outputs (plus those
required to be burned to publish theories and signatures).
This is not required for support,
and will not be true for coinstake transactions (which receive a reward).

We summarize the descriptions of these three main functions discussed above as follows:
\begin{itemize}
\item {\func{ctree\_lookup\_input\_assets}} takes a compact tree and an
{\type{addr\_assetid}} list (a list of transaction inputs)
and uses {\func{ctree\_lookup\_asset}} to look up the assets corresponding to
the assetids, returning the resulting list of pairs of addresses and assets.
If one of the assetids cannot be found in the compact tree,
the exception {\exc{NotSupported}} is raised.
\item {\func{ctree\_supports\_tx}} checks if a compact tree supports a transaction.
The function also requires an optional {\type{ttree}} (with all the currently known theories)
an optional {\type{stree}} (with all the currently known signatures)
and the current block height.
If the transaction is not supported, the exception {\exc{NotSupported}} is raised.
If the transaction is supported,
the difference between the currency units output or burned and the currency units input
is returned. If this value is negative, then it corresponds to a fee.
If the value is positive, then it corresponds to a reward.
\item {\func{ctree\_supports\_tx\_2}} is the same as {\func{ctree\_supports\_tx}}
except it also receives two extra inputs:
a list of the input addresses associated with their assets
and a list of those assets.
\end{itemize}

If a compact tree supports a transaction or list of transactions,
typically some small approximation of the compact tree also provides the support.
We next describe functions to construct such small approximations.
\begin{itemize}
\item {\func{full\_needed}} takes a {\type{addr\_preasset}} list (a list of transaction outputs)
and returns a list of bit sequences (addresses represented as boolean lists)
indicating which leaves need to have their full list of assets explicit
in order to check if the transaction with these outputs
is supported. 
\item {\func{get\_tx\_supporting\_octree}} takes a transaction and (optional) compact tree
and returns an approximation of the (optional) compact tree sufficient to support the transaction.
\item {\func{get\_txl\_supporting\_octree}} takes a list of transactions and (optional) compact tree
and returns an approximation of the (optional) compact tree sufficient to support the transactions.
\end{itemize}

There are two functions which transform (optional) compact trees using transactions.
\begin{itemize}
\item {\func{tx\_octree\_trans}} takes a block height, transaction and compact tree
and transforms the ctree by deleting assets consumed in the inputs and
create the new assets in the output. (The block height is needed to give
birthdays to the new assets.)
\item {\func{txl\_octree\_trans}} transforms a compact tree according to a
list of transactions, sequentially.
\end{itemize}

There are also four auxiliary functions exposed in the interface.
\begin{itemize}
\item {\func{strip\_bitseq\_true}} takes a list of pairs, the first component of which are bit sequences, and returns the list
filtered to the ones with a {\val{true}} as the head of the list with this {\val{true}} removed.
For example, the input
$$[((\val{false}::\overline{b_0}),x);((\val{true}::\overline{b_1}),y)]$$
would give the output
$$[(\overline{b_1},y)].$$
\item {\func{strip\_bitseq\_false}} takes a list of pairs, the first component of which are bit sequences, and returns the list
filtered to the ones with a {\val{false}} as the head of the list with this {\val{false}} removed.
For example, the input
$$[((\val{false}::\overline{b_0}),x);((\val{true}::\overline{b_1}),y)]$$
would give the output
$$[(\overline{b_0},x)].$$
\item {\func{strip\_bitseq\_true0}} takes a list of bit sequences, and returns the list
filtered to the ones with a {\val{true}} as the head of the list with this {\val{true}} removed.
\item {\func{strip\_bitseq\_false0}} takes a list of bit sequences, and returns the list
filtered to the ones with a {\val{false}} as the head of the list with this {\val{false}} removed.
\end{itemize}
These are exposed because they are used in the {\module{ctregraft}} module.\footnote{It might make more sense to combine the {\module{ctre}} and {\module{ctregraft}} modules so that these functions need not be exposed.}

\section{Grafting Trees}

The module {\module{ctregraft}}
has code for grafting subtrees onto a compact tree in order
to form an approximation with more information.
The purpose of this is so that a block header can
have a compact tree small enough to check the details of the asset which
staked the block,
and the block delta can have a graft extending this compact tree
to a larger compact tree with enough information to support
all the transactions in the block.

{\bf{Note:}} 
In the Coq formalization the Coq module {\coqmod{CTreeGrafting}}
corresponds to {\module{ctregrafting}}.
For more information, see~\cite{White2015b}.

The type {\type{cgraft}} is a list of hash values associated with compact trees.
The idea is simply to associate some hash roots with compact trees with these hash roots.
As usual, the serialization and deserialization functions
for this type are
{\serfunc{seo\_cgraft}}
and
{\serfunc{sei\_cgraft}}.

There are four functions exposed by {\module{ctregraft}}.
\begin{itemize}
\item {\func{cgraft\_valid}} checks if a graft is {\defin{valid}}, meaning simply that
each pair $(h,\cC)$ is such that the hash root of $\cC$ is $h$.
\item {\func{ctree\_cgraft}} takes a graft $\cG$ and a compact tree $\cC$
and replaces each ${\constr{CHash}}(h)$ in $\cC$ with $\cC'$
where $(h,\cC')$ is in $\cG$.
\item {\func{factor\_tx\_ctree\_cgraft}} takes a transaction and a compact tree $\cC$
and computes a pair $(\cC',\cG)$ of a compact tree $\cC'$ and a graft $\cG$.\footnote{This function is currently unused.}
Here $\cC'$ is an approximation of $\cC$
and {\func{ctree\_cgraft}} applied to $\cG$ and $\cC'$ yields $\cC$.
\item {\func{factor\_inputs\_ctree\_cgraft}} takes an {\type{addr\_assetid}}
(a list of transaction inputs)
and a compact tree $\cC$
and computes a pair $(\cC',\cG)$ of a compact tree $\cC'$ and a graft $\cG$.
Here $\cC'$ is an approximation of $\cC$
and {\func{ctree\_cgraft}} applied to $\cG$ and $\cC'$ yields $\cC$.
The purpose of this function is to factor the part of the compact tree
needed for the block header from the part needed for the rest of the block.
\end{itemize}


\chapter{Blocks and Block Chains}\label{chap:block}

The module {\module{block}}
contains code related to blocks and block chains.
This includes code to check if a block header is valid
(including verifying the properties of the staking asset in the ledger),
whether a block is valid
and if a block or block header is a valid successor to a block or block header.
In order to verify these properties we need to know when an asset
is allowed to stake a block.
We also allow for the possibility of forfeiture of block rewards as
a punishment for signing on two different short forks.

{\bf{Note:}} Unit tests for the {\module{block}} module have not been written.

{\bf{Note:}} 
In the Coq formalization the Coq module {\coqmod{Blocks}}
corresponds to {\module{block}}.
For more information, see~\cite{White2015b}.

\section{Stake Modifiers}

A {\defin{stake modifier}} is a 256 bit number.
The type {\type{stakemod}} is defined as four 64-bit integers
as a way of representing such a 256 bit number.
The functions {\serfunc{seo\_stakemod}} and {\serfunc{sei\_stakemod}}
serialize and deserialize stake modifiers.

At each block height there will be a current stake modifier and a future stake modifier.
The current stake modifier determines who will be able to stake the next block.
The future stake modifier influences the next 256 current stake modifiers.

The genesis current and future stake modifiers should be set in the variables
{\var{genesiscurrentstakemod}} and
{\var{genesisfuturestakemod}}.
These will determine who will be able to stake the first $256$ blocks
and will influence who will be able to stake the next $256$ blocks,
so it is important that these genesis stake modifiers are chosen in a fair manner.
The function {\func{set\_genesis\_stakemods}}
sets {\var{genesiscurrentstakemod}}
and {\var{genesisfuturestakemod}}
by taking a 160-bit number (as a 40 character hex string),
applying one round of {\tt{SHA256}} to obtain the value for
{\var{genesiscurrentstakemod}}
and another round of {\tt{SHA256}} to obtain the value for
{\var{genesisfuturestakemod}}.\footnote{The plan was to choose some Bitcoin height in the future and when that height was reached to obtain the 160-bit seed number from the last 20 bytes of the hash of the Bitcoin block header at that height.}

The following three functions operate on stake modifiers.
\begin{itemize}
\item {\func{stakemod\_pushbit}} takes a bit (as a boolean) and a stake modifier, shifts the 256-bit stake modifier (dropping the most significant bit) and using the new bit as the new least significant bit.
\item {\func{stakemod\_lastbit}} takes a stake modifier and returns its most significant bit (as a boolean).
\item {\func{stakemod\_firstbit}} takes a stake modifier and returns its least significant bit (as a boolean).
\end{itemize}
The current stake modifier changes from one block height to the next by
taking the last bit of the future stake modifier and pushing this bit onto the current stake modifier.
The future stake modifier changes from one block height to the next by
pushing a new bit (either $0$ or $1$) onto the current future stake modifier.
This implies those who stake blocks influence what will be the current stake modifiers,
but this influence is limited. If one staker staked 50\% of blocks, the staker would
choose approximately $128$ bits of the $256$ stake modifiers in the future.
The hope is that this influence is not enough to significantly improve their
chances in the future, as each bit not chosen by the staker also has a large influence
on who will be able to stake.

The function {\func{hitval}} performs one round of
{\tt{SHA256}} on the least significant 32-bits of a 64-bit integer (intended to be the current time),
a hash value (intended to be the asset id of the asset to stake) and a stake modifier (intended to be the current stake modifier).
It returns the result as 256-bit number called the {\defin{hit value}}.

\section{Targets}

A value of type {\type{targetinfo}} is a triple
consisting of the current stake modifier,
the future stake modifier
and the current target (represented by a {\type{big\_int}}).
The target info used to determine if an asset
is allowed to stake the next block.
In particular, an asset can stake
the hit value is less than the current target times the coinage of the staked asset
(or, the coinage times $1.25$
if proof of storage is used).

We have described above how the current and future stake modifiers
change at each block height.
The current target should also change in order to target an average $10$ minute
block. The function {\func{retarget}} defines how the target changes
after each block.

\begin{itemize}
\item {\var{genesistarget}} is set to the initial target used for the genesis block.\footnote{It is currently set to $2^{205}$, but this should be reevaluated after a test run and before the launch of Qeditas.}
\item {\var{max\_target}} is set to the maximum value for the target (i.e., the minimum difficulty). It
is currently set to $2^{220}$.
\item {\func{retarget}} takes a target $\tau$ and a number of seconds $\Delta$ and returns a new target.
The intention is that the given target is the current target and the number of seconds
is the number of seconds between the current block and the previous block.
The value is the minimum of either {\var{max\_target}} or
$\frac{\tau (9000 + \Delta)}{9600}$.
In particular, the value returned is never more than the value of {\var{max\_target}}
and remains $\tau$ if $\Delta$ is 600.
\end{itemize}

\section{Proof of Storage}

The consensus system for Qeditas is primarily proof-of-stake,
but also includes a proof-of-storage component.
A node can use evidence that it is storing some
part of a term or document
to increase the weight of its stake by $25\%$.
The evidence is a value of type {\type{postor}}, defined by two constructors:
\begin{itemize}
\item ${\constr{PostorTrm}}(h,s,\alpha,k)$ is evidence of storage of part of a term of a type in a theory at a term address.
The optional hash value $h$ identifies a theory,
$s$ is a term, $\alpha$ is a type
and $k$ is a hash value.
Here $s$ should have type $\alpha$ in the theory identified by $h$.
The way this typing constraint is ensured is by checking that the term
address correspond to the object $s$ in theory $h$
has an owner as an object.
This ownership asset should have assed id $k$.
The term $s$ is intended to be minimal:
all except exactly one left of the tree representing $s$ should be 
an abbreviation (i.e., {\constr{TmH}} of hash roots).
(This minimality condition is checked by {\func{check\_postor\_tm\_r}}.)
\item ${\constr{PostorDoc}}(\gamma,\nu,h,\Delta,k)$ is evidence of storage of part of a document at a publication address.
Here $\gamma$ is a pay address, $\nu$ is a hash value (nonce), $h$ is an optional hash value (identifying a theory), $\Delta$ is a partial document (of type {\type{pdoc}})
and $h$ is a hash value.
The intention is that $h$ is the asset id for an asset with preasset ${\constr{DocPublication}}(\gamma,\nu,h,\Delta')$
held and the publication address determined by hashing $\gamma$, $\nu$, $h$ and $\Delta'$.
Here $\Delta'$ is a document with the same hash root as the partial document $\Delta$.
The partial document $\Delta$ should be minimal:
with exactly one document item containing more than hashes
and with that one document item only containing one explicit leaf,
with others abbreviated by hash roots.
(This minimality condition is checked by {\func{check\_postor\_pdoc\_r}}.)
\end{itemize}
Values of type {\type{postor}}
can be serialized and deserialized using
{\serfunc{seo\_postor}} and
{\serfunc{sei\_postor}}.

The exception {\exc{InappropriatePostor}} is raised if a value of type {\type{postor}}
is not an appropriate proof of storage
because the term or partial document is not minimal.
\begin{itemize}
\item {\func{incrstake}} multiplies the number of cants being staked by $1.25$.
This is the adjusted stake used when proof of storage is included.
\item {\func{check\_postor\_tm\_r}} checks the minimality condition for a term,
returning the hash of the unique important leaf upon success.
\item {\func{check\_postor\_tm}} checks if
${\constr{PostorTrm}}(h,s,\gamma,k)$
can be used to increase the chances of staking.
Let $\alpha$ be the (p2pkh) address where the asset to be staked is held.
Let $\beta$ be the term address for the object $s$ of type $\gamma$ in the theory identified by $h$.
Let $h'$ be the hash of the unique exposed leaf given by {\func{check\_postor\_tm\_r}}.
Let $h''$ be the result of hashing the pair of $\beta$ and $h'$.
Let $h'''$ be the result of hashing $\alpha$ with $h''$.
There are two conditions:
\begin{enumerate}
\item A certain 16 bits of $h'''$ are all $0$. (This means that given a stake address $\alpha$,
only one in every 65536 items of the form
${\constr{PostorTrm}}(h,s,\gamma,k)$
can possibly ever be used to help $\alpha$ stake, independent of targets and stake modifiers.
\item The hit value of $h''$ is less than the target times the adjusted stake.
\end{enumerate}
\item {\func{check\_postor\_pdoc\_r}} checks the minimality condition for a partial document,
returning the hash of the unique important leaf upon success.
\item {\func{check\_postor\_pdoc}} checks if
${\constr{PostorDoc}}(\gamma,\nu,h,\Delta,k)$
can be used to increase the chances of staking.
Let $\alpha$ be the (p2pkh) address where the asset to be staked is held.
Let $\beta$ be the publication address for the corresponding document asset.
Let $h'$ be the hash of the unique exposed leaf given by {\func{check\_postor\_pdoc\_r}}.
Let $h''$ be the result of hashing the pair of $\beta$ and $h'$.
Let $h'''$ be the result of hashing $\alpha$ with $h''$.
\begin{enumerate}
\item A certain 16 bits of $h'''$ are all $0$. (This means that given a stake address $\alpha$,
only one in every 65536 items of the form
${\constr{PostorDoc}}(\gamma,\nu,h,\Delta,k)$
can possibly ever be used to help $\alpha$ stake, independent of targets and stake modifiers.
\item The hit value of $h''$ is less than the target times the adjusted stake.
\end{enumerate}
\end{itemize}

\section{Hits and Cumulative Stake}

We now describe two functions for checking if an asset
(optionally with proof of storage) is allowed to stake.
This is sometimes informally referred to as ``checking for a hit.''
A third function {\func{check\_hit}} is deferred until
we discuss block headers.

\begin{itemize}
\item {\func{check\_hit\_b}} is an auxiliary function which does most of the work
to check if an currency asset can stake a block.
It is given the block height, the birthday of the asset, the obligation of the asset,
the number of cants $v$ in the currency asset, the current stake modifier, the current target,
the current timestamp, the asset id of the asset to stake, the p2pkh address holding the stake address\footnote{Note that the obligation of the stake address may mean that a different person can spend the staking asset than the holder who can stake the asset. This could be used to, for example, ``loan'' assets to someone else to stake.}
and an optional proof of storage.
If no proof of storage is given,
the asset can stake if its hit value (relative to the time stamp and current stake modifier)
is less than the product of the target and the coinage (as computed by {\func{coinage}}) of the asset.
Suppose a proof of storage is given.
In this case, we consider an adjusted stake using $1.25 v$ instead of $v$.
The asset can stake if the hit value of the asset is less than the target times the coinage of the adjusted stake
and the proof of storage can be used (as judged by {\func{check\_postor\_tm}} or {\func{check\_postor\_pdoc}}).
\item {\func{check\_hit\_a}} is simply a wrapper function which takes the target info (of type {\type{targetinfo}})
and calls {\func{check\_hit\_b}} after extracting the current stake modifier and current target
from the target info. Factoring the functions this way makes it clear that
{\func{check\_hit\_b}} does not depend on the future stake modifier.
\end{itemize}

The best block chain will be the one with the most cumulative stake.\footnote{The intention is also to have rolling checkpoints to prevent long range attacks.}
The cumulative stake is represented by a {\type{big\_int}}.
The function {\func{cumul\_stake}} computes the new cumulative stake
given the previous cumulative stake, the current target $\tau$
and the latest delta time (time between blocks) $\Delta$.
It computes this by adding the following (big integer) value to the previous cumulative stake:
$$\lfloor \frac{\var{max\_target}}{\tau \Delta 2^{-20}} \rfloor$$
or adding $1$ if this value is less than $1$.

\section{Block Headers}

We now describe block headers.
A block header is made up of two sets of information:
the header data and the header signature.
The data part is represented using the
record type 
{\type{blockheaderdata}}
while the signature part is represented using the record type
{\type{blockheadersig}}.
A block header (of type {\type{blockheader}})
is simply a pair of the data with the signature.
The functions
{\serfunc{seo\_blockheader}} and
{\serfunc{sei\_blockheader}} serialize and deserialize block headers.
There is a value
{\var{fake\_blockheader}}
which can be used when some data structure needs a block header to be initialized.

The fields in the record type {\type{blockheaderdata}} are as follows:
\begin{itemize}
\item {\field{prevblockhash}} should contain the hash of the data in the previous block header (or {\val{None}} for the genesis block header).
\item {\field{newtheoryroot}} should be the hash root of the current theory tree (optional {\type{ttree}}) after the block with this header has been processed.
It will change if some transaction in the block publishes a theory specification.
\item {\field{newsignaroot}} should be the hash root of the current signature tree (optional {\type{stree}}) after the block with this header has been processed.
It will change if some transaction in the block publishes a signature specification.
\item {\field{newledgerroot}} should be the hash root of the current compact tree ({\type{ctree}})
after the block with this header has been processed.
This will always change since the asset staked will be spent and there will be
outputs to the coinstake transaction of the block.
\item {\field{stakeaddr}} should be the p2pkh address where the asset being staked is held.
\item {\field{stakeassetid}} should be the asset id of the asset being staked.
\item {\field{stored}} is an optional proof of storage ({\type{postor}})
and will be {\var{None}} if proof of storage was not used to help stake this block.
\item {\field{timestamp}} is a 64-bit integer time stamp and should correspond to the time the block was staked.
\item {\field{deltatime}} is a 32-bit integer which should contain the difference between the time stamp of this block and the time stamp of the previous block. (For the genesis block header, this should simply be $600$.)
\item {\field{tinfo}} should be the target information (current stake modifier, future stake modifier and current target) for this block header.
\item {\field{prevledger}} is an approximation of the compact tree before processing the block corresponding to this block header.
This approximation must contain the asset being staked and, if proof of storage is included,
the relevant object ownership asset
or document asset.
\end{itemize}

The fields in the record type {\type{blockheadersig}} are as follows:
\begin{itemize}
\item {\field{blocksignat}} is a cryptographic signature of type {\type{signat}}.
This should be a signature of a hash of the data in the block header.
Unless an endorsement is used, the signature should be by the private key
corresponding to the stake address.
If an endorsement is used, the signature should be by the private key
\item {\field{blocksignatrecid}} is an integer which should be between $0$ and $3$.
It is included to help recover the public key for the address (either stake or endorsed) from the signature
(see the function {\file{recover\_key}} in the module {\module{signat}}).
\item {\field{blocksignatfcomp}} is a boolean indicating if the address (either stake or endorsed) corresponds
to the compressed or uncompressed public key.
\item {\field{blocksignatendorsement}} is an optional endorsement.
If {\val{None}}, then signature corresponds to the stake address.
Suppose it is $(\beta,r,b,\sigma)$ where $\beta$ is p2pkh address (the endorsed address), $r$ is an integer ($0\leq r\leq 3$),
$b$ is a boolean and $\sigma$ is a cryptographic signature.
Here $\sigma$ should be a signature of the Bitcoin message
``\verb+endorse+ $\beta$''
where $\beta$ is the endorsed address (as a Qeditas address in base58 format).
The signature $\sigma$ should be by the private key corresponding to the address $\alpha$
and $r$ and $b$ are used to recover the public key.
\end{itemize}

The following functions operate on block headers:
\begin{itemize}
\item {\func{blockheader\_stakeasset}} takes block header data ({\type{blockheaderdata}})
and tries to return the staked asset by looking it up 
as {\field{stakeid}} at location {\field{stakeaddr}} in the compact tree {\field{prevledger}}.
This can fail in two ways.
First, it could be that the staked asset is not found, in which case an exception {\exc{HeaderNoStakedAsset}} is raised.
Second, it could be that {\field{prevledger}} includes more information than is necessary to give the staked asset,
in which case an exception {\exc{HeaderStakedAssetNotMin}} is raised.\footnote{The purpose of this condition is to prevent attackers from making unnecessarily large headers. The current implementation seems to be flawed, however, as it would not allow the relevant information from proof-of-storage to be included in {\field{prevledger}}.}
\item {\func{hash\_blockheaderdata}} hashes the data in the block header. This is to determine
the hash to be signed in the signature part
as well as the hash to be used in the {\field{previousblockhash}} field of the next block header.
\item {\func{check\_hit}} takes block header data ({\type{blockheaderdata}})
and checks if the given staked asset is allowed to create the block.
It simply calles {\func{check\_hit\_a}} after
extracting the target info ({\field{tinfo}}), time stamp ({\field{timestamp}}),
stake asset id ({\field{stakeassetid}}),
address where the staked asset is held ({\field{stakeaddr}})
and the optional proof of storage ({\field{stored}})
from given block header data.
\item {\func{valid\_blockheader}} determines if a block header is a valid block at the current height.
In order to check if the block is valid the staked asset must be retrieved from
the previous ledger.
The staked asset must be a currency asset worth $v$ cants.
The auxiliary function {\func{valid\_blockheader\_a}} is called with the extra information given by this asset
which in turn calls two (exported) functions:
{\func{valid\_blockheader\_signat}} and
{\func{valid\_blockheader\_allbutsignat}}.
{\func{valid\_blockheader\_signat}} verifies the signature 
in the blockheader to be a valid signature (either directly or via endorsement) of the hash given by {\func{hash\_blockheaderdata}}.
{\func{valid\_blockheader\_allbutsignat}} checks the following conditions:
\begin{enumerate}
\item The staked asset has the asset id declared in the header.
\item The delta time is greater than $0$.
\item The staked asset is a ``hit'' for the current block height.
\item If proof of storage is included, then the asset id given for the
object ownership of the term or
for the document
is in the given approximation of the previous ledger.\footnote{This probably no longer works if proof of storage is included, due to the minimality constraint on {\field{prevledger}}.}
\end{enumerate}
\item {\func{blockheader\_succ}} determines if a second block header is a valid successor to a first block header.
The following conditions must be checked:
\begin{enumerate}
\item The second {\field{prevblockhash}} is the hash of the data in the first given block header.
\item The second {\field{timestamp}} is the sum of the first {\field{timestamp}} and the second {\field{deltatime}}.
\item The current stake modifier given in the second {\field{tinfo}}
 is the result of pushing the last bit of the future stake modifier of the first {\field{tinfo}}
 onto the current stake modifier of the first {\field{tinfo}}.
\item The future stake modifier given in the second {\field{tinfo}}
      is the result of pushing a $0$ or a $1$ onto the future stake modifier of the first {\field{tinfo}}.
\item The target given in the second {\field{tinfo}} is the result of retargeting using
      the target given in the first {\field{tinfo}}
      and the first {\field{deltatime}}.
\end{enumerate}
\end{itemize}

\section{Proof of Forfeiture}

Proof of forfeiture is
optional data proving a staker signed on two recent chain forks within 6 blocks.
When such a proof is supplied by a staker of a block, the new staker can
take recent coinstake rewards from the double signing staker.
Such a proof is a value of type {\type{poforfeit}}
and consists of a 6-tuple
$$(b_1,b_2,\overline{c_1},\overline{c_2},d,\overline{h}).$$
Here $b_1$ and $b_2$ are block headers which should contain different data but both be signed by the
same stake address.
The values $\overline{c_1}$ and $\overline{c_2}$ are lists of block header data
each of which should have length at most 5.
Finally, $v$ is the number of cants being forfeited
and $\overline{h}$ is a list of hash values (asset ids of the rewards being forfeited).

The function {\func{check\_poforfeit}} verifies if the given value of type {\type{poforfeit}}
can be used to support forfeiture of rewards. It first verifies that the data in $b_1$ and $b_2$
are different (by ensuring their hashes are different)
and
are staked using assets at the same stake address $\alpha$.
It also verifies that $\overline{c_1}$
and $\overline{c_2}$ have no more than 5 elements.
It then verifies the signatures for $b_1$ and $b_2$.
It calls {\func{check\_bhl}} on $\overline{c_1}$ and $\overline{c_2}$
to ensure that each forms a (reverse) chain connecting $b_1$ and $b_2$
to some previous block hashes $k_1$ and $k_2$,
and then checks that $k_1 = k_2$. This implies $b_1$ and $b_2$ are signed block headers
forking from a common block (with hash $k_1$). (The function {\func{check\_bhl}} also ensures
that the hash of $b_2$ does not occur in $\overline{c_1}$ as this would mean
the second chain is a subchain for the first, rather than a fork. Likewise it ensures
the hash of $b_1$ does not occur in $\overline{c_2}$.)
Finally it calls {\func{check\_poforfeit\_a}}
which looks up assets by the asset ids listed in $\overline{h}$
and verifies that each is a reward less than 6 blocks old
which was paid to address $\alpha$
and that the sum of these rewards is $v$ cants.

\section{Blocks}

A {\defin{block}} consists of a block header and a block delta.
The block delta (implemented as the record type {\type{blockdelta}})
contains information about how to transform the previous ledger (compact tree)
into the next ledger (compact tree).
In particular, the stake output is given (which completes the coinstake transaction)
and all other transactions in the block are given.
In addition, an optional proof of forfeiture is given which may effectively increase
the rewards given to the staker of the block.
In order to transform the previous ledger,
one will generally need to graft more information about the previous
ledger than was given in the header.
This graft is also given.

The {\type{blockdelta}} record type consists of four fields:
\begin{itemize}
\item {\field{stakeoutput}} is the output to the coinstake transaction.
\item {\field{forfeiture}} is an optional proof that a recent staker signed on a recent fork, thus
justifying forfeiture of that staker's recent rewards.
\item {\field{prevledgergraft}} is a graft providing the extra information needed by the output of the coinstake transaction, the other transactions in the block and optionally the data in the {\field{forfeiture}} field.
\item {\field{blockdelta\_stxl}} is a list of signed transactions, the transactions in the block.
\end{itemize}
The functions {\serfunc{seo\_blockdelta}} and {\serfunc{sei\_blockdelta}} serialize
and deserialize block deltas.

The type {\type{block}} is the product of {\type{blockheader}} and {\type{blockdelta}}.
The functions {\serfunc{seo\_block}} and {\serfunc{sei\_block}} serialize
and deserialize blocks.

\begin{itemize}
\item {\func{coinstake}} builds the coinstake transaction by
using the staked asset possibly combined with forfeited rewards as the input
and taking {\field{stakeoutput}} from the block delta for the output.
\item {\func{ctree\_of\_block}} returns the compact tree of a block (approximating the ledger state before
processing the block) by taking {\field{prevledger}} from the block header data
and grafting on {\field{prevledgergraft}} from the block delta.
We call this the {\defin{compact tree of a block}}.
\item {\func{tx\_of\_block}} combines all the transactions in the block (including the coinstake) into one large transaction combining all the inputs and all the outputs.
This is used to check validity of blocks.
\item {\func{txl\_of\_block}} returns a list of all (unsigned) transactions in the block,
including the coinstake transaction and the underlying transactions listed in {\field{blockdelta\_stxl}} of the block delta.
\item {\func{rewfn}} returns the number of cants of the reward at the current block height.
The reward schedule is the same as Bitcoins (except for the amount of precision), except with the assumption that the first 350000 blocks have already passed (since this was the block height for the snapshot).
We begin counting with a block height of $1$. From blocks $1$ to $70000$,
the block reward is $25$ fraenks (2.5 trillion cants).
After this the reward halves every $210000$ blocks.
Since the initial distribution contained (slightly less than) 14 million fraenks, this leads to cap of 21 million fraenks.
\item {\func{valid\_block}} checks if a block is valid at the given height.
It does this by looking up the staked asset and passing the information to {\func{valid\_block\_a}}
which checks the following conditions:
\begin{enumerate}
\item The header must be valid.
\item The transaction outputs in {\field{stakeoutput}} must be valid (as judged by {\func{tx\_outputs\_valid}}).
\item If the staked asset has an explicit obligation, then ensure the first output on {\field{stakeoutput}}
is of a preasset with the same amount of cants and the same obligation sent to the stake address.\footnote{This is to support ``loaning'' assets for staking.}
\item All outputs in {\field{stakeoutput}} except possibly the first is explicitly must be marked as a reward
and
have a lock in the obligation at least as long as the value given by {\func{reward\_locktime}}.
Furthermore, all the outputs must be sent to the stake address.
If the first output in {\field{stakeoutput}} is not marked as a reward, then it must also
be sent to the stake address, must be a Currency asset with the same number of cants as the staked asset
and must have the same obligation (possibly the default {\val{None}} obligation) as the staked asset.
\item The compact tree of the block must support the coinstake transaction
and it must have a reward at least\footnote{This is to allow for collection of fees and of forfeiture of recent awards. The fact that the output is not too high is guaranteed later.}
as high as the value given by {\func{rewfn}}.
\item There are no duplicate transactions listed in {\field{blockdelta\_stxl}}.
\item The graft in {\field{prevledgergraft}} is valid.
\item Each transaction in {\field{blockdelta\_stxl}} has valid signatures, is valid and is supported by the compact tree of the block. Furthermore, none of these outputs are marked as rewards, none of these transactions spend the asset being staked. Finally, each transaction consumes at least as many cants as required.
\item No two transactions in {\field{blockdelta\_stxl}} spend the same input.
\item No two transactions in {\field{blockdelta\_stxl}} create ownership as an object (resp., as a proposition) at the same term address.
\item If a transaction in {\field{blockdelta\_stxl}} creates ownership as an object (resp., as a proposition)
at a term address, then the output of the coinstake transaction does not create the same kind of ownership at the term address.\footnote{It would make sense to simply disallow creation of non-currency assets in the coinstake transaction, but this is not currently the case.}
\item If proof of forfeiture is given, then check it is valid and remember the number of cants being forfeited.
\item Let $\cC$ be the result of transforming the compact tree of the block ({\func{ctree\_of\_block}})
using the transactions of the block ({\func{txl\_of\_block}}).
The hash root of $\cC$ must be {\field{newledgerroot}}.
\item Let $\tau=(\iota,o)$ be the transaction of the block. The following must hold:
\begin{itemize}
\item The cost of the outputs of $\tau$ (see {\func{out\_cost}})
is equal to the sum of the assets being spent
along with the reward ({\func{rewfn}})
and (possibly) the number of cants being forfeited.
\item The transformation of the current theory tree by $o$ must have hash root $\field{newtheoryroot}$.
\item The transformation of the current signature tree by $o$ must have hash root $\field{newsignatroot}$.
\end{itemize}
\end{enumerate}
\end{itemize}

\section{Databases for Block Information}

There are three databases for blocks, all using the hash of the block header as the key.
The module {\module{DbBlockHeader}} is a database for block headers (implemented using {\module{Dbbasic2keyiter}})
and
the module {\module{DbBlockDelta}} is a database for block deltas (implemented using {\module{Dbbasic2}}).

\section{Chains}

There are additional types
{\type{blockchain}}
and 
{\type{blockheaderchain}}.
These can be used to represent (nonempty) chains of blocks or block headers.\footnote{It is not clear if this is explicitly needed.}
In each case the representation is as a pair
where the first component should be the most recent block of block header
and the second component is a list of the previous blocks or block headers
in reverse order.

The variable {\var{genesisledgerroot}} gives the ledger root of the initial compact tree
with the initial distribution. The value is (as of September 2016):
\begin{verbatim}
fc25150b4880e27235d4878637d32f0ffe2280e6
\end{verbatim}

\begin{itemize}
\item {\func{blockchain\_headers}} converts a block chain into block header chain by dropping the block deltas.
\item {\func{ledgerroot\_of\_blockchain}} takes a block chain and returns the value of {\field{newledgerroot}} in the latest block header data.
\item {\func{valid\_blockchain}} checks if a block chain is valid at a given height.
This requires checking the validity of each block and that each block header is a valid
successor to the previous block header. It also requires keeping up with the
theory tree and signature tree.
In the case of the genesis block, the {\field{prevblockhash}} should be {\val{None}},
the {\field{prevledger}} should have hash root {\var{genesisledgerroot}},
the {\field{tinfo}} should be composed of the values in {\var{genesisccurrentstakemod}},
{\var{genesisfuturestakemod}} and {\var{genesistarget}}
and the {\field{deltatime}} should be $600$.\footnote{Alternatively, one could set a ``genesis timestamp'' and enforce that the {\field{deltatime}} of the genesis block is the difference between the time stamp of the genesis block and the fixed genesis timestamp.}
\item {\func{valid\_blockheaderchain}} checks the validity of a block header chain.
It is similar to {\func{valid\_blockchain}} but only checks the headers are valid
instead of the full blocks.
\end{itemize}


\chapter{Networking}\label{chap:net}

{\bf{Warning:}} The networking code is only partially written and may possibly be rewritten completely.
The information is this chapter is subject to changes in the near future.

The intention of the {\module{net}} module is to create and handle connections
to remote nodes.
We briefly describe what exists, but the code is unusable in its current form.
The networking code should be rethought, redesigned and rewritten from scratch.

\section{State}

There are a number of variables to keep up with the state of the system.
It is not clear these variables belong in this module.
\begin{itemize}
\item {\var{recentblockheaders}} is a list of recent block headers sorted according to cumulative stake.
This ensures the head of the list is the recent block header with the most cumulative stake.
\item {\var{recentledgerroots}} associates ledger roots (hash values) with abbreviations giving where the corresponding {\type{ctre}} has been saved. This is needed to restore the ledger tree into memory when it is needed.
\item {\var{recentcommitments}} is intended to store what commitments have been made by staking blocks to avoid double signing (signing blocks on two branches stemming from a recent fork).
\item {\var{recentblockdeltahs}} lists a skeleton form of recent block deltas, where the transactions in the block delta are refered to by their hash.
\item {\var{recentblockdeltas}} lists recent block deltas.
\item {\var{recentstxs}} lists recent transactions which may be included in a block.
\item {\var{waitingblock}} lists block headers which are not considered ``valid,'' but may be valid later (e.g., the time stamp is in the future).
\item {\var{preconns}} lists connections in the handshake phase before other messages can be exchanged.
\item {\var{conns}} lists active connections.
\item {\var{this\_nodes\_nonce}} is a random nonce to prevent nodes from connecting to themselves.
\end{itemize}

The function {\func{insertnewblockheader}} inserts a new block header into the sorted
list {\var{recentblockheaders}}.
This makes the ``valid'' block header with the most cumulative stake at the head of the
list. The intention is that blocks should be staked on top of this ``best'' block.

\section{Making Connections}

The function {\func{myaddr}} returns the address to optionally listen for connections on.
This must be specified in the configuration or the node will not listen for connections.

The function {\func{openlistener}} starts a listener socket to wait for incoming connections.

The function {\func{accept\_nohang}} checks if a node is attempting to make a connection
without waiting for a connection (i.e., with a timeout).

The exception {\exc{EnoughConnections}} is raised if a connection is being rejected
because there are enough connections already.
The exception {\exc{RequestRejected}} may be raised if the connection fails, e.g.,
if the SOCKS4 protocol indicates a rejection.

The functions {\func{connectpeer}} and {\func{connectpeer\_socks4}}
try to initiate a connection to a listening peer.
The function {\func{tryconnectpeer}} searches for a known peer to try to connect to.
The function {\func{getfallbacknodes}} lists fallback nodes to try if no other connection works.

The function {\func{initialize\_conn\_accept}} is called after the listener finds
a node trying to connect.
Both put the initial connection onto {\var{preconns}}
and attempt to perform the initial handshake. Only after the initial handshake
succeeds does the connection become a legitimate connection listed
in {\var{conns}}.

The functions
{\func{addknownpeer}},
{\func{getknownpeers}},
{\func{loadknownpeers}} and
{\func{saveknownpeers}}
are to keep up with information about peers in the recent past.

The function {\func{extract\_ip\_and\_port}} extracts an ip-address and a port number
from a string, with a boolean to indicate if the address is ipv6.

The function {\func{sethungsignalhandler}} sets the timeout handler to raise the exception {\exc{Hung}}
upon timeout. It is not clear why this should be exported.

\section{Messages and Communication}

Messages are elements of type {\type{msg}}.
The kinds of messages largely follow Bitcoin's messages, except there
are additional messages:
\begin{itemize}
\item ${\constr{MCTree}}(v,t)$ is the compact tree $t$ (of type {\type{ctre}}).
The 32-bit integer $v$ is a ``version number'' which is only included in case it's needed later.
\item ${\constr{Checkpoint}}(b,h)$ a ``checkpoint'' that the block at height $b$ has hash $h$.
\item ${\constr{AntiCheckpoint}}(b,h)$ an ``anticheckpoint'' that the block at height $b$ does not have hash $h$.
\end{itemize}

The exception {\exc{IllformedMsg}} is raised if something is wrong with the format of a message.

The function {\func{send\_msg}} sends a message through a channel to another node.
The function {\func{send\_reply}} sends a message through a channel to another node
and marks it as a reply to a previous message.

The function {\func{input\_byte\_nohang}} listens for a byte with a timeout (returning
{\val{None}} if no byte is available to be read).
The function {\func{rec\_msg\_nohang}} tried to receive a message with two timeouts.
The first timeout indicates how long to wait before a message begins
and the second timeout indicates how long to try to read a message after
it begins but until it ends.

The recursive {\type{pendingcallback}} type gives a way of specifying
what should happen if a reply is received.

The funciton {\func{broadcast\_inv}} broadcasts an inventory message to every connection.

The function {\func{handle\_msg}} is intended to handle messages received,
but most cases are unwritten.

The function {\func{try\_requests}} searches through the current connections
to try to find one from which it can request some data.

The functions {\func{handle\_orphans}} and {\func{handle\_delayed}}
are intended to look back at orphan block headers and block headers which
were delayed for some reason (e.g., if the time stamp was too far in the future).
These were written because in testing sometimes a branch with the most cumulative stake
would not be communicated because an earlier block header had been delayed or rejected.

% {\type{msg}}

% internal serialization and deserialization functions {\serfunc{seo\_msg}} and {\serfunc{sei\_msg}}
The function {\func{send\_initial\_inv}} is called when a connection is established (after the handshake)
to send the inventory.

\section{Connection State}

The type {\type{connstate}} consists of a number of mutable fields
which represent the state of a connection.
This includes whether the node is alive,
when the last message was received,
what requests are pending,
the known inventory of the remote node
and what inventory has been sent, what inventory has been requested.








\chapter{Staking Code}\label{chap:stk}

The file {\file{qeditasstk.ml}} has code to execute a staking process.
It creates an executable which is started by {\module{qeditasd}}
and runs in an independent process. It is sent the block height, target information
and stake modifier through standard input. It is also sent a list of assets
which can be used for staking
and storage assets which can be used to enhance staking.
The process then keeps checking to see if one of the
assets leads to a ``hit.'' When it finds a ``hit'' it reports this to
the parent via standard out.


\chapter{Top Level Code}\label{chap:top}

{\bf{Warning:}} The top level code is only partially written and may possibly be rewritten completely.
The information is this chapter is subject to changes in the near future.

The files
{\file{qeditascli.ml}} and {\file{qeditasd.ml}}
are used to create executables
{\exec{qeditasd}} and {\exec{qeditascli}}.
These executables are intended to provide a basic user interface.
In particular {\exec{qeditasd}} starts a ``daemon''
which connects to other nodes, interacts with other nodes,
handles a staking process (if staking is enabled)
and tries to publish transactions and blocks.
The executable {\exec{qeditascli}} provides a command line interface
which may (or may not) interact with {\exec{qeditasd}}.
The module {\module{commands}} handles commands and wallet information
that is used by both {\file{qeditascli.ml}} and {\file{qeditasd.ml}}.

\section{Files Instead of Sockets}

Originally the
two executables {\exec{qeditasd}} and {\exec{qeditascli}}
were intended to mimic the Bitcoin daemon and command line interface.
However, one of the goals was to have Qeditas run on the Tails Live CD (Debian) operating system.
It seems that (apparently) Tails does not allow local sockets to be opened.
As a consequence, files were used to handle some interaction between {\exec{qeditascli}}
and {\exec{qeditasd}}. For example, {\exec{qeditascli}} can be used to create transactions
which are then ``sent'' by placing them into a file where {\exec{qeditasd}} would read
and publish them. This is unlikely to be a good choice for how the processes should interact,
and it's possible there should simply not be two separate processes.
Another choice would be to have a single {\exec{qeditas}}
which handles the networking and staking in the background while providing a top level
(similar to the Bitcoin Core debug console) for user interaction.
Such a single executable would presumably work within Tails without requiring local sockets.
For other operating systems, perhaps a daemon/command line interface set-up communicating via
local sockets would still be appropriate.

\section{Commands}

The module {\module{commands}} is intended to support a variety of commands a user may need.
At the moment it only supports limited wallet and transaction creation commands.
Some state information is held in this module (although it likely should be moved elsewhere).
\begin{itemize}
\item {\var{walletkeys}} contains the private keys in the wallet.
More specifically it is a list of values of the form $(k,b,(x,y),w,h,a)$
where $k$ is the private key, $b$ is a boolean indicating if it is for the compressed public key,
$(x,y)$ is the public key, $w$ is the string base-58 WIF format, $h$ is the 20-byte hash value
corresponding to the p2pkh address and $a$ is the string base-58 Qeditas p2pkh address.
\item {\var{walletp2shs}} contains entries of the form $(h,a,\overline{b})$
where $h$ is the 20-byte hash value of a p2sh address, $a$ is the base-58 Qeditas p2sh address
and $\overline{b}$ is the sequence of bytes giving the script corresponding to $h$.
Note that this does not directly give a way of ``signing'' for the p2sh address.
\item {\var{walletendorsements}} contains the endorsements in the wallet.
In particular it is a list of values of the form $(\alpha,\beta,(x,y),b,\sigma)$
where $\alpha$ and $\beta$ are pay addresses,\footnote{Actually, in what is implemented we assume they are both p2pkh addresses. In principle endorsements involving p2sh addresses are supported by the code in {\module{sigant}} and {\module{script}}, but support has not been implemented in {\module{commands}}.}
$(x,y)$ is the public key for $\alpha$,
$b$ is a boolean indicating if $\alpha$ is the address for the compressed public key
and $\sigma$ is a signature for a
Bitcoin message of the form ``endorse $\beta$''
(or ``testnet:endorse $\beta$'' in the testnet)
signed with the private key for $\alpha$.
The private key for $\beta$ should be in {\var{walletkeys}}
and this private key along with the endorsement means the wallet can sign for $\alpha$.\footnote{The endorsement mechanism gives Bitcoin users a way to claim their part of the initial distribution without revealing their private keys.}
\item {\var{walletwatchaddrs}}
contains addresses to ``watch.'' 
\item {\var{stakingassets}}
contains a list of assets which the node can stake.
\item {\var{storagetrmassets}}
is intended contain a list of assets at term addresses which the node can use as proof-of-storage to improve the chances of staking.
Currently it is unused.
\item {\var{storagedocassets}}
is intended to contain a list of documents at publication assets which the node can use as proof-of-storage to improve the chances of staking.
Currently it is unused.
\item {\var{recenttxs}} is a hash table associating hash values (transaction ids)
with signed transactions. This is loaded and saved to the file {\file{recenttxs}}.
The intention is that this holds transactions which have been created but
are not yet published (and may only be partially signed).
\item {\var{txpool}} is a hash table associating hash values (transaction ids)
with signed transactions. This is loaded and saved to the file {\file{txpool}}.
The intention is that this holds transactions which have been published
but are not yet included in a block.
\end{itemize}

The following functions are for loading and saving the state in certain files.
\begin{itemize}
\item {\func{load\_currentframe}} loads the current ``frame'' which is used
to determine how the ledger tree is represented.
\item {\func{save\_currentframe}} saves the current ``frame.'' Various commands
could change the frame.
\item {\func{load\_recenttxs}} sets {\var{recenttxs}} by loading the contents fo {\file{recenttxs}}.
\item {\func{load\_txpool}} sets {\var{txpool}} by loading the contents fo {\file{txpool}}.
\item {\func{load\_wallet}} sets {\var{walletkeys}}, {\var{walletp2shs}},
{\var{walletendorsements}}
and {\var{walletwatchaddrs}}
by loading the file {\file{wallet}}.
\item {\func{save\_wallet}} saves the current wallet contents
(the values of {\var{walletkeys}}, {\var{walletp2shs}},
{\var{walletendorsements}}
and {\var{walletwatchaddrs}})
in {\file{wallet}}.
\item {\func{printassets}} prints the assets from the current ledger tree at the addresses
mentioned in the wallet.
\item {\func{btctoqedaddr}} parses a Bitcoin address (base-58 representation) and prints the corresponding Qeditas address (base-58 representation).
\item {\func{importprivkey}} imports a private key given in Qeditas WIF.
\item {\func{importbtcprivkey}} imports a private key given in Bitcoin WIF.
\item {\func{importendorsement}} imports an endorsement.
\item {\func{importwatchaddr}} imports a Qeditas address to watch.
\item {\func{importwatchbtcaddr}} imports a Bitcoin address in order to watch the corresponding Qeditas address.
\end{itemize}

\section{Command Line Interface}

The file {\file{qeditascli.ml}} has a partial implementation of a command line interface.
The value {\val{commhelp}} gives a list of some intended Qeditas commands
along with how many arguments the command may take and help strings.
The help contents can be viewed via the executable itself as follows:
\begin{verbatim}
./bin/qeditascli help
\end{verbatim}
(Note: this does not require the daemon to be running.)
Help for a specific command can be obtained as follows:
\begin{verbatim}
./bin/qeditascli help importendorsement
\end{verbatim}
Many of the commands implemented call functions from {\module{commands}}.

\section{Daemon}

To understand what the daemon executable {\exec{qeditasd}} does,
see the function {\func{main}} in the file {\file{qeditasd.ml}}.
We describe the tasks {\exec{qeditasd}} briefly.

\subsection{Initialization}

It begins by checking the command line arguments for an option
of the form \verb+-datadir=...+
which would override the default data directory (usually {\dir{.qeditas}}
in the user's home directory).
It then reads the config file {\file{qeditas.conf}} in the data directory.
This may override some default values in the {\module{config}} module.
It then reads the other command line arguments which may again override
some values in the {\module{config}} module.

The code then checks if either the {\var{seed}} variable in {\module{config}}
has been set to a nonempty string. (It should be a 40 character hex string.)
The current code in {\module{setconfig}} sets {\var{seed}} to
\begin{verbatim}
68324ba252550a4cb02b7279cf398b9994c0c39f
\end{verbatim}
unless it is specifically set in the configuration file or on the command line.
The value above is the last 20 bytes of the hash of Bitcoin block 378800,
and was a value included only for testing purposes.
The intention was to choose a Bitcoin block height roughly one week in the future
when the time came for Qeditas to launch. The last 20 bytes of that block hash would
be the value for {\var{seed}}. The purpose of this value is to initialize the
current stake modifier and the future stake modifier (see {\func{set\_genesis\_stakemods}}).
These stake modifiers affect which assets will stake within the first 512 blocks.
In particular it affects the genesis block (at Qeditas block height).
For the launch to be fair, these stake modifiers should not be predictable before launch.
This goal could be accomplished in other ways.

The possibility is left open in the future that {\var{seed}} is not set but
that a checkpoint has been set so that a node can begin following the block chain
from that checkpoint. (The full history is not required. Each ledger tree contains
the full information required to continue.)

If the {\var{testnet}} configuration variable is set, then the difficulty is decreased significantly
(setting {\var{genesistarget}} $2^{248}$ so that finding a hit to stake is not difficult
and setting {\var{max\_target}} $2^{255}$, practically as high as possible).
The ``current frame'' is loaded 
and {\var{localframe}} is set to it
and {\var{localframehash}} is set to its hash.
The frame is needed because a ledger root will, in general, correspond to multiple
ledger abbreviations since the abbreviation depends on the frame.
The function {\func{load\_root\_abbrevs\_index}}
loads the current information about the association of ledger hash roots with
frame hashes and ledger abbreviation hashes.
The ledger abbreviation hashes is the name of the file
in the {\dir{ctrees}} directory
where the corresponding {\type{ctre}} is stored.

A random seed is generated by reading 512 bits from {\file{/dev/random}} (or {\file{/dev/urandom}} for the testnet)
and this seed is used to generate random values in the future by hashing the value and taking
some of the bits when required.
{\bf{Warning:}} This is (probably) unsafe and should not be used in production.
Cryptographically safe random values should be obtained when required.
The code is the way it currently is simply because it was under development and
being tested.

The random value above is used to set {\var{this\_nodes\_nonce}} which prevents the node
from connecting to itself.

If {\var{ip}} and {\var{port}} are set, then a listener socket is opened (via {\func{openlistener}})
to listen for incoming connections.

If {\var{staking}} is set, then the wallet is loaded and the function {\func{start\_staking}}
starts a {\exec{qeditasstk}} process which independently searches for an asset that can stake.
This process is stopped when a hit is found and is restarted each time a decision is reached
to being staking on top of a new block.

The function {\func{sethungsignalhandler}} sets the signal handler for timeouts to raise the exception {\exc{Hung}}.
This is used to create functions that do not wait indefinitely for input or connections.

The function {\func{loadknownpeers}} loads a file with recent peers.
Then {\func{search\_for\_conns}} is called which tries to connect to these recent peers,
and, if no connections succeed, tries to connect to fallback nodes ({\func{getfallbacknodes}}).

Recent transactions and the transaction pool is loaded ({\func{load\_recenttxs}} and {\func{load\_txpool}}.

At this point the daemon is ready to enter the main process loop.

\subsection{Main Process Loop}

Currently the testnet sleeps for 1 second between many of the steps of the main loop.
This is simply to make the output reasonable while testing.

The next paragraphs are only relevant if {\var{staking}} is set to {\val{true}},
otherwise they can be skipped.

The process checks to see if a hit has been found (by reading from the
stdout of the {\exec{qeditasstk}} process). If a hit has been found, it is likely
for some timestamp some seconds in the future.
In preparation to stake, a block is created, possibly including transactions from the transaction pool.
The code checks that the block is valid and, if so, sets {\var{waitingblock}}
to a value that remembers when the hit will be valid and information about the new block.
Staking pauses until either a new block is received or until the time is reached
that the found hit will be valid.

If {\var{waitingblock}} is set and the current time is at least the time the hit is valid,
then the new block header is inserted into the sorted {\var{recentblockheader}} list
and the remaining block information is saved.
An inventory message with the block header, the block delta skeleton and the block delta
is broadcast to all connections.
Staking is restarted, presumably on top of this new block.

Assume there is no {\var{waitingblock}}.
If no staking process is running, then start a staking process.
If there is a staking process, check if it is staking on top of the block with the most
cumulative stake. If not, restart staking on top of the current best block.

From this point on in the main process loop, the value of {\var{staking}} is not relevant.

If there is a listener process, {\func{accept\_nohang}} checks (with a timeout of 0.1s)
if there is a connection to accept. If so, {\func{initialize\_conn\_accept}} begins the
handshake protocol.

For each preconnection on {\var{preconns}}, progress in the handshaking protocol is attempted.
The handshake protocol requires one process to send a {\constr{Version}} message,
with a {\constr{Verack}} sent back followed by a {\constr{Version}} message
and a final {\constr{Verack}}. This follows the Bitcoin handshake.
If the handshake fails, the connection is dropped and removed from {\var{preconns}}.
If the handshake succeeds, the connection is moved from {\var{preconns}} to {\var{conns}}.

We next turn to handling the connections on {\var{conns}}.
Here the connection state ({\type{connstate}}) plays an important role.

Each connection on {\var{conns}} is checked for incoming messages.

If there is an incoming message, {\func{handle\_msg}} is called to
perform the appropriate actions.
Assume there is no incoming message.
If some request was pending for more than 90s (e.g., a {\constr{Ping}} message
expecting a {\constr{Pong}} reply), then mark the close connection 
and mark it as no longer alive. If there have been no messages in 90 minutes,
send a {\constr{Ping}} and save this in {\var{pending}}.

Each connection that is no longer alive is removed from {\var{conns}}.

At the end of the main loop process, there are some checks to see if more connections should be searched
for. Also, {\func{handle\_orphans}} and {\func{handle\_delayed}}
are called in order to double check that the node is not missing a chain with more cumulative stake.
(Such chains were being missed during testing. It's unclear if 
{\func{handle\_orphans}} and {\func{handle\_delayed}} fixed the problem or not.)








\printindex

\bibliographystyle{plain}
\bibliography{refs}

\end{document}
