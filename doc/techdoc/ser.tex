The module \module{ser} (\file{ser.ml} and \file{ser.mli}) contains the basic code for serializing data.
Throughout the code there are functions with names of the form {\sf{seo\_}}$\tau$ (output)
and {\sf{sei\_}}$\tau$ (input).
In each case {\sf{seo\_}}$\tau$ is a function for creating a serialized output for an
element of type $\tau$
and {\sf{sei\_}}$\tau$ is a function for creating an element of type $\tau$ given its serialization.

There are two representations for the serializations: strings and channels
and the input and output functions are polymorphic so they can support both representations.
The types \type{seosbt} and \type{seist} correspond to the string representation
and there are corresponding atomic
functions \func{seosb} (output bits to a string buffer), \func{seosbf} (flush output to string buffer) and \func{seis} (input bits from a string).
The types \type{seosct} and \type{seict} correspond to the channel representation
and there are corresponding atomic
functions \func{seoc} (output bits to a channel), \func{seocf} (flush output channel) and \func{seic} (input bits from a channel).

The functions to output its take three arguments: the number of bits $n$, the bits to output as an integer $m$ with $0\leq m < 2^n$ and the serialization output object.
The functions to flush the output takes the serialization output object
and ensures any remaining its are output, assuming the remaining its are zero.
The functions to input its from a channel take two arguments: the number of bits $n$ to input and the serialization input object. It returns both an integer $m$ where $0\leq m < 2^n$ and the serialization input object.

The remainder of the functions defined in \module{ser} are for
serialization basic data: booleans, bytes, 32-bit integers, 64-bit integers, big integers (the OCaml type \type{big\_int}) assumed to be 256-bit integers and strings.
There are also other serialization functions for integers: \serfunc{seo\_varint} and \serfunc{sei\_varint} uses the varint representation used in Bitcoin, while \serfunc{seo\_varintb} and \serfunc{sei\_varintb} uses a different compact representation to represent numbers less than $65,536$.

In addition there are functions used to construct serialization functions
for list types, option types and product types with up to $6$ components.
For example, there are functions \serfunc{seo\_list} and \serfunc{sei\_list}.
When serialization functions are needed for lists of bytes, we simply
use \serfunc{seo\_list} applied to \serfunc{seo\_int8}
and \serfunc{sei\_list} applied to \serfunc{sei\_int8}.

Note that the serialization code is inherently higher-order (functions are first-class values).
Firstly, the atomic functions are passed as arguments to the serialization functions
for each type so the same serialization code can be used for both representations.
Secondly, the serialization functions themselves are passed to functions like \serfunc{seo\_list}
and \serfunc{sei\_list}.

There is one minor issue with the serialization code which may be confusing
and hopefully will not be a nightmare to maintain.
The bits are used to construct a byte from least significant to most significant.
As a consequence, different ways to output the same sequence of bits can be confusing.
Let $o$ be an atomic output function (either $\func{seosb}$ or $\func{seoc}$).
For example, suppose we wish to output the bit $0$ followed by the bit $1$.
We can do this in one of two ways:
\begin{itemize}
\item Call $o$ twice: first as $o\,1\,0$ (to output the one bit $0$) and then as $o\,1\,1$ (to output the one bit $1$).
\item Call $o$ once as $o\,2\,2$ (to output the two bits $10$, the binary representation of $2$).
\end{itemize}
In some places this can make serialization code difficult to correctly interpret.

{\bf{Note:}} The unit tests for the {\module{ser}} module are in {\file{basicunittests.ml}}
in the {\dir{src/unittests}}
directory in the {\branch{testing}} branch.
These unit tests give a number of examples demonstrating how the functions described above should behave.
