For several data types we will need to manipulate persistent storage of values 
indexed by a hash value. (We will call this a ``database'' although 
it is only a key-value mapping.)
One way to do this would be to use a standard library built for this purpose,
such as leveldb.
However, integrating leveldb with the OCaml code has proven challenging.
Instead (at least for the moment) the database has been implemented
by simply using files in directories.
The particular implementation has been abstracted using a module type
so that the implementation of the module can be easily replaced.\footnote{Trent Russell was responsible for the initial/current implementation of the database code.}

The module type {\moduletype{dbtype}} is actually a functor type.
It depends on a signature with
\begin{itemize}
\item a type {\tt{t}} (the type of the values to be stored),
\item a string {\tt{basedir}} (indicating the top level directory where these key-value pairs will be stored) and
\item functions {\tt{seival}} and {\tt{seoval}} for deserializating and serializing the data from and to channels.
\end{itemize}
An implementation of {\moduletype{dbtype}} must implement the following:
\begin{itemize}
\item {\func{dbget}} taking a hash value (as the key) to value of type {\tt{t}} (or raising {\exc{Not\_found}}).
\item {\func{dbexists}} takes a hash value (as the key) and returns {\val{true}} if there is an entry with this key and returns {\val{false}} otherwise. (One could use {\func{dbget}} for this purpose, but {\func{dbget}} must take the time to deserialize corresponding the value.)
\item {\func{dbput}} takes a hash value (as the key) and a value of type {\tt{t}} and stores the key-value pair.
\item {\func{dbdelete}} takes a hash value (as the key) and deletes the entry with this key, if one exists. If there is no entry, {\func{dbdelete}} does nothing.
\end{itemize}

There is a functor {\module{Dbbasic}} which returns
a module implementing {\moduletype{dbtype}}, given an implementation of {\tt{t}}, {\tt{basedir}}, {\tt{seival}} and {\tt{seoval}}.
The implementation of {\module{Dbbasic}}
uses subdirectories of {\tt{basedir}}
with three files: {\file{index}},
{\file{data}}
and {\file{deleted}}.
The file {\file{data}} contains serializations of the values stored in this directory
and the file {\file{index}} contains the keys (hash values) along with integers giving
the position of the corresponding data in {\file{data}}.
The file {\file{deleted}} is a list of heys (hash values)
that have been marked as deleted (but the keys are still in {\file{index}}
and the value is still in {\file{data}}).

The keys in {\file{index}} are ordered. The actual ordering is not important, as long
as it is a total ordering. Since hash values are internally implemented as
a 5-tuple signed 32-bit integers, the OCaml function {\func{compare}}
can be used. This means $h = (x_0,x_1,x_2,x_3,x_4)$ is less than $k=(y_0,y_1,y_2,y_3,y_4)$
if 
$x_0 < y_0$ (as a signed 32-bit integer)
or $x_0 = y_0$ and $x_1 < y_1$
or $x_0 = y_0$, $x_1 = y_1$ and $x_2 < y_2$
or $x_0 = y_0$, $x_1 = y_1$, $x_2 = y_2$ and $x_3 < y_3$
or $x_0 = y_0$, $x_1 = y_1$, $x_2 = y_2$, $x_3 = y_3$ and $x_4 < y_4$.

The maximum number of entries in the files in a directory is 65536,
but new entries are also not allowed after the {\file{data}} file
exceeds 100 MB.
After no more entries are permitted in a directory,
a subdirectory named using the next byte (in hex) of the key is
created (if necessary) and this subdirectory is used, unless it is also full.

Some auxiliary functions are used:
\begin{itemize}
\item {\func{find\_in\_deleted}} checks if a key is in the {\file{deleted}} file of a directory.
\item {\func{load\_deleted}} loads all the hash values (keys) in the {\file{deleted}} file of a directory.
\item {\func{undelete}} removes a key from the {\file{deleted}} file of a directory
by loading all the deleted keys and then recreating the {\file{deleted}} file without given the key.
\item {\func{count\_deleted}} gives the number of entries in the {\file{deleted}} file of a directory.
\item {\func{find\_in\_index}}
searches for a key
by loading the {\file{index}} file and doing a binary search.
If it is found, then the position of the value in the {\file{data}}
file is returned. Otherwise, {\exc{Not\_found}} is raised.
\item {\func{load\_index}} loads the index file as a list of pairs of hash values and positions.
This list is reverse sorted by the hash values.
\item {\func{count\_index}} gives the number of entries in the {\file{index}} file of a directory.
\item {\func{dbfind}} and {\func{dbfind\_a}}
are used to search for a subdirectory where the {\func{index}} file
contains a given key, returning the directory and value position
if one is found. If none is found, {\exc{Not\_found}}
is raised.
\item {\func{find\_next\_space}}, {\func{find\_next\_space\_a}} and {\func{find\_next\_space\_b}}
are used to find the next appropriate subdirectory and position where a key
can be included.
\item {\func{defrag}} cleans up by actually deleting key-value pairs which have been deleted.
\end{itemize}

The implementation of {\module{Dbbasic}} works as follows:
\begin{itemize}
\item There are two hash table {\val{cache1}} and {\val{cache2}}
which store (roughly) the last 128 to 256 entries looked up
so that these can be returned again quickly.
(When one cache hash 128 entries, the other is cleared and begins
to be filled.)
The internal functions {\func{add\_to\_cache}} and {\func{del\_from\_cache}}
handle adding and removing key-value pairs fromt he cache.
\item {\func{dbget}} tries to find the key in the cache.
If it is not in the cache, {\func{dbfind}} is called to 
try to find a subdirectory and value position.
If one is found and the key has not been deleted, then
the value is deserialized from the {\file{data}} file starting at the position
and returned.
Otherwise, {\exc{Not\_found}} is raised.
\item {\func{dbexists}} is analogous to {\func{dbget}} except it
does not deserialize the value if it is found.
Instead it returns {\val{true}} if a subdirectory and value position were found
(and the key is not marked as deleted),
and returns {\val{false}} otherwise.
\item {\func{dbput}} takes a key $k$ and value $v$.
The function {\func{dbfind}} is called to find an entry for $k$ if one exists.
If one is found and it has been marked as deleted, then undelete it.
If one is found and it has not been deleted, then simply return -- as the
key value pair already exists.
Otherwise, call {\func{dbfind\_next\_space}} to find the next subdirectory
and position where the new value can be stored (which is at the end of the {\file{data}} file,
or $0$ if no {\file{data}} file yet exists).
The function {\func{load\_index}} at this subdirectory is used
to get a reverse sorted list of the current keys and positions.
The list is reversed and the new key and position are merged into the list.
This new list is stored in {\file{index}} replacing the previous contents.
The value is deserialized and appended to the end of the {\file{data}} file.
\item {\func{dbdelete}} takes a key and uses {\func{dbfind}} to find a subdirectory
where the corresponding {\file{index}} file contains the key.
If none is found, then do nothing.
Assume a subdirectory is found.
If the key is already in the {\file{deleted}} file of this subdirectory, then do nothing.
Otherwise, append the key to the {\file{deleted}} file.
If the number of deleted entries in this subdirectory exceeds 1024, then {\func{defrag}} is called.
\end{itemize}

One simple database module {\module{DbBlacklist}}
is defined by giving {\module{Dbbasic}} the type {\type{bool}} and 
base directory {\dir{blacklist}}.
This is intended to save keys corresponding to some blacklisted data that
should not be requested from peers.

Other instantions of {\module{Dbbasic}}
occur where the corresponding data types are defined.
For assets this is in {\module{assets}}
and
for (signed) transactions this is in {\module{tx}}
(see Chapter~\ref{chap:assetstx}).
For hcons elements and ctree elements,
giving approximations of parts of ledger trees,
this is in {\module{ctre}}
(see Chapter~\ref{chap:ctre}).
For block headers and blocks,
this is in {\module{block}}
(see Chapter~\ref{chap:block}).
