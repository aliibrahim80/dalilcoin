The module \module{assets} defines a type {\type{asset}}.
It also contains code to support the inputs and outputs of transactions.
The module \module{tx} defines a type {\type{tx}} of transactions
and a type {\type{stx}} of signed transactions,
as well as code for checking the validity of transactions
and their signatures.
``Validity'' of a transaction is a weak form of correctness
that a transaction must satisfy before asking if it is supported by the current ledger.

\section{Assets}

An asset consists of four pieces of information:
a hash value (the {\defin{asset id}}),
a 64-bit integer giving the block in which the asset was published (the {\defin{birthday}}),
an {\type{obligation}} (indicating who controls the asset)
and a {\type{preasset}} (determining the kind of asset).
In the case of an asset in the initial distribution,
the asset id is the 160-bit hash value corresponding to the p2pkh or p2sh address.
(Since no p2pkh or p2sh addresses in the snapshot had the same 160-bit address,
these asset ids are unique.)
In the case of an asset created as the output of a transaction,
the asset id is formed from hashing the transaction id paired with the index of the output
creating the identifier.
Assets in the initial distribution are given birthday $0$
and the first block will be considered to have block height $1$.
Obligations and preassets are described below.

{\bf{Note:}} Unit tests for the {\module{assets}} module are in {\file{assetsunittests.ml}}
in the {\dir{src/unittests}}
directory in the {\branch{testing}} branch.
These unit tests give a number of examples demonstrating how the functions described below should behave.

{\bf{Note:}} The Coq module {\coqmod{Assets}} is intended to correspond to {\module{assets}}.
There are Coq types {\coqtype{preasset}}, {\coqtype{obligation}} and {\coqtype{asset}}
corresponding to the types with the same names defined in OCaml.
One difference is that in the OCaml code an obligation also keeps a boolean indicating if the
asset is a reward since this was needed to implement forfeiture of rewards in case a staker
double signs.
Readers can examine the formal properties proven in {\coqmod{Assets}} to have a better
idea of what properties corresponding OCaml functions should satisfy.
For more information, see~\cite{White2015b}, although preassets in the version described there are restricted to currency units.

\subsection{Obligations}

We first consider the type {\type{obligation}}:
\begin{verbatim}
type obligation = (payaddr * int64 * bool) option
\end{verbatim}
Note that an
obligation may be empty, which usually means the address that {\defin{holds}}
the asset can spend it (here {\defin{holds}} refers to the address in the ledger
tree where the asset is stored).
In case an obligation is not empty, it consists of a triple $(\alpha,n,r)$.
Here $\alpha$ is a pay address (a p2pkh address or p2sh address) which must
sign in order to spend the asset.
(The holder of the asset is the one who can use the asset to stake.
Hence obligations can be used to ``loan'' an asset to a staker
without giving the staker the ability to ``spend'' the asset.)
The integer $n$ is the earliest block height at which the asset can be spent.
(The intention here is to ``lock'' an asset for a period of time. Such ``locked''
assets are given preference when staking.)
The boolean $r$ indicates if the asset is a reward for staking a block.\footnote{Philosophically, this should not be part of the ``obligation,'' but the reward indicator was added late and this was a simple way to include it.}
Rewards are considered special in the sense that they can be forfeited in the first 6 blocks
after the reward is issued, if the issuer provably double signs within the next 6 blocks.

\subsection{Categories of Preassets and Assets}

There are 11 kinds of {\defin{assets}}, as determined
by the corresponding {\defin{preasset}}: currency units, bounties, object ownership,
proposition ownership, negated proposition ownership,
object rights, proposition rights, markers,
theory publications, signature publications and document publications.
The type {\type{preasset}} consists of the following 11 corresponding constructors:
\begin{itemize}
\item ${\constr{Currency}}(n)$ represents $n$ cants of currency units, where $n$ is a 64-bit integer.
A {\defin{cant}} is the smallest currency unit considered in Qeditas.\footnote{The word ``cants'' is pronounced with a hard c as it is derived from the name Cantor.}
Currency units can be transfered by fulfilling the appropriate obligation (which usually simply
means signing the transaction spending the asset with an appropriate private key).
% Currency units are always held at pay addresses. % I no longer remember if this is true.
\item ${\constr{Bounty}}(n)$ represents $n$ cants as a bounty on a proposition.
Bounties are held at term addresses, specifically at the term address of a proposition in a theory.
A bounty can be spent (and transformed into currency) by the proposition owner 
or negated proposition owner.
Typically neither the proposition nor its negation have been proven in the theory
(and so it has neither a proposition owner nor a negated proposition owner)
when the bounty is placed.\footnote{The ``proposition owner'' is determined by the (nonempty) obligation at the proposition ownership asset held at the term address, if there is such an asset. Likewise, the ``negated proposition owner'' is determined by the obligation at the negated proposition ownership asset held at the term address, if there is such an asset.}
If someone publishes a document in which the proposition is proven,
the publisher declares the proposition owner.
Likewise, if someone publishes a document in which the negation of the proposition is proven,
the publisher declares the negated proposition owner.
In either case, the new owner (presumably the publisher) can then collect the bounty.
\item ${\constr{OwnsObj}}(\alpha,p)$ corresponds to a declaration of object ownership of a term (either
a pure term or a term in a theory).
The $\alpha$ is a pay address and the $p$ is an optional 64 bit integer.
The actual object owner is determined by the obligation of the corresponding asset (and so may
or may not be $\alpha$).
The address $\alpha$ is intended as an address others can pay in order to purchase rights
to use the object (as an imported parameter) in future documents.
The optional value $p$ gives the price (in cants) to purchase one right.
If $p$ is $0$, then the object can be freely used (without a need to purchase rights).
If $p$ is {\val{None}}, then the object cannot be used
in this way at all (and rights cannot be purchased).
(The object can always be used in a new document by repeating the definition.)
\item ${\constr{OwnsProp}}(\alpha,p)$ corresponds to a declaration of proposition ownership of a term (either
a pure term or a term in a theory).
The $\alpha$ is a pay address and the $p$ is an optional 64 bit integer.
The actual proposition owner is determined by the obligation of the corresponding asset (and so may
or may not be $\alpha$).
The address $\alpha$ is intended as an address others can pay in order to purchase rights
to use the proposition (as an imported known) in future documents.
The optional value $p$ gives the price (in cants) to purchase one right.
If $p$ is $0$, then the proposition can be freely used (without a need to purchase rights).
If $p$ is {\val{None}}, then the proposition cannot be used at all (and rights cannot be purchased).
(The proposition can always be used in a new document by reproving it.)
\item ${\constr{OwnsNegProp}}$ corresponds to a declaration of a negated proposition ownership of a term
in a theory.
Again, the ``owner'' is determined by the corresponding obligation.
This kind of asset is only to facilitate the collection of a bounty by disproving a conjecture
with a bounty.
\item ${\constr{RightsObj}}(\alpha,n)$ corresponds to the right to use
the object with term address $n$ times.
Some or all of these rights will be consumed when publishing a document which imports the object
as a parameter (omitting the definition).
Note that to use objects within a theory, rights may be required for the
pure object (independent of the theory) and for the object within the theory.
These are two different term addresses.
\item ${\constr{RightsProp}}(\alpha,n)$ corresponds to the right to use the proposition
with term address $\alpha$ $n$ times.
Some or all of these rights will be consumed when publishing a document which imports the proposition
as a known (without proof).
Note that to use propositions within a theory, rights may be required for the
pure proposition (independent of the theory) and for the proposition within the theory.
These are two different term addresses.
\item ${\constr{Marker}}$ is for part of the protocol for publishing a document.
A publication address is determined by the (privately known) publication with a (privately known) nonce.
A marker must be at the publication address (as an {\defin{intention to publish}})
for 144 blocks (see {\var{intention\_minage}}) before the actual publication can be published.
The idea is that the true author of the document publishes the marker roughly a day before
revealing the publication itself. The publication is revealed in the transaction publishing it.
At that point, a plagiarist could take the publication, compute a new nonce, publish a new marker
and then try to publish their copy. However, they would need to wait at least 144 blocks before their
copied version could be published. By that time, the original publication should already be published.
The order of publication is important since this may determine ownership of
newly defined objects and newly proven propositions.
\item ${\constr{TheoryPublication}}(\alpha,\nu,\tau)$ is a preasset
for publishing a theory specification ({\type{theoryspec}}) $\tau$.
The pay address $\alpha$ identifies the author (possibly ``publisher'' is more accurate)
and the corresponding transaction
creating such an asset must be signed by $\alpha$.
The hash value $\nu$ is a nonce to determine the publication address for the marker
which must be published 144 blocks before the publication can be published.
\item ${\constr{SignaPublication}}(\alpha,\nu,h,\Sigma)$ is a preasset
for publishing a signature specification ({\type{signaspec}}) $\Sigma$.
The pay address $\alpha$ identifies the author and the corresponding transaction
creating such an asset must be signed by $\alpha$.
The hash value $\nu$ is a nonce to determine the publication address for the marker
which must be published 144 blocks before the publication can be published.
The optional hash value $h$ identifies the theory in which the signature belongs.
An object or proposition can only be included in a signature if no rights are required
to use the object or proposition.
(The empty theory is identified by giving {\val{None}} for $h$.)
\item ${\constr{DocPublication}}(\alpha,\nu,h,\Delta)$ is a preasset
for publishing a document ({\type{doc}}) $\Delta$.
The pay address $\alpha$ identifies the author and the corresponding transaction
creating such an asset must be signed by $\alpha$.
The hash value $\nu$ is a nonce to determine the publication address for the marker
which must be published 144 blocks before the publication can be published.
The optional hash value $h$ identifies the theory in which the signature belongs.
(The empty theory is identified by giving {\val{None}} for $h$.)
\end{itemize}

The type {\type{asset}} of assets is now simply defined as a product.
\begin{verbatim}
type asset = hashval * int64 * obligation * preasset
\end{verbatim}
The functions {\func{assetid}}, {\func{assetbday}},
{\func{assetobl}} and {\func{assetpre}}
extract the components from the asset.

\subsection{Types for Transaction Inputs and Outputs}

The inputs of transactions will be pairs of addresses and asset identifiers (hash values)
of assets held at these addresses. The type {\type{addr\_assetid}}
plays the role of a transaction input and is defined as follows:
\begin{verbatim}
type addr_assetid = addr * hashval
\end{verbatim}
The outputs of transactions are triples of addresses, obligations and preassets.
(The asset identifier is determined by the transaction itself and the birthday
is determined by the block height in which the transaction is included.)
The type {\type{addr\_preasset}} plays the role of a transaction output and is defined as follows:
\begin{verbatim}
type addr_preasset = addr * (obligation * preasset)
\end{verbatim}
The inputs and outputs of a transaction can be elaborated into a pair of an address with an asset
in certain situations.
While checking a transaction is supported the input assets are looked up from the ledger tree
using the asset identifier. 
A transaction output gives the obligation and preasset.
When a transaction is being included in a block at a given height,
we know the birthday and can use this (along with the asset identifier which 
is derived from the transaction) to form the asset.
The type {\type{addr\_asset}} is included to represent such an elaborated input or output.
\begin{verbatim}
type addr_asset = addr * asset
\end{verbatim}

\subsection{Functions}

The functions {\func{hashobligation}} hashes an obligation (returning {\val{None}}
for the {\val{None}} obligation).
The functions {\func{hashpreasset}},
{\func{hashasset}},
{\func{hash\_addr\_assetid}},
{\func{hash\_addr\_preasset}} and
{\func{hash\_addr\_asset}}
hash the corresponding types.

As usual, there are functions for serializing and deserializing elements of these types:
{\serfunc{seo\_obligation}},
{\serfunc{sei\_obligation}},
{\serfunc{seo\_preasset}},
{\serfunc{sei\_preasset}},
{\serfunc{seo\_asset}},
{\serfunc{sei\_asset}},
{\serfunc{seo\_addr\_assetid}},
{\serfunc{sei\_addr\_assetid}},
{\serfunc{seo\_addr\_preasset}},
{\serfunc{sei\_addr\_preasset}},
{\serfunc{seo\_addr\_asset}}
and
{\serfunc{sei\_addr\_asset}}.

The purpose of the remaining functions exported by the {\module{assets}} module are as follows:
\begin{itemize}
\item {\func{new\_assets}} takes a birthday $b$, an address $\alpha$,
an {\type{addr\_preasset}} list (transaction outputs), a hash value (which should
be the hash of the transaction)
and an output index (which should be $0$ in the initial call)
and returns a list of assets which would be put into address $\alpha$
if the transaction is published at block height $b$.
The transaction hash and output index are used to compute the asset ids.
\item {\func{remove\_assets}} takes an asset list and a list of asset identifiers (the ``spent list'') and
returns the asset list after removing the assets with ids in the spent list.
\item {\func{get\_spent}} takes an address $\alpha$ and an {\type{addr\_assetid}} list (transaction inputs)
and returns a list of asset ids being spent from the given address.
\item {\func{add\_vout}} is similar to {\func{new\_assets}} except it is not specific to an address.\footnote{It might make sense to delete one of these functions in favor of the other.}
It takes a birthday $b$, a hash value (which should be the hash of the transaction),
an {\type{addr\_preasset}} list (transaction outputs)
and an output index (which should be $0$ in the initial call),
and returns an {\type{addr\_asset}} list consisting of the fully elaborated output assets
(assuming the transaction is published at block height $b$).
\item {\func{preasset\_value}} returns the optional number of cants that a preasset is worth.
Only currency units and bounties are worth cants. For other preassets, {\val{None}} is returned.
\item {\func{asset\_value}} returns the value of the underlying preasset.
\item {\func{asset\_value\_sum}} returns the sum of the value of a list of assets (where {\val{None}} is counted as $0$).
\item {\func{output\_signaspec\_uses\_objs}} takes
an {\type{addr\_preasset}} list (transaction outputs)
and returns a list of pairs of term addresses.
For each object imported as a parameter by a signature specification being published as one of the outputs,
$(\alpha,\beta)$ will be on the output list
where $\alpha$ is the term address given by the hash root $h$\footnote{Recall that term addresses are actually hash values, so $\alpha = h$.}
of the term which was used to define the object
and $\beta$ is the term address given by hashing $h$ with the type of the object and with the identifier of the current theory
(and then tagging this with $32$ to avoid accidental collision).
If an object is imported by multiple
different signatures being published, then the pair will be on the list multiple times.
The information is obtained by calling {\func{signaspec\_uses\_objs}} on appropriate preassets.
\item {\func{output\_signaspec\_uses\_props}} takes
an {\type{addr\_preasset}} list (transaction outputs)
and returns a list of pairs of term addresses.
For each proposition imported as a known by a signature specification being published as one of the outputs,
$(\alpha,\beta)$ will be on the output list
where $\alpha$ is the term address given by the hash root $h$
of the proposition
and $\beta$ is the term address given by hashing $h$ with the identifier of the current theory
(and then tagging this with $33$ to avoid accidental collision).
If a proposition is imported by multiple
different signatures being published, then the pair will be on the list multiple times.
The information is obtained by calling {\func{signaspec\_uses\_props}} on appropriate preassets.
\item {\func{output\_doc\_uses\_objs}} takes
an {\type{addr\_preasset}} list (transaction outputs)
and returns a list of pairs of term addresses.
For each object imported as a parameter by a document being published as one of the outputs,
$(\alpha,\beta)$ will be on the output list
where $\alpha$ is the term address given by the hash root $h$
of the term which was used to define the object
and $\beta$ is the term address given by hashing $h$ with the type of the object and with the identifier of the current theory
(and then tagging this with $32$ to avoid accidental collision).
If an object is imported by multiple
different documents being published, then the pair will be on the list multiple times.
The information is obtained by calling {\func{doc\_uses\_objs}} on appropriate preassets.
\item {\func{output\_doc\_uses\_props}} takes
an {\type{addr\_preasset}} list (transaction outputs)
and returns a list of pairs of term addresses.
For each proposition imported as a known by a document being published as one of the outputs,
$(\alpha,\beta)$ will be on the output list
where $\alpha$ is the term address given by the hash root $h$
of the proposition
and $\beta$ is the term address given by hashing $h$ with the identifier of the current theory
(and then tagging this with $33$ to avoid accidental collision).
If a proposition is imported by multiple
different document being published, then the pair will be on the list multiple times.
The information is obtained by calling {\func{doc\_uses\_props}} on appropriate preassets.
\item {\func{output\_creates\_objs}} takes
an {\type{addr\_preasset}} list (transaction outputs)
and returns a list of triples $(t,h,k)$
identifying objects defined in a document being published as one of the outputs.
Here $t$ is the (optional) hash value identifier of the theory in which the document lives,
$h$ is hash root of the term defining the object
and $k$ is the hash of the type of the object.
(The term address of the pure object will be $h$
and the term address of the object in the theory will be the hash of
$h$ with $t$, $k$ and the tag $32$.)
If an object is created by multiple
different documents being published, then the triple will be on the list multiple times.
The information is obtained by calling {\func{doc\_creates\_objs}} on appropriate preassets.
If the pure term address for a created object is unowned,
then it is new and must be given an owner (both as a pure object
and as an object within the theory) with the same transaction
publishing the document.
If the pure term address for a created object is owned,
but the term address within the theory is unowned, then
the object is new for the theory and the term address within the theory
must be given an owner (as an object) with the same transaction
publishing the document.
\item {\func{output\_creates\_props}} takes
an {\type{addr\_preasset}} list (transaction outputs)
and returns a list of pairs $(t,h)$
identifying propositions which are known as the result of a publication in the outputs.
Usually, this will mean $t$ is the (optional) hash value identifier of the theory in which the document lives
and $h$ is the hash root of a proposition proven in a document being published.
Alternatively, a pair $(t,h)$ can be included due to an axiom being assumed in a newly published theory
specification. In this case, $t$ is the hash value identifier of the theory derived from the
new theory specification and $h$ is the hash root of one of the propositions given as an axiom of the theory.
Again, multiple publications may result in $(t,h)$ being included multiple times.
The function uses {\func{doc\_creates\_props}}.
If the pure term address for a created proposition is unowned,
then it is new and must be given an owner (both as a pure proposition
and as a proposition within the theory) with the same transaction
publishing the document.
If the pure term address for a created proposition is owned,
but the term address within the theory is unowned, then
the proposition is new for the theory and the term address within the theory
must be given an owner (as a proposition) with the same transaction
publishing the document.
\item {\func{output\_creates\_neg\_props}} takes
an {\type{addr\_preasset}} list (transaction outputs)
and returns a list of pairs $(t,h)$
identifying propositions whose negations are proven in a document published in the outputs.
Here this means there is a document being published in the theory identified by the
(optional) hash value $t$
and a proposition $\neg s$\footnote{Here $\neg s$ actually means literally negation ($\lambda_o x_0\to\bot$) applied to $s$ or $s\to\bot$ where $\bot$ is $\forall_o x_0$.}
is proven in the document and $h$ is the hash root of $s$.
The information is obtained by calling {\func{doc\_creates\_neg\_props}} on appropriate preassets.
There is no requirement to declare an owner for a created negated proposition.
Negated propositions cannot be ``used'' in the sense that an object or proposition can
be used. The only purpose for declaring ownership of a negated proposition is to collect a bounty.
\item {\func{rights\_out\_obj}} takes an {\type{addr\_preasset}} list (transaction outputs)
and a term address $\alpha$ and sums the number of rights to use $\alpha$ as an object
created by the outputs.
\item {\func{rights\_out\_prop}} takes an {\type{addr\_preasset}} list (transaction outputs)
and a term address $\alpha$ and sums the number of rights to use $\alpha$ as a proposition
created by the outputs.
\item {\func{count\_obj\_rights}} takes a list of assets and a term address $\alpha$
and sums the number of rights to use $\alpha$ as an object contained in the asset list.
\item {\func{count\_prop\_rights}} takes a list of assets and a term address $\alpha$
and sums the number of rights to use $\alpha$ as a proposition contained in the asset list.
\item {\func{count\_rights\_used}} takes list of pairs of term addresses and a term address $\alpha$
and counts the number of times $\alpha$ occurs as one of the pairs. (Technically it counts
how many times $\alpha$ is at least one of the pairs, but in practice $(\alpha,\alpha)$ will
never occur.)
The purpose is to determine how many times $\alpha$ is ``used'' by publications
given in outputs of a transaction.
\item {\func{obj\_rights\_mentioned}} takes an {\type{addr\_preasset}} list (transaction outputs)
and returns a list of term addresses.
A term address is included in the output if
object rights for it are being explicitly output
(as {\constr{RightsObj}} preassets)
or if the object is being used in a document being published.
(We do not count uses in signature specification publications since
a signature specifications will only be allowed if rights are not required.)
\item {\func{prop\_rights\_mentioned}} takes an {\type{addr\_preasset}} list (transaction outputs)
and returns a list of term addresses.
A term address is included in the output if
proposition rights for it are being explicitly output
(as {\constr{RightsProp}} preassets)
or if the proposition is being used in a document being published.
(We do not count uses in signature specification publications since
a signature specifications will only be allowed if rights are not required.)
\item {\func{rights\_mentioned}} combines the results of {\func{obj\_rights\_mentioned}}
and {\func{prop\_rights\_mentioned}} to give a list of all term addresses
where rights are either being output or may be consumed to publish a document.
\item {\func{units\_sent\_to\_addr}} takes an address $\beta$
and an {\type{addr\_preasset}} list (transaction outputs)
and sums the value of the currency units being sent to $\beta$.
The purpose of this is to facilitate the purchase of rights by paying $\beta$.
\item {\func{out\_cost}} takes an {\type{addr\_preasset}} list (transaction outputs)
and sums the total cost of publishing the transaction.
This includes the value of all currency and bounty assets created by the output
as well as the burn cost required to publish theory specifications
and signature specifications.
\end{itemize}

\subsection{Creation of Objects and Propositions}\label{sec:outputcreates}

Let $h$ be a theory identifier (an optional hash value),
$\Delta$ be a document
and $\alpha$ be a term address.
\begin{itemize}
\item We say $(h,\Delta)$ {\defin{creates the object}} at $\alpha$ if
      there is a definition ${\constr{DocDef}}(\delta,s)$ in $\Delta$
      where $\alpha$ is either the term address of the pure object $s$ or
      of the object $s$ of type $\delta$ in the theory with theory identifier $h$.
\item We say $(h,\Delta)$ {\defin{creates the proposition}} at $\alpha$ if
      there is a proof ${\constr{DocPfOf}}(s,\cD)$ in $\Delta$
      where $\alpha$ is either the term address of the pure proposition $s$ or
      of the proposition $s$ in the theory with theory identifier $h$.
\item We say $(h,\Delta)$ {\defin{creates the negated proposition}} at $\alpha$ if
      there is a proof ${\constr{DocPfOf}}(s,\cD)$ in $\Delta$
      and a proposition $t$ where $s$ is either $\neg t$ or $t\to\bot$\footnote{Here $\bot$ is $\forall_o x_0$ and $\neg$ is $\lambda_o x_0\to\bot$.}
      and
      $\alpha$ is the term address 
      of the proposition $t$ in the theory with theory identifier $h$.\footnote{We do not consider negated pure propositions. Negated propositions only need to be considered for collection of bounties by disproving a conjecture. Conjectures only make sense for propositions in a theory. (Note that every proposition is provable in an inconsistent theory, and so for every pure proposition there is some theory in which the proposition is provable.)}
\end{itemize}

We extend these definitions from single documents to transaction outputs (which
may publish several documents).

Let $o$ be an {\type{addr\_preasset}} list (transaction outputs)
and $\alpha$ be a term address.
\begin{itemize}
\item We say $o$ {\defin{creates the object}} at $\alpha$
if there is some 
$$(\delta,(\omega,{\constr{DocPublication}}(\gamma,\nu,h,\Delta)))\in o$$
where $(h,\Delta)$ creates the object at $\alpha$.
\item We say $o$ {\defin{creates the proposition}} at $\alpha$
if there is some 
$$(\delta,(\omega,{\constr{DocPublication}}(\gamma,\nu,h,\Delta)))\in o$$
where $(h,\Delta)$ creates the proposition at $\alpha$.
\item We say $o$ {\defin{creates the negated proposition}} at $\alpha$
if there is some 
$$(\delta,(\omega,{\constr{DocPublication}}(\gamma,\nu,h,\Delta)))\in o$$
where $(h,\Delta)$ creates the negated proposition at $\alpha$.
\end{itemize}

Ownership of an object, proposition or negated proposition will be originally
justified by the creation of the object, proposition or negated proposition.
We need a notion of {\defin{support}} for this purpose.
Ownership preassets
are those of the form
${\constr{OwnsObj}}(\beta,p)$,
${\constr{OwnsProp}}(\beta,p)$
and
${\constr{OwnsNegProp}}$.
Let $o$ be an {\type{addr\_preasset}} list (transaction outputs)
and $\alpha$ be a term address.
We define when $o$ supports an ownership preasset at $\alpha$
by considering the three kinds of preassets.
\begin{itemize}
\item We say $o$ {\defin{supports}} ${\constr{OwnsObj}}(\beta,p)$ at $\alpha$
      if $o$ creates the object at $\alpha$.
\item We say $o$ {\defin{supports}} ${\constr{OwnsProp}}(\beta,p)$ at $\alpha$
      if $o$ creates the proposition at $\alpha$.
\item We say $o$ {\defin{supports}} ${\constr{OwnsNegProp}}$ at $\alpha$
      if $o$ creates the negated proposition at $\alpha$.
\end{itemize}

These notions will be used when we give the conditions
for a ledger tree to support a transaction.
In terms of the code, there is no single function checking if $o$ support $u$ at $\alpha$.
The functions
{\func{output\_creates\_objs}},
{\func{output\_creates\_props}},
and
{\func{output\_creates\_neg\_props}}
can be used to obtain term addresses which are created as objects, propositions
and negated propositions.
When we need to check for support in {\func{ctree\_supports\_tx\_2}}
in the module {\module{ctre}}
we will already have the values returned by
{\func{output\_creates\_objs}},
{\func{output\_creates\_props}},
and
{\func{output\_creates\_neg\_props}}
and will make use of these values at the time.

\section{Transactions}

The module \module{tx} defines a type {\type{tx}} for transactions,
a type {\type{stx}} for signated transactions
and functions for testing the validity of transactions and signed transactions.
Here {\defin{validity}} of a transaction refers only to properties that
can be checked without reference to the state of the ledger.
For properties that require the ledger state, we will speak of {\defin{support}}
for a transaction (see Chapter~\ref{chap:ctre}).
There is a slight exception, however. We check validity of input signatures 
({\func{check\_tx\_in\_signatures}}) of
a transaction relative to a list of assets being spent.
These assets would need to be looked up in the ledger
since the transaction only mentions the asset identifiers.

{\bf{Note:}} Unit tests for the {\module{tx}} module are in {\file{txunittests.ml}}
in the {\dir{src/unittests}}
directory in the {\branch{testing}} branch.
These unit tests give a number of examples demonstrating how the functions described below should behave.

{\bf{Note:}} The Coq module {\coqmod{Transactions}} is intended to correspond to {\module{tx}}.
The Coq types {\coqtype{Tx}} and {\coqtype{sTx}} correspond to the types {\type{tx}} and {\type{stx}} in the OCaml version.
Readers can examine the formal properties proven in {\coqmod{Transactions}} to have a better
idea of what properties corresponding OCaml functions should satisfy.
For more information, see~\cite{White2015b}.

The type {\type{tx}} of transactions is simply defined as a pair of
an {\type{addr\_assetid}} list (a list of pairs of addresses and hash value asset ids)
and an {\type{addr\_preasset}} list (a list of addresses associated with an obligation and a preasset).
\begin{verbatim}
type tx = addr_assetid list * addr_preasset list
\end{verbatim}

The type {\type{stx}} of signed transactions is a transaction associated
with two lists of generalized signatures ({\type{gensignat}}).
\begin{verbatim}
type stx = tx * (gensignat list * gensignat list)
\end{verbatim}
The first list gives ``input'' signatures and the second list gives the ``output'' signatures.
The ``input'' signatures are required to spend or move the assets in the input.
The ``output'' signatures are for the authors of publications.
(Without these ``output'' signatures, a plagiarist could create his own transaction
with someone elses publications and use his transaction to assign ownership of
new objects and propositions.)

The serialization and deserialization functions are
{\serfunc{seo\_tx}},
{\serfunc{sei\_tx}},
{\serfunc{seo\_stx}}
and
{\serfunc{sei\_stx}}.

We briefly describe the following exported functions:
\begin{itemize}
\item {\func{hashtx}} hashes a transaction, giving the identifier for the transaction.
Note that this does not depend on signatures, and so transaction malleability is not an issue.
\item {\func{tx\_inputs}} is a projection function giving the input list of a transaction.
\item {\func{tx\_outputs}} is a projection function giving the output list of a transaction.
\item {\func{no\_dups}} is simply a helper function to ensure a list is duplicate free.\footnote{This is simply exported because the same function is required in the {\module{block}} module to ensure that no transaction is listed more than once in a block. As it has nothing to do with transactions, it should be moved to a more generic module imported by both {\module{tx}} and {\module{block}}.}
\item {\func{tx\_inputs\_valid}} takes a transaction input list
and checks that there is at least one input and there are no duplicate inputs.
\item {\func{tx\_outputs\_valid}} takes a transaction output list
and checks that it is valid in that the following conditions hold:
\begin{enumerate}
\item At most one owner (as an object or proposition) is declared for each term address.\footnote{Note that it is legal for one term address to obtain an owner as an object and another owner as a proposition. In fact, this should be common for pure terms. For example, the term $\forall_o x_0\to x_0$ (or $\top$) will have
a hash root $h_\top$. This can be both defined and owned as an object as well as proven and owned as a proposition.}
The function checking this is {\func{tx\_outputs\_valid\_one\_owner}}.
\item Each preasset is sent to an appropriate kind of address.
Ownership preassets are sent to term addresses
and publications and markers are sent to publication addresses.\footnote{This should probably be extended to ensure currency units and rights are only sent to pay addresses and bounties are only sent to term addresses.}
The function checking this is {\func{tx\_outputs\_valid\_addr\_cats}}.
\end{enumerate}
\item {\func{tx\_valid}} checks that both the inputs and outputs are valid in the sense above.
\item {\func{tx\_signatures\_valid}} takes a block height $b$, an asset list
and a signed transaction and checks that the input signatures and output signatures
are valid.
The work is partitioned in {\func{check\_tx\_in\_signatures}}
to check the input signatures
and {\func{check\_tx\_out\_signatures}}
to check the output signatures.
\begin{itemize}
\item {\func{check\_tx\_in\_signatures}} ensures that for each input
(except those spending markers and bounties)
there is an appropriate input signature. Here there are two possibilities.
There could be a signature permitting the spending of the asset ({\func{check\_spend\_obligation}})
or a signature permitting the movement of the asset ({\func{check\_move\_obligation}}).
A signature permitting the spending of the asset is either a signature
by the pay address in the obligation (assuming the appropriate block height has been reached)
or by the address where the asset is held if there is no obligation.\footnote{The obligation {\val{None}} defaults
to being the address where the asset is held with no block height requirement.}
A signature permitting the movement of the asset is by the address where the asset is held (assuming it is
a pay address) and is only allowed if there an output with exactly the same obligation and preasset
as the asset in question.
Essentially this allows the ``movement'' of an asset out of an address to a new address.\footnote{This could mitigate the effect of someone ``spamming'' someone else's address with unwanted and unowned assets.}
\item {\func{check\_tx\_out\_signatures}} ensures the authors of all publications
have signed the transaction. Note that the asset list is not required here.
\end{itemize}
\item {\func{txout\_update\_ottree}} takes a transaction output list
and uses it to update the current (possibly empty) {\type{ttree}} by including any new
theories created by publishing theory specifications.
\item {\func{txout\_update\_ostree}} takes a transaction output list
and uses it to update the current (possibly empty) {\type{stree}} by including any new
signatures created by publishing signature specifications.
\end{itemize}

