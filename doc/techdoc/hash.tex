The modules {\module{hashaux}}, {\module{sha256}}, {\module{ripemd160}}, {\module{hash}}
and {\module{htree}} contain code for cryptographic hashing functions.
The two hashing functions supported are {\tt{SHA256}}~\cite{sha256} and {\tt{RIPEMD-160}}~\cite{ripemd160}.
The {\tt{RIPEMD-160}} code only supports hashing a 256 bit input and is
assumed to be called on the result of applying {\tt{SHA256}}.

Profiling suggests that the hashing functions are the biggest computational bottleneck
in Qeditas. Improvements to the code described here could make Qeditas run significantly faster.

\section{Auxiliary Functions}

The module {\module{hashaux}} implements a few helper functions needed by both hashing functions.
\begin{itemize}
\item {\func{hexsubstring\_int32}} takes a string of hexadecimal digits and a position. The 8 characters starting at the position are interpreted as a 32-bit integer (big endian).
\item {\func{int32\_hexstring}} takes a string buffer and a 32-bit integer and adds 8 hexadecimal digits to the buffer representing the integer (big endian).
\item {\func{big\_int\_sub\_int32}} takes a big integer $x$ and an integer $i$ and returns the 32-bit integer resulting from shifting away $i$ bits of $x$ (i.e., dividing by $2^i$) and then taking the 32 least significant bits (i.e., modulo $2^{32}$).
\item {\func{int32\_big\_int\_bits}} takes a 32-bit integer $x$ and an integer $i$ and returns the big integer resulting from shifting $x$ forward by $i$ bits (i.e., multiplying by $2^i$).
\item {\func{int32\_rev}} takes a 32-bit integer of the form $b_3 2^{24} + b_2 2^{16} + b_1 2^{8} + b_0$
and returns the reversed 32-bit integer $b_0 2^{24} + b_1 2^{16} + b_2 2^{8} + b_3$.
\end{itemize}

\section{Sha256}

The module {\module{sha256}} defines a type {\type{md256}} (message digest of 256 bits)
as a product of 8 32-bit integers.
(The type {\type{md256}} is also sometimes used to represent other 256-bit numbers, such as the
$x$ or $y$ component of a public key.)
There is also an array {\var{currblock}} of 16 32-bit integers.
Various other arrays are used internally and not exposed by the interface.

The following functions are defined:

\begin{itemize}
\item {\func{sha256init}} initializes the state to begin performing the hashing operation.
\item {\func{sha256round}} performs one round of the hashing operation.
\item {\func{getcurrmd256}} returns the current {\type{md256}} (extracted from the internal array {\var{currhashval}}).
\item {\func{sha256str}} returns the result of hashing a given string with {\tt{SHA256}}.
\item {\func{sha256str}} returns the result of double hashing a given string with {\tt{SHA256}}.
\item {\func{md256\_hexstring}} converts a 256-bit message digest to the corresponding hexadecimal string.
\item {\func{hexstring\_md256}} converts a hexadecimal string to the 256-bit message digest to the corresponding hexadecimal string.
\item {\func{md256\_big\_int}} converts a 256-bit message digest to the corresponding big integer.
\item {\func{big\_int\_md256}} converts a big integer (assuming it is less than $2^{256}$) to the 256-bit message digest to the corresponding hexadecimal string.
\item {\func{seo\_md256}} serializes a 256-bit message digest.
\item {\func{sei\_md256}} deserializes a 256-bit message digest.
\end{itemize}

\section{Ripemd160}

The module {\module{ripemd160}} implements {\tt{RIPEMD-160}} restrict to 256-bit message digests as inputs.
The module defines a type {\type{md}} (message digest of 160 bits)
as a product of 5 32-bit integers.

The following functions are defined:

\begin{itemize}
\item {\func{ripemd160\_md256}} returns the result of hashing a given 256-bit message digest with {\tt{RIPEMD-160}}.
\item {\func{md\_hexstring}} converts a 160-bit message digest to the corresponding hexadecimal string.
\item {\func{hexstring\_md}} converts a hexadecimal string to the 160-bit message digest to the corresponding hexadecimal string.
\end{itemize}

\section{Hash}

The module {\module{hash}} is important. It defines a type {\type{hashval}} as 5 32-bit integers (representing a 160-bit hash value).

A function {\func{hash160}} takes an arbitrary string to the result of hashing first by {\tt{SHA256}} and then by {\tt{RIPEMD-160}}.
The type {\type{hashval}} is implemented the same way as the type {\type{md}} in the module {\module{ripemd160}}. If they were defined differently, the function {\func{hash160}} would be ill-typed.

There are a variety of functions for creating, using and combining hash values.
The following functions 
\begin{itemize}
\item {\func{hashval\_bitseq}} converts a hash value to a list of 160 booleans.
\item {\func{hashval\_hexstring}} converts a hash value to a string of 40 hexadecimal digits.
\item {\func{hexstring\_hashval}} converts a string of 40 hexadecimal digits to a hash value.
\item {\func{printhashval}} prints a hash value as 40 hexadecimal digits.
\item {\func{hashval\_rev}} performs a bytewise reversal of the hash value.\footnote{This seems to be unused.}
\item {\func{hashval\_big\_int}} converts a hash value to a big integer.
\item {\func{big\_int\_hashval}} converts a big integer to a hash value.
\item {\func{seo\_hashval}} serializes hash values.
\item {\func{sei\_hashval}} deserializes hash values.
\end{itemize}
The following functions create (effectively) unique hash values from given input.
Internally in each case the value being hashed is prefixed with a distinct 32-bit
integer so that the hash value given by different functions will be unique.
For example, {\func{hashint32}} prefixes the 32-bit integer with the 32-bit integer $1$
while {\func{hashint64}} prefixes the 64-bit integer with the 32-bit integer $2$.
\begin{itemize}
\item {\func{hashint32}} hashes a 32-bit integer.
\item {\func{hashint64}} hashes a 64-bit integer.
\item {\func{hashpair}} hashes a pair of hashes.
\item {\func{hashpubkey}} hashes a public key, given as two {\type{md256}} values.
\item {\func{hashpubkeyc}} hashes a compressed public key, given by a boolean (indicating if $y$ is even or odd) and one {\type{md256}} values (giving $x$).
\item {\func{hashtag}} combines a hash value with a 32-bit integer to create a different hash value. This
is used when we wish to ensure later data structures create unique hash values.
\item {\func{hashlist}} hashes a list of hash values. This could be implemented by a simple recursion using {\func{hashpair}}, but this would be inefficient. Instead the list is iterated over with {\func{sha256round}} being called when appropriate.
\item {\func{hashfold}} is given a function $f$ which returns a hashval for a given input and a list of appropriate inputs for $f$ and iteratively calls $f$ on the components of the list while performing {\func{sha256round}} to compute a hash value for the list of hash values computed by $f$ over the list.
\item {\func{hashbitseq}} takes a list of booleans and creates a hash values.
The naive way of doing this using {\func{hashlist}} would be too inefficient.
Instead the booleans are treated as 32-bit integers by considering them in groups of 32.
\end{itemize}
Sometimes optional hash values are used. This is important, for example, when we want to have an ``empty'' hash value $\bot$ corresponding to the hash of some ``empty'' data.
\begin{itemize}
\item {\func{ohashlist}} takes a list of hash values and computes an optional hash value.
The optional hash value is $\bot$ if and only if the input is the empty list.
\item {\func{hashopair}} takes two optional hash values and returns an optional hash value.
The output is $\bot$ if and only if both inputs were $\bot$.
\item {\func{hashopair1}} takes a hash value $x$ and an optional hash value $y$
and returns a hash value (known to not be $\bot$).
{\func{hashopair1}} is essentially the special case of {\func{hashopair}} where the first value is known not to be $\bot$.
\item {\func{hashopair2}} takes an optional hash value $x$ and a hash value $y$
and returns a hash value (known to not be $\bot$).
{\func{hashopair2}} is essentially the special case of {\func{hashopair}} where the second value is known not to be $\bot$.
\end{itemize}

In addition, various types of addresses are defined.
Fundamentally there are four kinds of addresses:
{\type{p2pkhaddr}} (pay to public key hash addresses), {\type{p2shaddr}} (pay to script hash addresses),
{\type{termaddr}} (term addresses) and {\type{pubaddr}} (publication addresses).
Each of these types is defined the same way as hash values (as 5 32-bit integers)
and so an object of one of these types can be used as an object of another.
\begin{itemize}
\item {\type{p2pkhaddr}} A pay to public key hash addresses is the hash value
obtained by hashing a public key. The intention is that the holder of the corresponding private key
can sign transactions related to the address.
The code for checking such signatures is in the module {\module{signat}}.
\item {\type{p2shaddr}} A pay to script hash address is the hash value
obtained by hashing a script.\footnote{Qeditas uses essentially the same scripting language as Bitcoin as of early 2015. Two Bitcoin operations are not supported: OP\_SHA1 and OP\_RIPEMD160.}
Such a script can act as a generalized signature in the following sense:
the script is executed and if the result is $1$ then the generalized signature is accepted.
This is a ``generalized signature'' since some of the script operations check a signature.
The code for executing scripts and checking generalized signatures is in the module {\module{script}}.
\item {\type{termaddr}}
Term addresses are hash values obtained in one of three ways:
\begin{enumerate}
\item A term address may be the hash root of a closed simply typed term $t$.
This is the global (theory independent) term address of $t$.
\item Given a theory $T$ and a closed term $t$ which has type $\tau$ in the theory $T$,
the combined hash of $T$, hash root of $t$ and the hash of $\tau$
gives a term address.\footnote{The combined hash is again hashed with a tag with $32$ to avoid the possibility that the combined hash value is the same as a different kind of term address.}
This is the address of the term $t$ in the theory $T$.
\item Given a theory $T$ and a closed proposition $t$,
the combined hash of $T$ and hashroot of $t$
gives a term address.\footnote{The combined hash is again hashed with a tag with $33$ to avoid the possibility that the combined hash value is the same as a different kind of term address.}
\end{enumerate}
Ownership information about a term or proposition (either globally or as part of a theory)
is stored at corresponding term address.
The author of the first document published which defines a term or proves a proposition can and must
also supply ownership information. This ownership information determines the conditions under
which the term or proposition can be imported into future documents.
Term addresses corresponding to terms or propositions within a theory are also used
to ensure terms have the correct type (without needing to repeat the definition) in the theory
and to ensure propositions are already known (without needing to repeat a proof).
\item {\type{pubaddr}}
A publication address corresponds to the hash root of a published document, theory specification or signature specification.
\end{itemize}

In addition to the four basic kinds of addresses, there are two other types of addresses:
\begin{itemize}
\item {\type{payaddr}}
The type {\type{payaddr}} (of pay addresses) is the disjoint sum of the types {\type{p2pkhaddr}} and {\type{p2shaddr}}.
This is implemented by taking a boolean along with 8 32-bit integers.
The 8 32-bit integers is a hash value representing either a {\type{p2pkhaddr}}
or a {\type{p2pkhaddr}}.
The boolean is {\val{false}} if the hash value represents a {\type{p2pkhaddr}},
and is {\val{true}} if the hash value represents a {\type{p2shaddr}}.
\item {\type{addr}}
The type {\type{addr}} (of addresses) is the disjoint sum of the four types
{\type{p2pkhaddr}}, {\type{p2shaddr}},
{\type{termaddr}} and {\type{pubaddr}}.
This is implemented by taking an integer $i\in\{0,1,2,3\}$ along with 8 32-bit integers.
The 8 32-bit integers is a hash value representing either a {\type{p2pkhaddr}}, a {\type{p2shaddr}},
a {\type{termaddr}} or a {\type{pubaddr}}.
If $i=0$, the hash value represents a {\type{p2pkhaddr}}.
If $i=1$, the hash value represents a {\type{p2shaddr}}.
If $i=2$, the hash value represents a {\type{termaddr}}.
If $i=3$, the hash value represents a {\type{pubaddr}}.
\end{itemize}

The following functions operate on addresses.
\begin{itemize}
\item {\func{hashval\_p2pkh\_payaddr}} gives the pay address corresponding to a hash value interpreted as a pay to public key hash addresses.
\item {\func{hashval\_p2sh\_payaddr}} gives the pay address corresponding to a hash value interpreted as a pay to script hash address.
\item {\func{hashval\_p2pkh\_addr}} gives the address corresponding to a hash value interpreted as a pay to public key hash addresses.
\item {\func{hashval\_p2sh\_addr}} gives the address corresponding to a hash value interpreted as a pay to script hash address.
\item {\func{hashval\_term\_addr}} gives the address corresponding to a hash value interpreted as a term address.
\item {\func{hashval\_pub\_addr}} gives the address corresponding to a hash value interpreted as a publication address.
\item {\func{addr\_bitseq}} returns a list of 162 booleans corresponding to an address, where the first two booleans determine which kind of address it is and the remaining 160 are the hash value.
\item {\func{bitseq\_addr}} returns an address given a list of 162 booleans.
\item {\func{p2pkhaddr\_payaddr}} converts a pay to public key hash address (a hash value) to a pay address by indicating it is a pay to public key hash address.
\item {\func{p2shaddr\_payaddr}} converts a pay to public key hash address (a hash value) to a pay address by indicating it is a pay to script hash address.
\item {\func{p2pkhaddr\_addr}} converts a pay to public key hash address (a hash value) to an address by indicating it is a pay to public key hash address.
\item {\func{p2shaddr\_addr}} converts a pay to public key hash address (a hash value) to an address by indicating it is a pay to script hash address.
\item {\func{payaddr\_addr}} converts a pay address to an address. In practice this simply means converting the first component from a boolean to an integer ({\val{false}} to $0$ and {\val{true}} to $1$).
\item {\func{termaddr\_addr}} converts a term address (a hash value) to an address by indicating it is a term address.
\item {\func{pubaddr\_addr}} converts a publication address (a hash value) to an address by indicating it is a publication address.
\item {\func{payaddr\_p}} checks if an address is a pay address.
\item {\func{p2pkhaddr\_p}} checks if an address is a pay to public key hash address.
\item {\func{p2shaddr\_p}} checks if an address is a pay to script hash address.
\item {\func{termaddr\_p}} checks if an address is a term address.
\item {\func{pubaddr\_p}} checks if an address is a publication address.
\item {\func{hashaddr}} hashes the address creating a unique hash value. (This is different from the underlying hash value of the address since the prefix is included before rehashing.)
\item {\func{hashpayaddr}} performs the same operation of {\func{hashaddr}} but only on pay addresses.
\item {\func{hashtermaddr}} performs the same operation of {\func{hashaddr}} on term addresses.
\item {\func{hashpubaddr}} performs the same operation of {\func{hashaddr}} on publication addresses.
\item {\func{seo\_addr}} serializes an address.
\item {\func{sei\_addr}} deserializes an address.
\item {\func{seo\_payaddr}} serializes a pay address.
\item {\func{sei\_payaddr}} deserializes a pay address.
\item {\func{seo\_termaddr}} serializes a term address.
\item {\func{sei\_termaddr}} deserializes a term address.
\item {\func{seo\_pubaddr}} serializes a publication address.
\item {\func{sei\_pubaddr}} deserializes a publication address.
\end{itemize}

\section{Htree}

The module {\module{htree}} defines a polymorphic type {\type{htree}}
which stores data in a tree indexed by a list of booleans.
In practice the list of booleans comes from a hash value.
The following functions are exported:
\begin{itemize}
\item {\func{htree\_lookup}} is given a boolean list and an {\type{htree}}
and returns the data if found and {\val{None}} if it is not found.
\item {\func{htree\_create}} is given a boolean list $\overline{b}$ and data $x$
and returns a new {\type{htree}} with only this single entry ($x$ at position $\overline{b}$).
\item {\func{htree\_insert}} is given an {\type{htree}}, a boolean list and data $x$
and returns the result of inserting the data into the {\type{htree}} ($x$ at position $\overline{b}$).
This will shadow data already at position $\overline{b}$ if there were any. (In practice this should never
happen since $\overline{b}$ should be obtained from a hash value determined by $x$.)
\item {\func{ohtree\_hashroot}} computes an optional hash value
given a function $f$ (which computes optional hash values for members of the underlying type),
the current depth $n$
and an optional {\type{htree}}.
This is essentially the Merkle root of the {\type{htree}}.
\end{itemize}

In practice {\type{htree}} is used in two ways: one is to story theories
and the other is to store theory-specific signatures.\footnote{Here {\em{signature}} is used
to mean a collection of constants, definitions and known propositions and should not
be confused with cryptographic signatures.}
These specific uses will be discussed in Chapter~\ref{chap:math}
where the module {\module{mathdata}} is considered.
