The file {\file{qeditas.ml}} is
used to create an executable
{\exec{qeditas}}.
This executable starts several threads for networking, staking (optionally)
and finally for a console-style interface.

To understand what the executable {\exec{qeditas}} does,
see the code near the end of the file {\file{qeditas.ml}}.
We describe the tasks {\exec{qeditas}} executes briefly.
The function {\func{initialize}}
is called first and is described in Section~\ref{sec:init}.
Next the function {\func{initnetwork}} is called,
see Section~\ref{sec:initnetwork}.
If the configuration variable {\var{staking}} is set to {\val{true}},
then a thread is created which calls the function {\func{stakingthread}},
see Chapter~\ref{chap:stk}.
Finally, a main loop begins for reading and processing commands
from the console begins,
see Section~\ref{sec:mainloop}.

\section{Initialization}\label{sec:init}

The {\func{initialize}} function begins by checking the command line arguments for an option
of the form \verb+-datadir=...+
which would override the default data directory (usually {\dir{.qeditas}}
in the user's home directory).
It then reads the config file {\file{qeditas.conf}} in the data directory.
This may override some default values in the {\module{config}} module.
It then reads the other command line arguments which may again override
some values in the {\module{config}} module.
If the {\var{testnet}} configuration variable is {\val{true}} (which is mandatory at the moment),
then the {\dir{testnet}} subdirectory of the main data directory is used.
This implies the testnet will have its own databases, its own wallet, and so on.
The database directory is set and configured and a log file is opened.

If there is a {\file{.lock}} file in the data directory,
then Qeditas exits to prevent two instances of Qeditas from using the same data directory.
Otherwise, Qeditas creates a {\file{.lock}} file in the data directory
and sets the variable {\var{exitfn}} to a function which calls {\func{saveknownpeers}}
(to remember the peers discovered during the session) and remove the {\file{.lock}} file.
At this point, the only way Qeditas should exit is via a call to {\var{exitfn}}.

The database is initialized by calling {\func{dbconfig}} with a {\dir{db}} subdirectory of the data directory
and then calling {\func{dbinit}} on each database.

Next the {\file{log}} file is opened (via {\func{openlog}}).

The code then checks if either the {\var{seed}} variable in {\module{config}}
has been set to a nonempty string. (It should be a 40 character hex string.)
The current code in {\module{setconfig}} sets {\var{seed}} to
\begin{verbatim}
68324ba252550a4cb02b7279cf398b9994c0c39f
\end{verbatim}
unless it is specifically set in the configuration file or on the command line.
The value above is the last 20 bytes of the hash of Bitcoin block 378800,
and was a value included only for testing purposes.
The intention was to choose a Bitcoin block height roughly one week in the future
when the time came for Qeditas to launch. The last 20 bytes of that block hash would
be the value for {\var{seed}}. The purpose of this value is to initialize the
current stake modifier and the future stake modifier (see {\func{set\_genesis\_stakemods}}).
These stake modifiers affect which assets will stake within the first 512 blocks.
In particular it affects the genesis block (at Qeditas block height).
For the launch to be fair, these stake modifiers should not be predictable before launch.
This goal could be accomplished in other ways.
The possibility is left open in the future that {\var{seed}} is not set but
that a checkpoint has been set so that a node can begin following the block chain
from that checkpoint. (The full history is not required. Each ledger tree contains
the full information required to continue.)

The function {\func{initblocktree}} from the {\module{blocktree}} module
is called. In particular, this processes all currently known headers in the database
to build a block tree and determine the current best node (and hence block chain).

If the {\var{testnet}} configuration variable is set, then the difficulty is decreased significantly
(setting {\var{genesistarget}} to $2^{200}$ so that finding a hit to stake is not difficult
and setting {\var{max\_target}} to $2^{208}$).

Next the wallet and the transaction pool are loaded using {\func{load\_wallet}} and {\func{load\_txpool}} from the {\module{commands}} module (see Chapter~\ref{chap:commands}).

A random 64-bit nonce is generated in order to prevent
the node from connecting to itself.
The variable {\var{this\_nodes\_nonce} from the {\module{net}} module
is set to the nonce.

\section{Initialization of the Network}\label{sec:initnetwork}

The {\func{initnetwork}} function starts several threads to
handle connections to the network.
First, if the configuration variable {\var{ip}}
has been set to an IP-address, then
a listener socket is opened and a thread
is started to listen for incoming connections (using
the function {\func{netlistener}}).
Then the function {\func{netseeker}} is called which
loads the known peers
and calls {\func{netseeker\_loop}}.
The loop tries to connect to peers
every 10 minutes, unless the maximum number of connections has been reached.

Each time a connection is created, threads are created for 
listening to the peer and for sending messages to the peer.
Most of this networking code is in the {\module{net}} module
which is described in Chapter~\ref{chap:net}.

\section{Main Loop}\label{sec:mainloop}

This main loop prints a prompt (see the configuration variable {\var{prompt}}),
reads a line and sends this line to {\func{do\_command}}.
If the user presses CTRL-D, the {\exc{End\_of\_file}} exception is raised
resulting in the process exiting.
The process will also exit if the user issues the command {\command{exit}}.
A command may also raise the exception {\exc{Exit}} which will be silently ignored
and the loop continues, issuing prompt and waiting for the next command.
All other exceptions are displayed to the user before the loop continues.

The supported commands are not fully implemented.
The code for executing most commands is in the {\module{commands}} module
(see Chapter~\ref{chap:commands}).
Here are a few that are currently supported:
\begin{itemize}
\item {\command{exit}} - exit qeditas. (CTRL-D also exits.)
\item {\command{printassets}} {\it{[ledger root in hex]}} - print the assets held at the addresses in the wallet in the ledger with the given root. This only prints the visible assets. Some assets will be missing if the local approximation of the compact tree does not include the assets explicitly. If the ledger root is omitted, then the ledger root of the most recent best block is used.
\item {\command{importprivkey}} - import a private key in Qeditas WIF format.
\item {\command{importbtcprivkey}} - import a private key in Bitcoin WIF format.
\item {\command{importwatchaddr}} - import a Qeditas address for watching assets, without importing the corresponding private key.
\item {\command{importwatchbtcaddr}} - import a Bitcoin address as a Qeditas address for watching assets, without importing the corresponding private key.
\item {\command{importendorsement}} - import a Bitcoin signed message to allow a Qeditas private key (held in the wallet) to sign for a corresponding Bitcoin address. This is the safe way to claim assets in the initial distribution.
\item {\command{btctoqedaddr}} - give the Qeditas address that corresponds to a given Bitcoin address.
\item {\command{addnode}} {\it{ip:port}} {\it{[add|remove|onetry]}} - explicitly add or remove a connection to a node.
\item {\command{clearbanned}} - call {\func{clearbanned}} in {\module{net}} to clear all banned nodes from memory.
\item {\command{listbanned}} - list all banned nodes in {\var{bannedpeers}} from {\module{net}}.
\item {\command{bannode}} {\it{ip:port}} - call {\func{bannode}} in {\module{net}} to add the given node to the set of banned nodes.
\item {\command{getpeerinfo}} - list all current connections.
\item {\command{nettime}} - print the current network time and median skew.
\item {\command{printtx}} {\it{txid}} - call {\func{printtx}} in {\module{commands}}
      to print the tx with the given id.
\item {\command{printasset}} {\it{assetid}} - call {\func{printasset}} in {\module{commands}}
      to print the asset with the given id.
\item {\command{printhconselt}} {\it{hconseltroot}} - call {\func{printhconselt}} in {\module{commands}}
      to print the hcons element with the given root.
\item {\command{printctreeelt}} {\it{ctreeeltroot}} - call {\func{printctreeelt}} in {\module{commands}}
      to print the ctree element with the given root.
\item {\command{printctreeinfo}} {\it{ctreehashroot}} - call {\func{printctreeinfo}} in {\module{commands}}
      to print information about the ctree with the given root.
\item {\command{createsplitlocktx}} {\it{address assetid numouts lockheight fee [newaddress [newobligationaddress [ledgerroot]]]}} - 
      create a transaction with the given asset at the given address as the only input,
      with the number of outputs indicated
      where each output has the given lockheight.
      The values in the outputs are evenly divided (except possibly the last output) after the fee has been subtracted.
      If no {\it newaddress} is given, then the output is to the same address and the same address is used in the new obligations.
      If {\it newaddress} is given and {\it newobligationaddress} is not, then the outputs
      are to {\it newaddress} and {\it newaddress} is used in the obligations.
      If {\it newaddress} and {\it newobligationaddress} are given, then the outputs
      are to {\it newaddress} and {\it newobligationaddress} is used in the obligations.
      If {\it ledgerroot} is given, then the corresponding ctree is used to look up the asset for the input.
      (By default, the ledger root of the current best node is used.)
      Most of the work is done by the function {\func{createsplitlocktx}} in {\module{commands}}.\footnote{The purpose of this specialized command is to easily split one large asset into several smaller assets which can stake independently. Ultimately, there would need to be a more general command for creating transactions.}
\item {\command{signtx}} {\it{txinhex}} - call {\func{signtx}} in {\module{commands}} to (partially or completely) sign the transaction, giving the signed transaction in hex as output.
\item {\command{savetxtopool}} {\it{txinhex}} - call {\func{savetxtopool}} in {\module{commands}} to save the transaction to the local pool.
\item {\command{sendtx}} {\it{txinhex}} - call {\func{sendtx}} in {\module{commands}} to send the given transaction to peers.
\item {\command{bestblock}} - print information about the current best node ({\var{bestnode}} from {\module{blocktree}}).
\item {\command{difficulty}} - print the difficulty target of the current best node ({\var{bestnode}} from {\module{blocktree}}).
\item {\command{blockchain}} - print the header hash and ledger root of the last 1000 blocks of the current block chain.
\item {\command{dumpstate}} {\it{filename}} - dump information about the current state to the given file, as a text file. This is for debugging purposes.
\end{itemize}

