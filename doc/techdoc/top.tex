{\bf{Warning:}} The top level code is only partially written and may possibly be rewritten completely.
The information is this chapter is subject to changes in the near future.

The files
{\file{qeditascli.ml}} and {\file{qeditasd.ml}}
are used to create executables
{\exec{qeditasd}} and {\exec{qeditascli}}.
These executables are intended to provide a basic user interface.
In particular {\exec{qeditasd}} starts a ``daemon''
which connects to other nodes, interacts with other nodes,
handles a staking process (if staking is enabled)
and tries to publish transactions and blocks.
The executable {\exec{qeditascli}} provides a command line interface
which may (or may not) interact with {\exec{qeditasd}}.
The module {\module{commands}} handles commands and wallet information
that is used by both {\file{qeditascli.ml}} and {\file{qeditasd.ml}}.

\section{Files Instead of Sockets}

Originally the
two executables {\exec{qeditasd}} and {\exec{qeditascli}}
were intended to mimic the Bitcoin daemon and command line interface.
However, one of the goals was to have Qeditas run on the Tails Live CD (Debian) operating system.
It seems that (apparently) Tails does not allow local sockets to be opened.
As a consequence, files were used to handle some interaction between {\exec{qeditascli}}
and {\exec{qeditasd}}. For example, {\exec{qeditascli}} can be used to create transactions
which are then ``sent'' by placing them into a file where {\exec{qeditasd}} would read
and publish them. This is unlikely to be a good choice for how the processes should interact,
and it's possible there should simply not be two separate processes.
Another choice would be to have a single {\exec{qeditas}}
which handles the networking and staking in the background while providing a top level
(similar to the Bitcoin Core debug console) for user interaction.
Such a single executable would presumably work within Tails without requiring local sockets.
For other operating systems, perhaps a daemon/command line interface set-up communicating via
local sockets would still be appropriate.

\section{Commands}

The module {\module{commands}} is intended to support a variety of commands a user may need.
At the moment it only supports limited wallet and transaction creation commands.
Some state information is held in this module (although it likely should be moved elsewhere).
\begin{itemize}
\item {\var{walletkeys}} contains the private keys in the wallet.
More specifically it is a list of values of the form $(k,b,(x,y),w,h,a)$
where $k$ is the private key, $b$ is a boolean indicating if it is for the compressed public key,
$(x,y)$ is the public key, $w$ is the string base-58 WIF format, $h$ is the 20-byte hash value
corresponding to the p2pkh address and $a$ is the string base-58 Qeditas p2pkh address.
\item {\var{walletp2shs}} contains entries of the form $(h,a,\overline{b})$
where $h$ is the 20-byte hash value of a p2sh address, $a$ is the base-58 Qeditas p2sh address
and $\overline{b}$ is the sequence of bytes giving the script corresponding to $h$.
Note that this does not directly give a way of ``signing'' for the p2sh address.
\item {\var{walletendorsements}} contains the endorsements in the wallet.
In particular it is a list of values of the form $(\alpha,\beta,(x,y),b,\sigma)$
where $\alpha$ and $\beta$ are pay addresses,\footnote{Actually, in what is implemented we assume they are both p2pkh addresses. In principle endorsements involving p2sh addresses are supported by the code in {\module{sigant}} and {\module{script}}, but support has not been implemented in {\module{commands}}.}
$(x,y)$ is the public key for $\alpha$,
$b$ is a boolean indicating if $\alpha$ is the address for the compressed public key
and $\sigma$ is a signature for a
Bitcoin message of the form ``endorse $\beta$''
(or ``testnet:endorse $\beta$'' in the testnet)
signed with the private key for $\alpha$.
The private key for $\beta$ should be in {\var{walletkeys}}
and this private key along with the endorsement means the wallet can sign for $\alpha$.\footnote{The endorsement mechanism gives Bitcoin users a way to claim their part of the initial distribution without revealing their private keys.}
\item {\var{walletwatchaddrs}}
contains addresses to ``watch.'' 
\item {\var{stakingassets}}
contains a list of assets which the node can stake.
\item {\var{storagetrmassets}}
is intended contain a list of assets at term addresses which the node can use as proof-of-storage to improve the chances of staking.
Currently it is unused.
\item {\var{storagedocassets}}
is intended to contain a list of documents at publication assets which the node can use as proof-of-storage to improve the chances of staking.
Currently it is unused.
\item {\var{recenttxs}} is a hash table associating hash values (transaction ids)
with signed transactions. This is loaded and saved to the file {\file{recenttxs}}.
The intention is that this holds transactions which have been created but
are not yet published (and may only be partially signed).
\item {\var{txpool}} is a hash table associating hash values (transaction ids)
with signed transactions. This is loaded and saved to the file {\file{txpool}}.
The intention is that this holds transactions which have been published
but are not yet included in a block.
\end{itemize}

The following functions are for loading and saving the state in certain files.
\begin{itemize}
\item {\func{load\_currentframe}} loads the current ``frame'' which is used
to determine how the ledger tree is represented.
\item {\func{save\_currentframe}} saves the current ``frame.'' Various commands
could change the frame.
\item {\func{load\_recenttxs}} sets {\var{recenttxs}} by loading the contents fo {\file{recenttxs}}.
\item {\func{load\_txpool}} sets {\var{txpool}} by loading the contents fo {\file{txpool}}.
\item {\func{load\_wallet}} sets {\var{walletkeys}}, {\var{walletp2shs}},
{\var{walletendorsements}}
and {\var{walletwatchaddrs}}
by loading the file {\file{wallet}}.
\item {\func{save\_wallet}} saves the current wallet contents
(the values of {\var{walletkeys}}, {\var{walletp2shs}},
{\var{walletendorsements}}
and {\var{walletwatchaddrs}})
in {\file{wallet}}.
\item {\func{printassets}} prints the assets from the current ledger tree at the addresses
mentioned in the wallet.
\item {\func{btctoqedaddr}} parses a Bitcoin address (base-58 representation) and prints the corresponding Qeditas address (base-58 representation).
\item {\func{importprivkey}} imports a private key given in Qeditas WIF.
\item {\func{importbtcprivkey}} imports a private key given in Bitcoin WIF.
\item {\func{importendorsement}} imports an endorsement.
\item {\func{importwatchaddr}} imports a Qeditas address to watch.
\item {\func{importwatchbtcaddr}} imports a Bitcoin address in order to watch the corresponding Qeditas address.
\end{itemize}

\section{Command Line Interface}

The file {\file{qeditascli.ml}} has a partial implementation of a command line interface.
The value {\val{commhelp}} gives a list of some intended Qeditas commands
along with how many arguments the command may take and help strings.
The help contents can be viewed via the executable itself as follows:
\begin{verbatim}
./bin/qeditascli help
\end{verbatim}
(Note: this does not require the daemon to be running.)
Help for a specific command can be obtained as follows:
\begin{verbatim}
./bin/qeditascli help importendorsement
\end{verbatim}
Many of the commands implemented call functions from {\module{commands}}.

\section{Daemon}

To understand what the daemon executable {\exec{qeditasd}} does,
see the function {\func{main}} in the file {\file{qeditasd.ml}}.
We describe the tasks {\exec{qeditasd}} briefly.

\subsection{Initialization}

It begins by checking the command line arguments for an option
of the form \verb+-datadir=...+
which would override the default data directory (usually {\dir{.qeditas}}
in the user's home directory).
It then reads the config file {\file{qeditas.conf}} in the data directory.
This may override some default values in the {\module{config}} module.
It then reads the other command line arguments which may again override
some values in the {\module{config}} module.

The code then checks if either the {\var{seed}} variable in {\module{config}}
has been set to a nonempty string. (It should be a 40 character hex string.)
The current code in {\module{setconfig}} sets {\var{seed}} to
\begin{verbatim}
68324ba252550a4cb02b7279cf398b9994c0c39f
\end{verbatim}
unless it is specifically set in the configuration file or on the command line.
The value above is the last 20 bytes of the hash of Bitcoin block 378800,
and was a value included only for testing purposes.
The intention was to choose a Bitcoin block height roughly one week in the future
when the time came for Qeditas to launch. The last 20 bytes of that block hash would
be the value for {\var{seed}}. The purpose of this value is to initialize the
current stake modifier and the future stake modifier (see {\func{set\_genesis\_stakemods}}).
These stake modifiers affect which assets will stake within the first 512 blocks.
In particular it affects the genesis block (at Qeditas block height).
For the launch to be fair, these stake modifiers should not be predictable before launch.
This goal could be accomplished in other ways.

The possibility is left open in the future that {\var{seed}} is not set but
that a checkpoint has been set so that a node can begin following the block chain
from that checkpoint. (The full history is not required. Each ledger tree contains
the full information required to continue.)

If the {\var{testnet}} configuration variable is set, then the difficulty is decreased significantly
(setting {\var{genesistarget}} $2^{248}$ so that finding a hit to stake is not difficult
and setting {\var{max\_target}} $2^{255}$, practically as high as possible).
The ``current frame'' is loaded 
and {\var{localframe}} is set to it
and {\var{localframehash}} is set to its hash.
The frame is needed because a ledger root will, in general, correspond to multiple
ledger abbreviations since the abbreviation depends on the frame.
The function {\func{load\_root\_abbrevs\_index}}
loads the current information about the association of ledger hash roots with
frame hashes and ledger abbreviation hashes.
The ledger abbreviation hashes is the name of the file
in the {\dir{ctrees}} directory
where the corresponding {\type{ctre}} is stored.

A random seed is generated by reading 512 bits from {\file{/dev/random}} (or {\file{/dev/urandom}} for the testnet)
and this seed is used to generate random values in the future by hashing the value and taking
some of the bits when required.
{\bf{Warning:}} This is (probably) unsafe and should not be used in production.
Cryptographically safe random values should be obtained when required.
The code is the way it currently is simply because it was under development and
being tested.

The random value above is used to set {\var{this\_nodes\_nonce}} which prevents the node
from connecting to itself.

If {\var{ip}} and {\var{port}} are set, then a listener socket is opened (via {\func{openlistener}})
to listen for incoming connections.

If {\var{staking}} is set, then the wallet is loaded and the function {\func{start\_staking}}
starts a {\exec{qeditasstk}} process which independently searches for an asset that can stake.
This process is stopped when a hit is found and is restarted each time a decision is reached
to being staking on top of a new block.

The function {\func{sethungsignalhandler}} sets the signal handler for timeouts to raise the exception {\exc{Hung}}.
This is used to create functions that do not wait indefinitely for input or connections.

The function {\func{loadknownpeers}} loads a file with recent peers.
Then {\func{search\_for\_conns}} is called which tries to connect to these recent peers,
and, if no connections succeed, tries to connect to fallback nodes ({\func{getfallbacknodes}}).

Recent transactions and the transaction pool is loaded ({\func{load\_recenttxs}} and {\func{load\_txpool}}.

At this point the daemon is ready to enter the main process loop.

\subsection{Main Process Loop}

Currently the testnet sleeps for 1 second between many of the steps of the main loop.
This is simply to make the output reasonable while testing.

The next paragraphs are only relevant if {\var{staking}} is set to {\val{true}},
otherwise they can be skipped.

The process checks to see if a hit has been found (by reading from the
stdout of the {\exec{qeditasstk}} process). If a hit has been found, it is likely
for some timestamp some seconds in the future.
In preparation to stake, a block is created, possibly including transactions from the transaction pool.
The code checks that the block is valid and, if so, sets {\var{waitingblock}}
to a value that remembers when the hit will be valid and information about the new block.
Staking pauses until either a new block is received or until the time is reached
that the found hit will be valid.

If {\var{waitingblock}} is set and the current time is at least the time the hit is valid,
then the new block header is inserted into the sorted {\var{recentblockheader}} list
and the remaining block information is saved.
An inventory message with the block header, the block delta skeleton and the block delta
is broadcast to all connections.
Staking is restarted, presumably on top of this new block.

Assume there is no {\var{waitingblock}}.
If no staking process is running, then start a staking process.
If there is a staking process, check if it is staking on top of the block with the most
cumulative stake. If not, restart staking on top of the current best block.

From this point on in the main process loop, the value of {\var{staking}} is not relevant.

If there is a listener process, {\func{accept\_nohang}} checks (with a timeout of 0.1s)
if there is a connection to accept. If so, {\func{initialize\_conn\_accept}} begins the
handshake protocol.

For each preconnection on {\var{preconns}}, progress in the handshaking protocol is attempted.
The handshake protocol requires one process to send a {\constr{Version}} message,
with a {\constr{Verack}} sent back followed by a {\constr{Version}} message
and a final {\constr{Verack}}. This follows the Bitcoin handshake.
If the handshake fails, the connection is dropped and removed from {\var{preconns}}.
If the handshake succeeds, the connection is moved from {\var{preconns}} to {\var{conns}}.

We next turn to handling the connections on {\var{conns}}.
Here the connection state ({\type{connstate}}) plays an important role.

Each connection on {\var{conns}} is checked for incoming messages.

If there is an incoming message, {\func{handle\_msg}} is called to
perform the appropriate actions.
Assume there is no incoming message.
If some request was pending for more than 90s (e.g., a {\constr{Ping}} message
expecting a {\constr{Pong}} reply), then mark the close connection 
and mark it as no longer alive. If there have been no messages in 90 minutes,
send a {\constr{Ping}} and save this in {\var{pending}}.

Each connection that is no longer alive is removed from {\var{conns}}.

At the end of the main loop process, there are some checks to see if more connections should be searched
for. Also, {\func{handle\_orphans}} and {\func{handle\_delayed}}
are called in order to double check that the node is not missing a chain with more cumulative stake.
(Such chains were being missed during testing. It's unclear if 
{\func{handle\_orphans}} and {\func{handle\_delayed}} fixed the problem or not.)






