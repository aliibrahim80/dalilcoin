The modules \module{secp256k1}, \module{cryptocurr}, \module{signat} and \module{script}
contain code for cryptocurrency related operations.
In particular, \module{secp256k1} implements the elliptic curve {\tt{secp256k1}}~\cite{sec2final},
\module{cryptocurr} supports base 58 formats for private keys and addresses,
\module{signat} supports cryptographic signatures
and \module{script} supports the Bitcoin scripting language (mostly).

{\bf{Note:}} The unit tests for these modules are in {\file{basicunittests.ml}}
in the {\file{src/unittests}}
directory in the {\branch{testing}} branch.
These unit tests give a number of examples demonstrating how the functions described below should behave.

\section{Elliptic Curve Code}

The module \module{secp256k1} contains the operations for the corresponding elliptic curve~\cite{sec2final}.
Most of the code in this module was taken from the code for Egal~\cite{Brown2014}.
256-bit integers are represented using big integers (\type{big\_int}) from ocaml's {\tt{nums}} library.
A function {\func{evenp}} is defined and exposed since it is used elsewhere.

The 256-bit prime $p$ used in the elliptic curve is
exposed as the big integer {\val{\_p}}.
The following functions implement operations modulo $p$:
\begin{itemize}
\item {\func{add}} implements addition modulo $p$.
\item {\func{mul}} implements multiplication modulo $p$.
\item {\func{pow}} implements $x^i$ modulo $p$ where $x$ is a big integer and $i$ is an integer.
\item {\func{eea}} implements the Extended Euclidean Algorithm which is used to compute
multiplicative inverses modulo $p$.
\end{itemize}

Points on the elliptic curve are represented by element of type {\type{pt}}
which is defined to be an optional pair $(x,y)$ of big integers.
{\val{None}} represents the zero point (point at infinity).
The following functions are defined and exposed:
\begin{itemize}
\item {\func{addp}} implements addition of two points.
\item {\func{smulp}} implements scalar multiplication of a big integer to a point.
\end{itemize}

The base point $g$ on the curve (which generates the group) is exposed as {\val{\_g}}.
The order $n$ of $g$ is exposed as the big integer {\val{\_n}}.
The function {\func{curve\_y}} takes a boolean $e$ and a big integer $x$
and returns the big integer $y$ such that $(x,y)$ is a point on the curve
where $y$ is even if $e$ is true
and $y$ is odd if $e$ is false.

As usual, there are serialization functions {\serfunc{seo\_pt}} and {\serfunc{sei\_pt}} for
points. The serialization functions assume the components $x$ and $y$ of the point
are positive integers less than $2^{256}$.

\section{Cryptocurrency Operations}

The module {\module{cryptocurr}} implements
functions which convert private keys and addresses 
to and from base 58 representations.
\begin{itemize}
\item {\func{base58}} converts a big integer into a base 58 string.
\item {\func{frombase58}} converts a base 58 string to a big integer.
\item {\func{qedwif}} converts a big integer (private key) and a boolean (indicating if it is for a compressed address) to a base 58 string.
The Qeditas WIF format uses a two byte prefix of $5,8$ for compressed addresses
and $2,30$ for uncompressed addresses.
The result is that compressed WIFs start with the character {\tt{k}} and uncompressed WIFs start
with the character {\tt{K}}.
\item {\func{privkey\_from\_wif}} takes a Qeditas WIF string and returns the corresponding
private key (as a big integer) along with a boolean indicating if it is for the compressed address.
\item {\func{privkey\_from\_btcwif}} takes a Bitcoin WIF string and returns the corresponding
private key (as a big integer) along with a boolean indicating if it is for the compressed address.
This function is included to make it easy for people to import Bitcoin private keys corresponding to the
initial distribution.
\item {\func{pubkey\_hashval}} takes a non-zero public key (a pair $(x,y)$) and a boolean (indicating if the compressed address should be used)
and returns the $20$ bytes which result from hashing
either the compressed public key ($2$ with $x$ if $y$ is even; $3$ with $x$ if $y$ is odd)
or the uncompressed public key ($4$ with $x$ and $y$).
This hash value is also the corresponding pay to public key hash address.
\item {\func{hashval\_from\_addrstr}} takes a string with a Qeditas address and returns the underlying hash value.
\item {\func{hashval\_btcaddrstr}} takes a hash value and returns the corresponding Bitcoin address.
\item {\func{addr\_qedaddrstr}} takes a Qeditas address and returns a base 58 string representation of the address.
The prefix byte used is different for the four different kinds of addresses:
\begin{itemize}
\item Pay to public key hash addresses use a prefix byte of $58$
and so these Qeditas addresses begin with the character {\tt{Q}}.
\item Pay to script hash addresses use a prefix byte of $120$
and so these Qeditas addresses begin with the character {\tt{q}}.
\item Term addresses use a prefix byte of $66$
and so these Qeditas addresses begin with the character {\tt{T}}.
\item Publication addresses use a prefix byte of $56$
and so these Qeditas addresses begin with the character {\tt{P}}.
\end{itemize}
\item {\func{qedaddrstr\_addr}} takes a string with a base 58 Qeditas address and returns the Qeditas address.
\item {\func{btcaddrstr\_addr}} takes a string with a base 58 Bitcoin address (either p2pkh or p2sh) and returns the Qeditas address.
\end{itemize}
Some of the code in this module was taken from the code for Egal~\cite{Brown2014}.
Egal included {\tt{BIP 32}} code that isn't needed in Qeditas.
Egal relied on openssl to compute {\tt{SHA256}} and {\tt{ripemd160}} hashes, but Qeditas does this itself.

\section{Cryptographic Signature Checking}

The module {\module{signat}} implements functions for creating and verifying
cryptographic signatures over the elliptic curve.
A cryptographic signature (represented by the type {\type{signat}})
is a pair $(r,s)$ of big integers.
As usual, the functions ${\serfunc{seo\_signat}}$ and ${\serfunc{sei\_signat}}$
serialize and deserialize elements of type {\type{signat}}.
Let $n$ be the order of the group for {\tt{secp256k1}}.
\begin{itemize}
\item {\func{decode\_signature}} takes a base 64 string and returns $(i,c,(r,s))$
where $i\in\{0,1,2,3\}$ (``recid'') is a tag to help recover the public key from the signature
and what was signed,
$c$ (``fcomp'') is a boolean indicating if the signature is for a compressed public key
and $(r,s)$ is the cryptographic signature.
\item {\func{signat\_big\_int}} takes a big integer $e<n$ (in practice $e<2^{160}$),
a big integer private key $k<n$ and a random big integer $R<n$
and returns a signature $(r,s)$.
The signature $(r,s)$ signs $e$ with the private key $k$.
\item {\func{signat\_hashval}} is the same as {\func{signat\_big\_int}} except
it is given a hash value $h$ to sign instead of a big integer.
The implementation simply converts $h$ to a (160-bit) big integer using {\func{hashval\_big\_int}} and calls
{\func{signat\_big\_int}}.
The result is a signature $(r,s)$ signing $h$ with the given private key.
\item {\func{verify\_signed\_big\_int}} takes a big integer $e$, a point (public key) $(x,y)$
and a signature $(r,s)$
and returns a boolean indicating if $(r,s)$ is a valid
signature of $e$ by the private key corresponding to $(x,y)$.
\item {\func{verify\_p2pkhaddr\_signat}}
takes a big integer $e$, a {\type{p2pkhaddr}} $\alpha$ (equivalently, a hash value),
a signature $(r,s)$,
an integer $i\in\{0,1,2,3\}$
and a boolean $c$.
It uses $e$, $(r,s)$ and $i$ to recover a point on the curve using
{\func{recover\_key}}.
If {\func{recover\_key}} returns the zero point, then the signature is not valid and the boolean false is returned.
Otherwise, {\func{recover\_key}} returns a public key $(x,y)$.
The {\type{p2pkhaddr}} corresponding to $(x,y)$ (compressed if $c$ is true, uncompressed otherwise)
is computed using {\func{pubkey\_hashval}} and compared with $\alpha$.
If they are the same, then the signature is valid and the boolean true is returned.
Otherwise, the signature is not valid and the boolean false is returned.
\item {\func{verifybitcoinmessage}} is used to verify a bitcoin signed message returning a boolean (true if valid, false otherwise).
The inputs are a {\type{p2pkhaddr}} $\alpha$, $i\in\{0,1,2,3\}$, a boolean $c$, a cryptographic signature $(r,s)$
and a string $m$ (the message).
If Qeditas is running in the testnet, then the message is prefixed with {\tt{testnet:}}.
This allows people to sign, for example, endorsements which are valid on the testnet, but not on the mainnet.
The message is then modified the same way as the Bitcoin core client (essentially
including {\tt{Bitcoin Signed Message:}} and some characters indicating the length of this prefix and the length of the message).
The remainder of the work is performed by the internal {\func{verifymessage}} function:
The message is double {\tt{SHA256}} hashed and converted to a big integer $e$.
The public key is attempted to be recovered
using $e$, $(r,s)$ and $i$ using {\func{recover\_key}} (with false returned upon failure).
Assuming the public key $(x,y)$ is recovered, the final check verifies that the hash of the public key
(compressed if $c$ is true, uncompressed otherwise) is $\alpha$.
\item {\func{verifybitcoinmessage\_recover}} is used to verify a bitcoin signed message returning an optional public key $(x,y)$ (the corresponding public key if the signature is valid, the {\tt{None}} if not).
It behaves equivalently to {\func{verifybitcoinmessage}}
except upon success it returns the public key as ${\tt{Some}}(x,y)$
and returns {\tt{None}} upon failure.
In this case, internal {\func{verifymessage\_recover}} function is used.
\end{itemize}

There is also an internal function {\func{recover\_key}} which
computes a public key $(x,y)$
from a big integer $e$ (from the hash value of what was signed), a signature $(r,s)$ and a ``recid'' $i\in\{0,1,2,3\}$.
This should be the public key corresponding to the private key which was used to construct $(r,s)$
from $e$.

\section{Scripts and Generalized Signatures}

The module {\module{script}} implements the Bitcoin scripting language
(with the exceptions of OP\_SHA1 and OP\_RIPEMD160).
The main reason the Bitcoin scripting language is included is so p2sh
addresses in the Bitcoin snapshot can be redeemed.
Scripts are represented by lists of integers which should be bytes.
The following functions operate on scripts:
\begin{itemize}
\item {\func{hash160\_bytelist}} takes a script and computes its hash by
taking the SHA256 and then RIPEMD160.
The hash value returned should be interpreted as a {\type{p2shaddr}}.
The procedure is the same way Bitcoin computes p2sh addresses.
\item {\func{eval\_script}}
evaluates a script in context.
The inputs are a big integer $e$ (which corresponds to what is meant to be ``signed''),
the script $\overline{s}$ and two stacks.
The function returns the two stacks which result from evaluating the script.
\item {\func{verify\_p2sh}}
compares a script's hash against the given {\type{p2shaddr}}
and verifies that the script evaluates to ``true.''
A boolean is returned.
The inputs are a big integer $e$ (which corresponds to what is meant to be ``signed''),
a {\type{p2shaddr}} $\beta$ and a script $\overline{s}$.
The only way {\val{true}} will be returned is if the following occurs:
\begin{enumerate}
\item The script is evaluated and the top of the main stack is a script $\overline{s_1}$ which hashes to give $\beta$.
\item The script $\overline{s_1}$ is popped off the main stack and evaluated leaving something nonzero at the top of the main stack.
\end{enumerate}
\end{itemize}
The process of evaluating the script is more complicated than it is in Bitcoin.
The reason is that {\tt{OP\_CHECKSIG}} and {\tt{OP\_CHECKMULTISIG}} may be signatures
by endorsement. An endorsement may be, for example, an endorsement of a p2sh address
to a p2pkh address. In order to check the endorsement, another script must be checked
to be a valid ``signature'' of a different value (the hash of the endorsement message).
For this reason, the main functions that do the work: {\func{eval\_script}},
{\func{eval\_script\_if}},
{\func{checksig}},
{\func{checkmultisig}} and 
{\func{check\_p2sh}}
are mutually recursive.

In order to account for endorsements in a uniform way,
the type {\type{gensignat}} of ``generalized signatures'' is defined.
There are six constructors of type {\type{gensignat}} corresponding to 
six ways of making a signature:
\begin{itemize}
\item {\val{P2pkhSignat}} is an ordinary cryptographic signature $(r,s)$ corresponding
to a given public key $(x,y)$ which may or may not be compressed. (Note that the public key is explicitly given here and need not be recovered during signature checking.)
The public key corresponds to a {\tt{p2pkhaddr}} (again, one compressed and one uncompressed).
\item {\val{P2shSignat}} is a script and should be checked by calling {\func{verify\_p2sh}} above.
The script corresponds to a certain {\tt{p2shaddr}}. (Note that the correspondence is not
direct. The script itself is not hashed to obtain the {\tt{p2shaddr}}.
Instead the script should evaluate to yield a script which hashes
to the {\tt{p2shaddr}} at the top of the main stack.)
\item {\val{EndP2pkhToP2pkhSignat}} This is a p2pkh signature via a p2pkh endorsement.
This means that two public keys (and two booleans indicating compressed or uncompressed) are given
along with two signatures.
One of the signatures is of a Bitcoin message with the appropriate base 58 Qeditas pay to public key hash address, signed with the private key for the other public key.
The other signature is the signature of what should be signed.
\item {\val{EndP2pkhToP2shSignat}}
This is a p2sh signature via an endorsement a p2pkh endorsement.
That is, a signature of a Bitcoin message with the appropriate base 58 Qeditas pay to script hash
address is given, corresponding to the public key of the p2pkh address.
Also, a script corresponding to the p2sh address is given which ``signs'' what should be signed.
\item {\val{EndP2shToP2pkhSignat}}
This is a p2pkh signature via an endorsement a p2sh endorsement.
Here a script is given which ``signs'' the Bitcoin message with an endorsement to a base 58 Qeditas pay to public key hash address.
An ordinary cryptographic signature (corresponding to the p2pkh address)
signing what is to be ``signed'' is given.
\item {\val{EndP2shToP2shSignat}}
This is a p2sh signature via an endorsement a p2sh endorsement.
Here a script is given which ``signs'' the Bitcoin message with an endorsement to a base 58 Qeditas pay to script hash address.
A separate script corresponding to the other p2sh address is given which ``signs'' what should be signed.
\end{itemize}
As usual, there are two functions {\serfunc{seo\_gensignat}} and
{\serfunc{sei\_gensignat}} for serializing generalized signatures.

There is one important function for working with generalized signatures:
\begin{itemize}
\item {\func{verify\_gensignat}} takes a big integer $e$ (corresponding to the hash value
of what should be signed), a generalized signature
and an address, and returns a boolean indicating if the generalized signature
verifies that the owner of the address ``signed'' $e$.
The address should be either a pay to public key hash address or pay to script hash address.
(If a term address or publication address is given, {\val{false}} is returned.)
Each of the six cases is considered separately in the code, but the idea is clear:
check that the signature corresponds to the given address and
check that the signature ``signs'' $e$.
\end{itemize}
