The modules \module{config} and \module{setconfig}
are for customizing the configuration of Qeditas.
The \file{configure} script creates an OCaml file \file{config.ml}
setting default values for the variables exposed in the interface \file{config.mli}:
\begin{itemize}
\item \var{datadir} : the location of the main directory containing the local Qeditas configuration file, wallet file, and other data (usually \dir{.qeditas} in the user's home directory)
\item \var{testnet} : set to true if Qeditas is running on the testnet instead of the mainnet
\item \var{staking} : set to true if Qeditas should stake
\item \var{ip} : optionally the IP address to listen for incoming connections
\item \var{ipv6} : optionally the IPv6 address to listen for incoming connections
\item \var{port} : the port to listen for incoming connections
\item \var{socks} : None if connections are not routed through SOCKS; Some($v$) if connections are routed through SOCKS protocol $v$ where $v$ is 4 or 5\footnote{At the moment, 5 is not yet supported.}
\item \var{maxconns} : the maximum number of connections
\item \var{seed} : the initial seed which is used to initialized the current stake modifier and future stake modifier.
\item \var{lastcheckpoint} : the last checkpoint (currently unused)
\item \var{randomseed} : an optional string used to seed the OCaml Random module. If {\var{randomseed}} is given it should be cryptographically strong and new each time Qeditas is started. If {\var{randomseed}} is not given, then Random is initialized using data from {\file{/dev/random}}. If {\file{/dev/random}} does not exist (e.g., under MS Windows), {\var{randomseed}} must be given and it is the user's responsibility to ensure {\var{randomseed}} is cryptographically strong and fresh.
\item \var{checkpointskey} : the private key for signing checkpoints (in Qeditas testnet WIF format). Signed checkpoints are only intended for the testnet, and only until the testnet is sufficiently stable. The corresponding public key is $(x,y)$ where
$$
\begin{array}{r}
x = 6371720373269100296662749352347839551092563796413818519910\\ 1530429614494608215 \\
y = 1455153899310935243255864964656407400101005941691152503242\\ 8212764782058901234
\end{array}
$$
These values can be found in {\file{blocktree.ml}} as the
settings for {\var{checkpointspubkeyx}}
and {\var{checkpointspubkeyy}}.
If a future version of Qeditas should use a different signing key for checkpoints, simply update {\var{checkpointspubkeyx}}
and {\var{checkpointspubkeyy}} in {\file{blocktree.ml}} to the new public key values.
\end{itemize}

The functions exposed in the interface \file{setconfig.mli}
override the default compiled settings by reading a configuration file
and checking the command line arguments of \exec{qeditasd} or \exec{qeditascli}.
This is done by calling \func{datadir\_from\_command\_line}
to set \var{datadir} from the command line if the argument \commlinearg{-datadir} was given,
then calling \func{process\_config\_file} to read the \file{qeditas.conf} file in \var{datadir},
and finally calling \func{process\_config\_args} to set the remaining configuration variables
by processing other command line arguments than \commlinearg{-datadir}.

